\documentclass{anablatt}
\blattno4
\begin{document}
\makeheader
\task
\ttask
\begin{theorem}
Sei $X$ ein Messraum und seien $f,g : X \rightarrow \R$ messbar. Dann ist $f \cdot g$ als
\[ f \cdot g: X \rightarrow \R ~,~ x \mapsto f(x) \cdot g(x) \]
ebenfalls messbar.
\end{theorem}
\begin{proof}
Nach Satz 3.3 gibt es zwei Folgen von einfachen Funktionen $(f_n)_{n \in \N}, (g_n)_{n \in \N} \in \qty(X^\R)^\N$ welche jeweils punktweise gegen $f$ und $g$ konvergieren. Man betrache nun ein beliebiges $n \in \N$: Nach Definition lassen gibt es Partitionen $A, B \subset \mathfrak M$ von messbaren Mengen welche jeweils Konstant auf $f$ und $g$ sind.

Man definiere die Partition $R \subset \mathfrak M$ von $X$.
\[
    R \coloneqq \qty{A_i\setminus B \mid A_i \in A} \cup \qty{B_i\setminus A \mid B_i \in B} \cup \qty{A_i \cap B_j \mid A_i \in A~,~ B_j \in B}
\]
Nach Lemma 3.1.1 besteht $R$ wieder aus messbaren Mengen.

Da $f$ und $g$ jeweils konstant auf allen $A_i$ und $B_i$ sind, sind auch alle $R_i \in R$ auf $f \cdot g$ konstant, sodass $f_n \cdot g_n$ eine einfache Funktion ist. Nach Rechenregeln für konvergente Folgen gilt für beliebiges $x \in X$
\[ \lim_{n \rightarrow \infty} f_n(x)\cdot g_n(x) = \qty(\lim_{n \rightarrow \infty}f_n(x)) \cdot \qty (\lim_{n \rightarrow \infty} g_n(x)) = (f \cdot g)(x) \quad.\]
So konvergiert die Folge $(f_n \cdot g_n)_{n\in\N}$ von einfachen Funktionen punktweise gegen $f \cdot g$. Nach Satz 3.3 ist $f \cdot g$ damit messbar.
\end{proof}
\ttask
\begin{theorem}
Sei $(X, \mathfrak M)$ ein Messraum, $(\mu)_{n\in\N}$ eine Folge von Maßen auf $X$ und $(a_n)_{n\in\N} \in \R_{\geq0}^\N$. Sei $\mu : \mathfrak M \rightarrow \R_{\geq0}$ definiert als
\[ \mu(A) \coloneqq \sum_{n=1}^\infty a_n\mu_n(A) \qquad \forall A \in \mathfrak M \quad. \]
Dann ist $\mu$ ein Maß auf $X$.
\end{theorem}
\begin{proof} Es seien $(\mu)_{n\in\N}$, $(a_n)_{n\in\N} \in \R_{\geq0}^\N$ nach Voraussetzung gegeben.
\begin{enumerate}[(i)]
    \item
        \[ \mu(\emptyset) = \sum_{n=1}^\infty a_n\mu_n(\emptyset) 
        = \sum_{n=1}^\infty a_n\cdot0 = 0 \]
    \item Es sei $(A_n)_{n\in\N} \in \mathfrak M^\N$ eine Folge disjunkter messbarer Mengen $A_i \cap A_j = \emptyset ~\forall i,j \in \N~,~i \neq j$
        \[
            \mu\qty(\bigcup_{i \in \N} A_i)
            =\sum_{j\in\N} a_j\mu_j\qty(\bigcup_{i \in \N} A_i)
            =\sum_{j\in\N} a_j\sum_{i \in \N} \mu_j(A_i)
            =\sum_{i\in\N} \sum_{j \in \N} a_j\mu_j(A_i)
            =\sum_{i\in\N} \mu(A_i)
        \]
\end{enumerate}
\end{proof}

\end{document}
