\documentclass{anablatt}
\blattno8

\begin{document}
\makeheader
\setcounter{taski}2
\task
Sei $\qty([0,1],\, \mathcal B\eval_{[0,1]},\, \mu\eval_{[0,1]})$ ein Maßraum. Es wird folgende Definition für die Folge $(A_n)_{n\in\N} \in \mathfrak M^\N$ genutzt:
Sei $(B_n)_{n\in\N} \in \mathfrak M^\N$ eine Folge mit
\begin{align*}
    B_1 &= [0,1] \\
    B_{n+1} &= \qty{\frac x3 ~\Big\vert~ x \in B_n} \cup \qty{\frac{2+x}3 ~\Big\vert~ x \in B_n} \qquad \forall n \in \N \\
    C &\coloneqq \bigcap_{n\in\N} B_n 
\end{align*}

Dann ist $(A_n)_{n\in\N}$ gegeben als
\[ A_n \coloneqq [0,1] \setminus B_n \qquad \forall n\in\N \]
\[ G \coloneqq [0,1] \setminus C \]
Es wird ohne Beweis angenommen dass für jedes $M \in \mathfrak M ~,~ a \in \R$
\begin{gather*}
   \mu(aM) = \mu\qty(\qty{ax \mid x \in M}) = |a|\cdot\mu(M)  \\
   \mu(a+M) = \mu\qty(\qty{a+x \mid x \in M}) = \mu(M)  \\
\end{gather*}
\ttask
Da für alle Mengen $M \subseteq [0,1]$  gilt $\frac13 M \cap (\frac23+\frac13 M) = \emptyset$. Damit folgt für alle $n \in \N$
\[ \mu\qty(B_{n+1}) = \mu\qty(\frac13 B_n + \qty(\frac23+\frac13 B_n)) = \frac23 \mu(B_n) \]
Mit $\mu(B_1) = 1$ gilt als explizite Vorschrift für $n \in \N$
\[ \mu(B_n) = \qty(\frac23)^{n-1} \xrightarrow{n \rightarrow \infty} 0 \]
Damit ist nach $C \sqcup G = [0, 1]$
\[ \mu(G) = 1-\mu(C) = 1 \]
\ttask
\begin{theorem}
    \[ x \in C \quad\Longleftrightarrow\quad
    \exists (a_n)_{n\in\N} \in \{0,2\}^\N : x = \sum_{n=1}^\infty a_n3^{-n} \qquad (x \in [0,1]) \]
\end{theorem}
\newpage
\begin{proof}
\glqq$\Leftarrow$\grqq \quad Sei $(a_n)_{n\in\N} \in \{0,2\}^\N$ beliebig mit $ x = \sum_{n=1}^\infty a_n3^{-n}$

Induktionsanfang: $x \in B_1 = [0,1]$ da
\[ 0 = \sum_{n=1}^\infty 0\cdot3^{-n} \leq x \leq \sum_{n=1}^\infty 2\cdot3^{-n} = -2 + \sum_{n=0}^\infty 2\cdot3^{-n} = -2 + \frac2{1-\frac13} = 1 \]
Induktionsbehauptung: \quad$ x \in B_n$ für beliebiges $(a_n)_{n\in\N}$ \\
Induktionsschritt: Z. z.: \quad $x \in B_{n+1}$ \\

Sei also $(a_n)_{n\in\N} \in \{0,2\}^\N$ beliebig mit $x \coloneqq \sum_{n=1}^\infty a_n3^{-n} \in B_n$.

Man definiere $(a_n')_{n\in\N} ~,~ (a_n'')_{n\in\N} \in \{0,2\}^\N$ mit
\begin{gather*}
    a_1' = 0 \qquad a_1'' = 2 \\ 
    a_n' = a_n'' = a_{n+1}  \quad\forall n \in \N\setminus\{1\}
\end{gather*}
Beide Reihen befinden sich nach Voraussetzung wieder in $B_n$. Durch Konstruktion folgt
\begin{gather*}
    \frac13x = \sum_{n=1}^\infty a_n3^{-(n+1)} = \sum_{n=1}^\infty a_n'3^{-n} \in B_n \\
    \frac23 + \frac13x = 2\cdot3^{-1} + \sum_{n=1}^\infty a_n'3^{-n} = \sum_{n=1}^\infty a_n''3^{-n} \in B_n
\end{gather*}
Nach Definition von $B_{n+1}$ ist demnach jedes $x \in B_{n+1}$ durch eine Potenzreihe aus $\{0,1\}^\N$ darstellbar.
\end{proof}
\end{document}
