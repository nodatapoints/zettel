\documentclass[a4paper, 12pt]{scrartcl}

\usepackage[utf8]{inputenc}
\usepackage[T1]{fontenc}
\usepackage[ngerman]{babel}

\usepackage{amssymb}
\usepackage{amsmath}
\usepackage{physics}
\usepackage{framed}
\usepackage{float}
\usepackage{mathtools}
\usepackage{marvosym}
\usepackage{enumerate}

\usepackage{tikz}
\usepackage{chngcntr}

\usepackage{amsthm}
\usepackage{thmtools}

\usepackage[left=2cm, right=2cm, top=2cm]{geometry}

\allowdisplaybreaks

\setlength{\parindent}{0pt}

\setkomafont{paragraph}{\normalfont\itshape}

\declaretheoremstyle[%
  spaceabove=0,%
  spacebelow=6pt,%
  headfont=\normalfont\itshape,%
  postheadspace=1em,%
  headpunct={}
]{mystyle}

\declaretheorem[name={Behauptung}, style=mystyle, unnumbered]{theorem}
\declaretheorem[name={Lemma}, style=mystyle]{lemma}
\declaretheorem[name={Voraussetzung}, style=mystyle, unnumbered]{precondition}
\let\proof\oldproof
\declaretheorem[name={Beweis}, style=mystyle, qed=\qedsymbol, unnumbered]{proof}

\let\phi\varphi
\let\epsilon\varepsilon
\let\bar\overline

\newcounter{taski}
\newcounter{taskii}[taski]
\newcounter{taskiii}[taskii]

\newcommand{\task}{\stepcounter{taski}\textbf{Aufgabe \arabic{taski}}~}
\newcommand{\ttask}{\stepcounter{taskii}\textbf{(\alph{taskii})}~}
\newcommand{\tttask}{\stepcounter{taskiii}\quad(\roman{taskiii})~}

\begin{document}
\begin{flushright}
    Kamal Abdellatif
\end{flushright}
\begin{center}
    \textbf{Höhere Analysis Abgabeblatt 2}\\[2em]
	\def\arraystretch{2}
    \begin{tabular}{|l|l|l||p{18mm}|}
        \hline
         Aufgabe 1 & Aufgabe 2 & Aufgabe 3 & Summe:~ \\
         \hline &&&\\
         \hline  
    \end{tabular}
\end{center}

\task Sei $E$ ein $\mathbb{R}$-Vektorraum ausgestattet mit der Seminorm $\norm{\cdot}$, $E_0 \coloneqq \qty{x \in E ~\big\vert~ \norm{x} = 0}$. \\

$E_0$ ist abgeschlossen unter Addition und skalarer Multiplikation. Seien $a,\, b \in E_0,\, \lambda \in \mathbb{R}$ beliebig.
\begin{gather*}
    0 \stackrel{\text{Norm}}\leq \norm{a+b} \stackrel{\text{Norm}}\leq \norm{a} + \norm{b} \stackrel{E_0}= 0 + 0 = 0 \\
    \implies \norm{a+b} = 0 ~\xRightarrow{E_0}~ a + b \in E_0 \\
    \norm{\lambda a} = |\lambda| \cdot \norm{a} = |\lambda| \cdot 0 = 0 ~\xRightarrow{E_0}~ \lambda a \in E_0
\end{gather*}
Es gilt weiterhin $0 \in E_0$ sodass $E_0 \neq \emptyset$. Damit ist $E_0$ ein Unterraum von $E$.

Die Norm $\norm{x + E_0} \coloneqq \norm{x} ~,~ x \in E$ auf dem Faktorraum $E/E_0$
\begin{enumerate}[(i)]
\item ist positiv definit da für alle $x \in E$
\begin{gather*}
    \norm{x} \geq 0 \\
    \norm{\bar x} = 0 \implies \norm{x} = 0 \implies x \in E_0 ~,~ x \sim 0 \implies \bar x = \bar 0
\end{gather*}
\item erfüllt die Dreiecksungleichung mit
    \[ \norm{\bar x + \bar y} = \norm{\bar{x+y}} = \norm{x + y} \stackrel{\text{Seminorm}}\leq \norm{x} + \norm{y} = \norm{\bar x} + \norm{\bar y} \quad x,\, y \in E\]
\item und skaliert mit einem Faktor $\lambda \in \mathbb{R}$
    \[ \norm{\lambda \bar x} = \norm{\bar{\lambda x}} = \norm{\lambda x} = |\lambda|\cdot\norm{x} = |\lambda|\cdot\norm{\bar x} \qquad x \in E \]
\end{enumerate}

\task Es seien $E,\,F$ lineare normierte $\mathbb{R}$-Vektorräume, $F$ ein \textsc{Banach}-Raum. $L(E,\,F)$ bezeichne den $\mathbb{R}$-Vektorraum der stetigen linearen Abbildungen $\phi:E \rightarrow F$.
\\

\begin{lemma}
Seien $\phi \in L(E,\,F) ~,~ x_0 \in E$
\[ \norm{\phi(x_0)} \leq \norm{\phi}\cdot\norm{x_0} \]
\end{lemma}
\begin{proof}
\begin{gather*}
    \sup_{x \in E} \frac{\norm{\phi(x)}}{\norm{x}}
    = \sup_{x \in E} \norm{\frac{\phi(x)}{\norm{x}}}
    = \sup_{x \in E} \norm{\phi\qty(\frac{x}{\norm{x}})}
    = \sup_{\substack{x' \in E\\\norm{x'}=1}} \norm{\phi(x')}
    = \norm{\phi}
    \\
    \implies \forall x \in E : \frac{\norm{\phi(x)}}{\norm{x}} \leq \norm{\phi} \\
    \implies \forall x \in E : \norm{\phi(x)} \leq \norm{\phi} \cdot \norm{x}
\end{gather*}
\end{proof}
\end{document}
