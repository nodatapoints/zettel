\documentclass[a4paper, 12pt]{scrartcl}

\usepackage[utf8]{inputenc}
\usepackage[T1]{fontenc}
\usepackage[ngerman]{babel}

\usepackage{amssymb}
\usepackage{amsmath}
\usepackage{physics}
\usepackage{framed}
\usepackage{float}
\usepackage{mathtools}
\usepackage{marvosym}
\usepackage{enumerate}

\usepackage{tikz}
\usepackage{chngcntr}

\usepackage{amsthm}
\usepackage{thmtools}

\usepackage[left=2cm, right=2cm, top=2cm]{geometry}

\allowdisplaybreaks

\setlength{\parindent}{0pt}

\setkomafont{paragraph}{\normalfont\itshape}

\declaretheoremstyle[%
  spaceabove=0,%
  spacebelow=6pt,%
  headfont=\normalfont\itshape,%
  postheadspace=1em,%
  headpunct={}
]{mystyle}

\declaretheorem[name={Behauptung}, style=mystyle, unnumbered]{theorem}
\declaretheorem[name={Lemma}, style=mystyle]{lemma}
\declaretheorem[name={Voraussetzung}, style=mystyle, unnumbered]{precondition}
\let\proof\oldproof
\declaretheorem[name={Beweis}, style=mystyle, qed=\qedsymbol, unnumbered]{proof}

\newcounter{taski}
\newcounter{taskii}[taski]
\newcounter{taskiii}[taskii]

\newcommand{\task}{\stepcounter{taski}\textbf{Aufgabe \arabic{taski}}~}
\newcommand{\ttask}{\stepcounter{taskii}\textbf{(\alph{taskii})}~}
\newcommand{\tttask}{\stepcounter{taskiii}\quad(\roman{taskiii})~}

\begin{document}
\begin{flushright}
    Kamal Abdellatif
\end{flushright}
\begin{center}
    \textbf{Höhere Analysis Abgabeblatt 2}\\[2em]
	\def\arraystretch{2}
    \begin{tabular}{|l|l|l||p{18mm}|}
        \hline
         Aufgabe 1 & Aufgabe 2 & Aufgabe 3 & Summe:~ \\
         \hline &&&\\
         \hline  
    \end{tabular}
\end{center}

\task
$E_0$ ist abgeschlossen unter Addition und skalarer Multiplikation. Seien $a,\, b \in E_0,\, \lambda \in \mathbb{R}$ beliebig.
\begin{gather*}
    0 \stackrel{\text{Norm}}\leq \norm{a+b} \stackrel{\text{Norm}}\leq \norm{a} + \norm{b} \stackrel{E_0}= 0 + 0 = 0 \\
    \implies \norm{a+b} = 0 ~\xRightarrow{E_0}~ a + b \in E_0 \\
    \norm{\lambda a} = |\lambda| \cdot \norm{a} = |\lambda| \cdot 0 = 0 ~\xRightarrow{E_0}~ \lambda a \in E_0
\end{gather*}
Damit ist $E_0$ ein Unterraum von $E$ ???

Die Norm $\norm{x + E_0} \coloneqq \norm{x} ~,~ x \in E$ auf dem Faktorraum $E/E_0$
\begin{enumerate}[(i)]
\item ist positiv definit da für alle $x \in E$
\begin{gather*}
    \norm{x} \geq 0 \\
    \norm{\bar x} = 0 \implies \norm{x} = 0 \implies x \in E_0 ~,~ x \sim 0 \implies \bar x = \bar 0
\end{gather*}
\item erfüllt die Dreiecksungleichung mit
\[ \norm{\bar x + \bar y} = \norm{x + y} \stackrel{\text{Norm}}\leq \norm{x} + \norm{y} = \norm{\bar x} + \norm{\bar y} \quad x,\, y \in E\]
\end{enumerate}
\end{document}
