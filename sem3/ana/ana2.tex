\documentclass{anablatt}
\blattno{2}

\begin{document}
\makeheader

\task Sei $E$ ein \R-Vektorraum ausgestattet mit der Seminorm $\norm{\cdot}$, $E_0 \coloneqq \qty{x \in E ~\big\vert~ \norm{x} = 0}$. \\

$E_0$ ist abgeschlossen unter Addition und skalarer Multiplikation. Seien $a,\, b \in E_0,\, \lambda \in \R$ beliebig.
\begin{gather*}
    0 \stackrel{\text{Norm}}\leq \norm{a+b} \stackrel{\text{Norm}}\leq \norm{a} + \norm{b} \stackrel{E_0}= 0 + 0 = 0 \\
    \implies \norm{a+b} = 0 ~\xRightarrow{E_0}~ a + b \in E_0 \\
    \norm{\lambda a} = |\lambda| \cdot \norm{a} = |\lambda| \cdot 0 = 0 ~\xRightarrow{E_0}~ \lambda a \in E_0
\end{gather*}
Es gilt weiterhin $0 \in E_0$ sodass $E_0 \neq \emptyset$. Damit ist $E_0$ ein Unterraum von $E$.

Die Norm $\norm{x + E_0} \coloneqq \norm{x} ~,~ x \in E$ auf dem Faktorraum $E/E_0$
\begin{enumerate}[(i)]
\item ist positiv definit da für alle $x \in E$
\begin{gather*}
    \norm{x} \geq 0 \\
    \norm{\bar x} = 0 \implies \norm{x} = 0 \implies x \in E_0 ~,~ x \sim 0 \implies \bar x = \bar 0
\end{gather*}
\item erfüllt die Dreiecksungleichung mit
    \[ \norm{\bar x + \bar y} = \norm{\bar{x+y}} = \norm{x + y} \stackrel{\text{Seminorm}}\leq \norm{x} + \norm{y} = \norm{\bar x} + \norm{\bar y} \quad x,\, y \in E\]
\item und skaliert mit einem Faktor $\lambda \in \R$
    \[ \norm{\lambda \bar x} = \norm{\bar{\lambda x}} = \norm{\lambda x} = |\lambda|\cdot\norm{x} = |\lambda|\cdot\norm{\bar x} \qquad x \in E \]
\end{enumerate}
\task Es seien $E,\,F$ lineare normierte $\R$-Vektorräume, $F$ ein \textsc{Banach}-Raum. $L(E,\,F)$ bezeichne den $\R$-Vektorraum der stetigen linearen Abbildungen $\phi:E \rightarrow F$.
\\

\begin{lemma}
Seien $\phi \in L(E,\,F) ~,~ x_0 \in E$
\[ \norm{\phi(x_0)} \leq \norm{\phi}\cdot\norm{x_0} \]
\end{lemma}
\begin{proof}
\begin{gather*}
    \sup_{x \in E} \frac{\norm{\phi(x)}}{\norm{x}}
    = \sup_{x \in E} \norm{\frac{\phi(x)}{\norm{x}}}
    = \sup_{x \in E} \norm{\phi\qty(\frac{x}{\norm{x}})}
    = \sup_{\substack{x' \in E\\\norm{x'}=1}} \norm{\phi(x')}
    = \norm{\phi}
    \\
    \implies \forall x \in E : \frac{\norm{\phi(x)}}{\norm{x}} \leq \norm{\phi} \\
    \implies \forall x \in E : \norm{\phi(x)} \leq \norm{\phi} \cdot \norm{x}
\end{gather*}
\end{proof}
\newpage
\begin{theorem}
    $L(E,\, F)$ unter Standard-Norm $\norm{\cdot}$ ist ein \textsc{Banach}-Raum.
\end{theorem}
\begin{proof}
Sei $(\phi_n)_{n \in \N}$ eine \textsc{Cauchy}-Folge in $L(E,\, F)$. Sei $x_0 \in E$ fest, $\epsilon > 0$ beliebig. Somit gibt es ein $N \in \N$ sodass für alle $n,\, m \in \N ~,~ N < n,\, m$ gilt
\[
    \norm{x_0}\cdot\epsilon
    \stackrel{\textsc{Cauchy}}> \norm{x_0}\cdot\norm{\phi_m-\phi_n} 
    \stackrel{\text{Lemma 1}}\geq \norm{(\phi_m-\phi_n)(x_0)}
    \stackrel{L \text{ linear}}= \norm{\phi_m(x_0) - \phi_n(x_0)}
\]
Mit $\epsilon' = \frac\epsilon{\norm{x_0}}$ ist die Folge $\qty(\phi_n(x_0))_{n \in \N}$ demnach \textsc{Cauchy}-Folge in $F$. Da $F$ vollständig ist, existiert $\lim_{n \rightarrow \infty} \phi_n(x_0)$. Da $x_0$ anfänglich fest aber beliebig war, kann der Grenzwert für $\phi_n$ festgelegt werden als
\[ \lim_{n \rightarrow \infty}\phi_n \quad\coloneqq\quad \phi : E \rightarrow F ~,~ x \mapsto \lim_{n \rightarrow \infty} \phi_n(x) \]
Die Linearität von $\phi$ folgt direkt aus den Rechenregeln für konvergente Folgen. Es muss noch die Stetigkeit gezeigt werden: $\qty(\norm{\phi_n})_{n \in \N}$ ist eine \textsc{Cauchy}-Folge in $\R$ da nach umgestellter Dreiecksungleichung
\[ \abs\big{\norm{\phi_m} - \norm{\phi_n}} = \norm{\phi_m} - \norm{\phi_n} \leq \norm{\phi_m - \phi_n} \qquad m,\, n \in \N ~,~ \norm{\phi_m} > \norm{\phi_n} \text{ o.B.d.A}\]
Da $\R$ vollständig ist existiert der Grenzwert $\norm{\phi} \coloneqq \lim_{n \rightarrow \infty} \norm{\phi_n}$. Weiterhin ist die Norm $\norm{\cdot} : L(E,\, F) \rightarrow \R$ stetig. Demnach gilt für beliebiges $x \in E$
\[
    \norm{\phi(x)} = \norm{\lim_{n\rightarrow\infty}\phi_n(x)}
    \stackrel{\text{stetig}}= \lim_{n\rightarrow\infty}\norm{\phi_n(x)} 
    \stackrel{\text{Lemma 1}}\leq \norm{x}\lim_{n\rightarrow\infty}\norm{\phi_n}
    = \norm{x}\cdot\norm{\phi}
\]
Damit ist nach Satz 2.1 der Vorlesung $\phi$ stetig und schlussendlich $\phi \in L(E,\, F)$.
\end{proof}
\task Sei $\mathscr{S} \subset \R$ die Menge der höchstens abzählbaren Teilmengen von $\R$.

\begin{lemma}
Folgende Aussagen werden ohne Beweis genutzt. (Siehe Anhang)
\begin{enumerate}[(i)]
    \item Die Vereinigung höchstens abzählbar vieler höchstens abzählbaren Mengen ist wieder höchstens abzählbar.
    \item Der Schnitt höchstens abzählbar vieler höchstens abzählbaren Mengen ist wieder höchstens abzählbar.
    \item Jede Teilmenge einer höchstens abzählbaren Menge ist wieder höchstens abzählbar.
\end{enumerate}
\end{lemma}
\begin{theorem}
Die von $\mathscr{S}$ erzeugte $\sigma$-Algebra $\mathfrak{M}$ besteht genau aus allen Teilmengen von \R die selbst höchstens abzählbar sind oder deren Komplement höchstens abzählbar ist.
\end{theorem}
\def\S{\mathscr{S}}
\def\frN{\ensuremath{\mathfrak{N}} }
\def\frM{\ensuremath{\mathfrak{M}} }
\begin{proof}
1. Teil: Z. z.: Ist $\mathfrak{N} \subset \mathcal{P}(\R)$ eine $\sigma$-Algebra mit $\mathscr{S} \subseteq \mathfrak{N}$ so folgt $\mathfrak{N} \subseteq \mathfrak{M}$, wobei $\mathfrak{M}$ gegeben ist als
\[ \frM \coloneqq \S ~\cup~ \qty{X^c \mid X \in \S} \]
Sei also eine $\sigma$-Algebra \frN mit $\S \subseteq \frN$ gegeben. Sei $X \in \S$; dann ist nach Voraussetzung auch $X \in \frN$. Da \frN abgeschlossen unter Komplementbildung ist muss demnach auch $X^c \in \frN$. Damit gilt nach Definition von \frM
\[ \frM \subseteq \frN \]
2. Teil: Z. z.: \frM ist eine $\sigma$-Algebra.

\frM ist nach Definition abgeschlossen unter Komplementbildung. Man betrachte nun die Vereinigung einer höchstens abzählbaren Familie $\mathfrak{F} = \qty{X_n}_{n \in \N} \in \mathcal{P}(\frM)$. (Endliche Vereinigungen können hier durch die unendliche Wiederholung des letzten Elements ebenfalls über abzählbarer Familien dargestellt werden.)

Nach Assoziativität der Vereinigung und der Definition von \frM kann diese Vereinigung in zwei höchstens abzählbare Vereinigungen höchstens abzählbarer Mengen aufgeteilt werden.
\begin{align*}
    \bigcup_{n \in \N} X_n &= \bigcup_{\substack{n \in \N\\X_n \in \S}}X_n ~\cup~ \bigcup_{\substack{n \in \N\\X_n^c \in \S}}X_n
    = \bigcup_{\substack{n \in \N\\X_n \in \S}}X_n ~\cup~ \bigcup_{\substack{n \in \N\\X_n^c \in \S}}\qty(X_n^c)^c \\
    &= \overbrace{\bigcup_{\substack{n \in \N\\X_n \in \S}}X_n}^{\eqqcolon A} ~\cup~ \overbrace{\qty\Big(\bigcap_{\substack{n \in \N\\X_n^c \in \S}}X_n^c)^c}^{\eqqcolon B^c}
    = A \cup B^c = \qty(B \cap A^c)^c = (B \setminus A)^c
\end{align*}
Nach Lemma 2 sind sowohl $A$ als auch $B$ höchstens abzählbare Mengen, also $A,\,B \in \S$.
\[
    B \setminus A \subseteq B \quad \xRightarrow{\text{Lemma 2}} \quad B \setminus A \in \S
    \quad \xRightarrow{\text{Def. \frM}} \quad \bigcup_{n \in \N} X_n = (B \setminus A)^c \in \frM
\]
Somit ist \frM unter höchstens abzählbarer Vereinigung abgeschlossen.

Da nach Definition von Abzählbarkeit $\emptyset \in \S \subseteq \frM$ ist \frM damit eine $\sigma$-Algebra.
\end{proof}
\newpage
\textbf{Anhang} \quad Es wurde in Analysis I auf dem 3. Übungsblatt bereits gezeigt dass die Vereinigung abzählbar vieler abzählbarer Mengen ebenfalls abzählbar ist. Im Folgenden ist nochmals mein Beweis dargestellt. \\

\begin{theorem}
Für eine Folge $(M_n)_{n \in \mathbb{N}}$
	\[ \forall n \in \mathbb{N} : M_n\ \text{abzählbar} \Rightarrow \bigcup_{n \in \mathbb{N}}M_n\quad\text{abzählbar}\]
\end{theorem}
\begin{proof}
\[ \txtimpl{Def. abz.} \forall n \in \mathbb{N}\ \exists \varphi_n:\mathbb{N} \rightarrow M_n\ ,\ \varphi_n\ \text{bijektiv} \]
Es sei
\[ \sigma : \mathbb{N}^2 \rightarrow \bigcup_{n \in \mathbb{N}}M_n\ ,\ (i,\ j) \mapsto \varphi_i(j) \]
$\sigma$ ist nach Konstruktion surjektiv. Da nach Satz 1.5.3 $\mathbb{N}^2$ abzählbar ist, sodass eine Bijektion $\tau : \mathbb{N} \rightarrow \mathbb{N}^2$ existiert.
\[ \Rightarrow \sigma \circ \tau : \mathbb{N} \rightarrow \bigcup_{n \in \mathbb{N}}M_n \quad \text{surjektiv} \]
Nach Lemma \ref{surj} ist $\bigcup_{n \in \mathbb{N}}M_n$ somit abzählbar.
\end{proof}

\begin{lemma}\label{surj}
	Gibt es für eine Menge $M$ eine surjektive Abbildung $\varphi:\mathbb{N} \rightarrow M$, so ist $M$ abzählbar.	
\end{lemma}
\begin{proof}
\begin{gather*}
	\txtimpl{Def. Surj.} \forall x \in M \ \exists n \in \mathbb{N} : \varphi(n) = s \\
	\Rightarrow \forall x \in M : \varphi^{-1}(\{x\}) \neq \emptyset
\end{gather*}
Es sei nun eine Abbildung $\theta: M \rightarrow \mathbb{N}$, welche ein $x \in M$ auf das kleinste Element in seinem Urbild abbildet. Es existiert ein eindeutiges kleinstes Element, da $\varphi^{-1}(\{x\}) \subseteq \mathbb{N}$.
\[ \theta(x) \coloneqq \min\big(\varphi^{-1}(\{x\})\big) \]
Surjektivität bleibt nach Konstruktion erhalten, und aus der Eindeutigkeit des kleinsten Elements folgt
\[ \forall x,\ y \in M : \min\big(\varphi^{-1}(\{x\})\big) = \min\big(\varphi^{-1}(\{y\})\big) \eqqcolon n \Rightarrow x = \varphi(n) = y\]
sodass $\theta$ injektiv und so bijektiv ist.
\[ \defimpl{\sim} M \sim \mathbb{N} \txtimpl{Def. abz.} M\ \text{abzählbar} \]
\end{proof}

\end{document}
