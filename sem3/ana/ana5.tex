\documentclass{anablatt}
\blattno5

\begin{document}
\makeheader
\task
Sei $(\N, \N, \mu)$ mit $\mu: \N \rightarrow [0,\infty] ~,~ A \mapsto |A|$ ein Maßraum.
\begin{theorem}
Eine Funktion $f:\N \rightarrow \R$ ist genau dann integrierbar bezüglich $\mu$ wenn $\sum_{n=1}^\infty f(n)$ absolut konvergent ist. Ist dies der Fall, so gilt
\[ \int_\N f\dd\mu = \sum_{n=1}^\infty f(n)  \quad.\]
\end{theorem}
\begin{proof}
Man betrachte die Folge $(t_n)_{n\in\N} \in T(\N,\R)^\N$. Für alle $n \in \N$ sei $\mathfrak P_n = \{\{i\} \mid i \in \range{1,n}\}$ eine Partition von messbaren Mengen der Menge $\range{1,n} \subset \N$. Die entsprechende Treppenfunktion $t_n$ soll gegeben sein als
\[
    t_n: \N \rightarrow \R ~,~
    i \mapsto \begin{cases}
        f(i) \quad & i \leq n \\
           0 \quad & i > n
       \end{cases} \quad.
\]
Damit ist $t_n\eval_A$ nach Definition konstant auf allen $A \in \mathfrak P_n$. Durch Konstruktion approximiert $(t_n)_{n\in\N}$ die Funktion $f$. Damit gilt
\begin{align*}
    \infty \stackrel{\mu\text{ int'bar}}> \int_\N f\dd\mu
    &= \lim_{n \rightarrow \infty} \int_\N t_n\dd\mu 
    = \lim_{n \rightarrow \infty} \sum_{\{i\}\in\mathfrak P_n} \overbrace{\mu(\{i\})}^1t_n(i)
    = \lim_{n \rightarrow \infty} \sum_{i=1}^n f(i)
    = \sum_{i=1}^\infty f(i)
\end{align*}
Die Rückrichtung gilt gleichermaßen, da das Integral wohldefiniert ist und somit unabhängig von der Wahl der approximierenden Folge.
\end{proof}

\task nicht bearbeitet

\task nicht bearbeitet
\end{document}
