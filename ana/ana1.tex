\documentclass[a4paper, 12pt]{scrartcl}

\usepackage[utf8]{inputenc}
\usepackage[T1]{fontenc}
\usepackage[ngerman]{babel}

\usepackage{amssymb}
\usepackage{amsmath}
\usepackage{framed}
\usepackage{float}
\usepackage{mathtools}

\usepackage{tikz}

\usepackage{amsthm}
\usepackage{thmtools}

\usepackage[left=2cm, right=2cm, top=2cm]{geometry}

\allowdisplaybreaks

\setlength{\parindent}{0pt}

\declaretheoremstyle[%
  spaceabove=0,%
  spacebelow=6pt,%
  headfont=\normalfont\itshape,%
  postheadspace=1em,%
  headpunct={}
]{mystyle}

\declaretheorem[name={Behauptung}, style=mystyle, unnumbered]{theorem}
\declaretheorem[name={Lemma}, style=mystyle]{lemma}
\declaretheorem[name={Voraussetzung}, style=mystyle, unnumbered]{precondition}
\let\proof\oldproof
\declaretheorem[name={Beweis}, style=mystyle, qed=\qedsymbol, unnumbered]{proof}

\newcounter{taski}
\newcounter{taskii}[taski]
\newcounter{taskiii}[taskii]

\newcommand{\task}{\stepcounter{taski}\textbf{Aufgabe \arabic{taski}}\\}
\newcommand{\ttask}{\stepcounter{taskii}\textbf{(\alph{taskii})}\par}
\newcommand{\tttask}{\stepcounter{taskiii}\quad(\roman{taskiii})\par}

\begin{document}
\begin{lemma}
\begin{equation}
	m = n \Rightarrow m+r = n+r \qquad (m,\ n,\ r \in \mathbb{N})
\end{equation}
\end{lemma}
\begin{theorem}
Für alle $m,\ n,\ p \in \mathbb{N}$ gelte
\begin{equation}\label{beh}
	m < n \:\Rightarrow\: p \cdot m < p \cdot n \quad .
\end{equation}
\end{theorem}
\begin{proof}
Sei für alle $m,\ n \in \mathbb{N}$ die Menge $\mathcal{M}$
\[ \mathcal{M} = \left\{ p \:\vert\: p \cdot m < p \cdot n\ ,\ p \in \mathbb{N} \right\} \quad. \]
Aus $m < n$ folgt als für $p = 1$ die Behauptung \eqref{beh}
\begin{equation}\label{IA}
	m \cdot 1 < n \cdot 1 \:\Rightarrow\: 1 \in \mathcal{M} \quad.	
\end{equation}
Gilt \eqref{beh} für ein beliebiges $p$, so existiert ein $r \in \mathbb{N}$ mit
\begin{align}
	m \cdot p &< n \cdot p \nonumber\\
	m \cdot p + r &= n \cdot p \nonumber\\
	m \cdot p + r + n &= n \cdot p + n \qquad. \nonumber
\intertext{Da $m < n$ , existiert ein $r' \in \mathbb{N}$ mit $m + r' = n$ .}
	m \cdot p + m + r + r' &= n \cdot p + n \nonumber\\
	m \cdot (p + 1) + r + r' &= n \cdot (p+1) \tag*{$r + r' \in \mathbb{N}$} \nonumber\\
	m \cdot (p + 1) &< n \cdot (p+1) \label{IS}
\end{align}
Aus \eqref{IA} als Induktionsanfang und \eqref{IS} als Induktionsschritt folgt dass $\mathcal{M} = \mathbb{N}$ .
\end{proof}
\end{document}