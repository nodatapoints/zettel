\documentclass[a4paper, 12pt]{scrartcl}

\usepackage[utf8]{inputenc}
\usepackage[T1]{fontenc}
\usepackage[ngerman]{babel}

\usepackage{amssymb}
\usepackage{amsmath}
\usepackage{framed}
\usepackage{float}
\usepackage{mathtools}

\usepackage{tikz}

\usepackage{marvosym}
\usepackage{amsthm}
\usepackage{thmtools}
\usepackage{marvosym}

\usepackage[left=2cm, right=2cm, top=1cm]{geometry}

\allowdisplaybreaks
\setkomafont{paragraph}{\normalfont\itshape}

\setlength{\parindent}{0pt}

\declaretheoremstyle[%
  spaceabove=0,%
  spacebelow=6pt,%
  headfont=\normalfont\itshape,%
  postheadspace=1em,%
  headpunct={}
]{mystyle}

\declaretheorem[name={Behauptung}, style=mystyle, unnumbered]{theorem}
\declaretheorem[name={Lemma}, style=mystyle]{lemma}
\declaretheorem[name={Voraussetzung}, style=mystyle, unnumbered]{precondition}
\let\proof\oldproof
\declaretheorem[name={Beweis}, style=mystyle, qed=\qedsymbol, unnumbered]{proof}

\newcounter{taski}
\newcounter{taskii}[taski]
\newcounter{taskiii}[taskii]


\newcommand{\task}{\stepcounter{taski}\textbf{Aufgabe \arabic{taski}}\\}
\newcommand{\ttask}{\stepcounter{taskii}\textbf{(\alph{taskii})}\par}
\newcommand{\tttask}{\stepcounter{taskiii}\quad(\roman{taskiii})\par}

\newcommand{\defimpl}[1]{\stackrel{\text{Def.}\;#1}{\Longrightarrow}}
\newcommand{\defImpl}[1]{\stackrel{\text{Def.}\;#1}{\Longleftrightarrow}}
\newcommand{\txtimpl}[1]{\stackrel{\text{#1}}{\Longrightarrow}}
\newcommand{\txtImpl}[1]{\stackrel{\text{#1}}{\Longleftrightarrow}}

\begin{document}
\begin{flushright}
	Kamal\\
	Maximilian Neumann
\end{flushright}
\begin{center}
	\bfseries Analysis I Blatt 2
\end{center}
\task
\ttask
\begin{theorem}
Für endliche Mengen $M,\ N$ gilt
	\[ M \sim N\ \Leftrightarrow |M| = |N|  \]
\end{theorem}
\begin{proof}
Aus Def. | | seien $M \sim A_{|M|}$ , $N \sim A_{|N|}$
\[ M \sim N \txtImpl{Trans. $\sim$} A_{|M|} \sim A_{|N|} \txtImpl{Satz 1.4.1} |M| = |N| \]
\end{proof}
\ttask
\begin{theorem}
	Ist eine Menge $M$ endlich, so ist jede Teilmenge $M' \in \mathcal{P}(M)$ ebenfalls endlich, und es gilt $|M'| \leq |M|$.
\end{theorem}
\begin{proof}
Nach Def. Endl. folgt $\forall M' \in \mathcal{P}(M)\ \exists A_n \sim M\ \text{mit bijektivem}\ \varphi:M \rightarrow A_n$

Sei $S \coloneqq \varphi(M')$ , $S \in \mathcal{P}(A_n)$ für ein beliebiges $M' \in \mathcal{P}(M)$
\begin{equation}\label{phiprime}
	\defimpl{\varphi} \varphi\rvert_{M'}:M' \rightarrow S\ \text{ist bijektiv}
\end{equation}
Nach Lemma \ref{lemma1} ist $S$ endlich mit bijektivem $f:S \rightarrow A_p$ , $|S| = p \leq n = |M|$
\[ \txtimpl{\eqref{phiprime}} f \circ \varphi\rvert_{M'}: M' \rightarrow A_p\ \text{ist bijektiv} \txtimpl{Def. Endl.} M'\ \text{ist endlich}\:\wedge\: |M'| = p \leq |M|\]
\end{proof}
\begin{lemma}\label{lemma1}
	Jede Teilmenge $S \in \mathcal{P}(A_n)$ eines endlichen $A_n$ ist endlich, und es gilt $|S| \leq n$.
\end{lemma}
\begin{proof}\ \vspace{-1em}
\paragraph*{Induktionsanfang} Alle Teilmengen $S \in \mathcal{P}(\emptyset) = \left\{ \emptyset \right\}$ sind endlich. Es gilt $|\emptyset| = 0 \leq 0$
\paragraph*{Induktionsbehauptung} $\forall S \in \mathcal{P}(A_k):S$ ist endlich $\wedge\: |S| \leq k$
\paragraph*{Induktionsschritt} Z. z. $\forall S' \in \mathcal{P}(A_{k+1}): S'$ ist endlich $\wedge\: |S'| \leq k+1$
\begin{gather*}
	\mathcal{P}(A_{k+1}) \defimpl{A} \mathcal{P}\left( A_k \cup \{k+1\} \right)
	\defimpl{\mathcal{P}} \mathcal{P}(A_k) \:\cup\: \left\{ S \cup \{k+1\} \mid S \in \mathcal{P}(A_k) \right\} \\
	\defimpl{\cup} \forall S' \in \mathcal{P}(A_{k+1}): \big( S' \in \mathcal{P}(A_k) \ \vee\ S' \in \left\{ S \cup \{k+1\} \mid S \in \mathcal{P}(A_k) \right\} \big) \\
	S' \in \mathcal{P}(A_k) \txtimpl{IB} S'\ \text{ist endlich}\:\wedge\: |S'| \leq k \leq k+1
\end{gather*}
Nach Def. Endl. folgt $\quad\forall S \in \mathcal{P}(A_k)\ \exists A_p \sim S\ \text{mit bijektivem}\ \varphi:S \rightarrow A_p$

Es sei für ein beliebiges $S \in \mathcal{P}(A_k)$ ein $\varphi': S \cup \{k+1\} \rightarrow A_{p+1}$ definiert als
\[ \varphi'(i) = \begin{cases}
	{p+1} &\quad i = k + 1 \\
	\varphi(i) &\quad \text{sonst}
\end{cases} \]
Da $\varphi'\rvert_S = \varphi$ nach IB bijektiv ist und $k+1$ eineindeutig auf $p+1$ abgebildet wird, ist $\varphi'$ ebenfalls bijektiv. Demnach ist $S \cup \{k+1\} \sim A_{p+1}$ und somit endlich. Nach Def. $|\ |$ ist
\[ \left\lvert S \cup \{k+1\} \right\rvert = p+1 = |S|+1 \stackrel{\text{IB}}{\leq} k+1 \]
\end{proof}
\newpage
\begin{theorem}
Für endliche Mengen $M,\ N$ gilt
\[ M \subseteq N \:\wedge\: |M| = |N| \ \Leftrightarrow\ M = N \]
\end{theorem}
\begin{precondition}
Für endliche Mengen $M,\ N$
\begin{align*}
	 M = N &\Leftrightarrow M \cup \{e\} = N \cup \{e\} & |M| = |N| &\Leftrightarrow |M \cup \{e\}| = |N \cup \{e\}| \\
	 M \subseteq N &\Leftrightarrow M \cup \{e\} \subseteq N \cup \{e\} & e \notin M &\Leftrightarrow |M\cup \{e\}| = |M| + 1
\end{align*}
\end{precondition}
\begin{proof}\ \vspace{-1em}
\paragraph*{Induktionsanfang} $\emptyset \subseteq \emptyset$ , $0 = 0$ , $\emptyset = \emptyset$
\paragraph*{Induktionsbehauptung} $M \subseteq N \:\wedge\: |M| = |N| = n \ \Leftrightarrow\ M = N$
\paragraph*{Induktionsschritt} Z. z. $M' \subseteq N' \:\wedge\: |M'| = |N'| = n+1 \ \Leftrightarrow\ M' = N'$

Es sei $e \notin M,\ N$ ein beliebiges Element, $M' \coloneqq M \cup \{e\}$ , $N' \coloneqq N \cup \{e\}$

Demnach ist nach Vorr.
\begin{gather*}
	|M'| = |M|+1 = n+1 \qquad |N'| = |N| + 1 = n+1 \\
	|M'| = |N'| \:\wedge\: M' \subseteq N' \Leftrightarrow |M| = |N| \:\wedge\: M \subseteq N \Leftrightarrow M = N \Leftrightarrow M' = N'
\end{gather*}
\end{proof}
\paragraph*{Anmerkung} Nach mehreren Tagen Auseinandersetzung war es mir nicht möglich, einen vollständig schlüssigen Beweis zu führen. Ich bin mir durchaus bewusst, dass die Vorraussetzungen keineswegs trivial sind und selbst Beweise erfordern, welche ich aus Zeitgründen hier nicht angeben konnte. Jede endliche Menge lässt sich auf diese Weise konstruieren, da das neue Element über die Umkehrfunktion der Bijektion $\varphi: M' \rightarrow A_{n+1}$ als $e = \varphi^{-1}(n+1)$ erhalten werden kann, während $\varphi^{-1}(A_n) = M$ bestehen bleibt.

\task
\begin{theorem}
\[ f:\mathbb{N}^2 \rightarrow \mathbb{N}\ ,\ (n,\ m) \mapsto 2^n3^m \qquad\text{ist injektiv} \]
\end{theorem}
\begin{proof}
Nach Eindeutigkeit der Primzahlfaktorzerlegung unterscheiden sich zwei Zahlen genau dann, wenn sie sich in den Exponenten ihrer Primzahlfaktorzerlegung unterscheiden. Demnach gilt für $a = 2^n3^m$ , $b = 2^{n'}3^{m'}$
\[ a \neq b \Leftrightarrow (n \neq n' \vee m \neq m') \txtImpl{Def. Tupel} (m,\ n) \neq (m',\ n') \]
\end{proof}
\begin{theorem}
	\[ \mathbb{N}^2 \sim \mathbb{N} \]
\end{theorem}
\begin{proof}
$\mathbb{N} \sim \mathbb{N}$, und das Bild $f(\mathbb{N}^2)$ ist nach Def. $f$ Teilmenge von $\mathbb{N}$
\[ f(\mathbb{N}^2) \subseteq \mathbb{N} \ \txtimpl{Satz 1.5.1}\ f(\mathbb{N}^2) \sim \mathbb{N} \]
Sei $g:\mathbb{N}^2 \rightarrow f(\mathbb{N}^2)$ , $(m,\ n) \mapsto f(m,\ n)$

Da $f$ injektiv ist, ist auch $g$ injektiv. Nach Konstruktion ist $g$ surjektiv. Demnach ist $g$ bijektiv, und $f(\mathbb{N}^2) \sim \mathbb{N}^2$
\[ f(\mathbb{N}^2) \sim \mathbb{N}^2\:\wedge\: f(\mathbb{N}^2) \sim \mathbb{N}\ \txtimpl{Trans. $\sim$}\ \mathbb{N}^2 \sim \mathbb{N}\]
\end{proof}
\newpage
\task
\ttask
\begin{theorem}
\[ \left\{ n \mid n \in \mathbb{N}\:\wedge\: n^2 < 2^n \right\} = \mathbb{N}\backslash\{2,\ 3,\ 4\} \]
\end{theorem}
\begin{proof}
\begin{gather*}
	1^2 < 2^1 \qquad 2^2 \nless 2^2 \qquad 3^2 \nless 2^3 \qquad 4^2 \nless 2^4
\end{gather*}
\paragraph*{Induktionsanfang} $5^2 < 2^5 \:\wedge\: 4 < 5$
\paragraph*{Induktionsbehauptung} $n^2 < 2^n \:\wedge\:4 < n$
\paragraph*{Induktionsschritt} Z. z. $(n+1)^2 < 2^{n+1}$
\begin{align*}
	n^2 &< 2^n \\
	2n^2 &< 2^{n+1} \\
	(n+1)^2 &< 2^{n+1} \tag{nach Lemma \ref{lemma2}, da $3 < 4 < n$}
\end{align*}
\end{proof}
\begin{lemma}\label{lemma2}
	\[ \forall n \in \mathbb{N} : 3 < n \Rightarrow (n+1)^2 < 2n^2 \]
\end{lemma}
\begin{proof}
\begin{align*}
	2 &< 3 < n \leq n^2 \\
	n^2 + 4n + 4 &< 2n^2 + 4n < 2n^2 + 4n + 2 \\
	(n+2)^2 &< 2(n+1)^2
\end{align*}
Es gilt für $n+1$ wenn $2 < n$, sodass es für $n$ gilt wenn $3 < n$.
\end{proof}

\ttask
Es wird davon ausgegangen, dass für ein beliebiges $n$ mit $X$ , $|X| = n+1$ ein $A,\ B \subsetneq X$ ausgewählt werden kann sodass
\[ A \cup B = X \qquad |A| = |B| = n \qquad A \cap B \neq \emptyset \]
Dies ist jedoch nicht der Fall, da für $n = 2$ mit $X = \left\{ e_1,\ e_2 \right\}$ , $e_1 \neq e_2$ o.\,B.\,d.\,A
\begin{gather*}
	\left\{ M \mid M \subsetneq X \right\} = \left\{ \emptyset,\ \{e_1\},\ \{e_2\}\right\}
\end{gather*}
alle echten Teilmengen von $X$ paarweise disjunkt sind, sodass $ \quad\forall A,\ B \subsetneq X : A \cap B = \emptyset $
\vfill
\Large
\begin{tabular}{c|c|c|c}
	1 & 2 & 3 & $\Sigma$ \\
	\hline
	\ \ /\:\ \ & \ \ /\:\ \ & \ \ /\:\ \ & \ \ /\:\ \ \\
\end{tabular}
\end{document}
