\documentclass[a4paper, 12pt]{scrartcl}

\usepackage[utf8]{inputenc}
\usepackage[T1]{fontenc}
\usepackage[ngerman]{babel}

\usepackage{amssymb}
\usepackage{amsmath}
\usepackage{framed}
\usepackage{float}
\usepackage{mathtools}

\usepackage{tikz}

\usepackage{amsthm}
\usepackage{thmtools}
\usepackage{marvosym}

\usepackage{url}
\usepackage[hidelinks]{hyperref}

\usepackage[left=1.8cm, right=1.8cm, top=1cm]{geometry}

\allowdisplaybreaks
\setkomafont{paragraph}{\normalfont\itshape}

\setlength{\parindent}{0pt}

\declaretheoremstyle[%
  spaceabove=0,%
  spacebelow=6pt,%
  headfont=\normalfont\itshape,%
  postheadspace=1em,%
  headpunct={}
]{mystyle}

\declaretheorem[name={Behauptung}, style=mystyle, unnumbered]{theorem}
\declaretheorem[name={Lemma}, style=mystyle]{lemma}
\declaretheorem[name={Voraussetzung}, style=mystyle, unnumbered]{precondition}
\let\proof\oldproof
\declaretheorem[name={Beweis}, style=mystyle, qed=\qedsymbol, unnumbered]{proof}

\newcounter{taski}
\newcounter{taskii}[taski]
\newcounter{taskiii}[taskii]


\newcommand{\task}{\stepcounter{taski}\textbf{Aufgabe \arabic{taski}}\\}
\newcommand{\ttask}{\stepcounter{taskii}\textbf{(\alph{taskii})}\par}
\newcommand{\tttask}{\stepcounter{taskiii}\quad(\roman{taskiii})\par}

\newcommand{\defimpl}[1]{\stackrel{\text{Def.}\;#1}{\Longrightarrow}}
\newcommand{\defImpl}[1]{\stackrel{\text{Def.}\;#1}{\Longleftrightarrow}}
\newcommand{\txtimpl}[1]{\stackrel{\text{#1}}{\Longrightarrow}}
\newcommand{\txtImpl}[1]{\stackrel{\text{#1}}{\Longleftrightarrow}}
\newcommand{\refimpl}[1]{\txtimpl{\eqref{#1}}}
\newcommand{\refImpl}[1]{\txtImpl{\eqref{#1}}}

\begin{document}
\begin{flushright}
	Kamal\\
	Maximilian Neumann
\end{flushright}
\begin{center}
	\bfseries Analysis I Blatt 6
\end{center}
\task
\ttask

\begin{theorem}
	Sei $(a_n)_{n \in \mathbb{N}}$ eine reell-wertige Folge
	\[ \lim_{n \rightarrow \infty} a_{2n} = \lim_{n \rightarrow \infty} a_{2n+1} = a \quad \Rightarrow \quad \lim_{n \rightarrow \infty} a_n = a \]
\end{theorem}
\begin{proof}
\[ \defimpl{\lim} \quad \forall \varepsilon \in \mathbb{R}^+\ \exists k,\ \ell \in \mathbb{N}\ \forall n \in \mathbb{N} : (2k<n \vee 2\ell+1<n) \Rightarrow |a_n - a| < \varepsilon \]
Seien beliebige $\varepsilon,\ k,\ \ell$ welche obige Bedingung erfüllen gegeben.
\begin{gather*}	
	n_0 \coloneqq \max(2k,\ 2\ell+1) + 1 \qquad 2k < n_0\ ,\ 2\ell+1 < n_0\\
	\Longrightarrow \quad \forall \varepsilon \in \mathbb{R}^+\ \exists n_0 \in \mathbb{N}\ \forall n \in \mathbb{N} : n_0 < n \Rightarrow |a_n - a| < \varepsilon \\
	\defimpl{\lim} \quad \lim_{n \rightarrow \infty} a_n = a
\end{gather*}
\end{proof}

\ttask

$(a_{6n})_{n \in \mathbb{N}}$ ist eine Teilfolge von $(a_{2n})_{n \in \mathbb{N}}$ und $(a_{3n})_{n \in \mathbb{N}}$, sodass
\[ A = \lim_{n \rightarrow \infty} a_{6n} = C \]
$(a_{3(2n+1)})_{n \in \mathbb{N}}$ ist eine Teilfolge von $(a_{2n+1})_{n \in \mathbb{N}}$ und $(a_{3n})_{n \in \mathbb{N}}$, sodass
\[ B = \lim_{n \rightarrow \infty} a_{3(2n+1)} = C \]
Aus (a) folgt da $A = B$ sofort dass
\[ a = A = B = C \]

\task
Nach Rechenregeln für konvergente Folgen (3.3)
\begin{align*}
	\lim_{n \rightarrow \infty} \frac{n}{2^n} &= 2 \lim_{n \rightarrow \infty} \frac{n}{2^{n+1}} = 2 \lim_{n \rightarrow \infty} \frac{n+1}{2^{n+1}} - \underbrace{\quad\lim_{n \rightarrow \infty} \frac{1}{2^{n+1}}\quad}_{=0\ \text{nach 3.3 Bsp 1}} \\
	\lim_{n \rightarrow \infty} \frac{n}{2^n} &= 2 \lim_{n \rightarrow \infty} \frac{n}{2^n} \\
	0 &= \lim_{n \rightarrow \infty} \frac{n}{2^n}
\end{align*}

Nach ÜB 1 ist
\begin{align*}
	6(1^2 + \dots + n^2) &= n(n+1)(2n+1) \\
	b_n = \frac{1^2 + \dots + n^2}{n^3} &= \frac{n(n+1)(2n+1)}{6n^3} = \frac{2n^3 + 3n^2 + n}{6n^3} = \frac{1}{3} + \frac{1}{2}n^{-1} + \frac{1}{6}n^{-2} \\\\
	\lim_{n \rightarrow \infty} b_n &= \frac{1}{3} + \frac{1}{2}\lim_{n \rightarrow \infty} n^{-1} + \frac{1}{6} \lim_{n \rightarrow \infty} n^{-2} \stackrel{\text{(3.3 Bsp 1)}}{=} \frac{1}{3}
\end{align*}
\newpage
\task
\begin{theorem}
	Jede reell-wertige Folge $(a_n)_{n \in \mathbb{N}}$ hat eine monotone Teilfolge.
\end{theorem}
\begin{proof}
	Sei $\mathcal{H}$ definiert als
\[ \mathcal{H} \coloneqq \left\{ n_0 \mid n_0 \in \mathbb{N}\ ,\ \forall n \in \mathbb{N} : n_0 < n\ \Rightarrow\ a_{n_0} \leq a_n \right\} \]
\emph{Fall 1}: $\mathcal{H}$ ist unendlich. Es kann demnach ein $(b_{n_0})_{n_0 \in \mathcal{H}}$ konstruiert werden welches nach Konstruktion monoton steigend ist.

\emph{Fall 2}: $\mathcal{H}$ ist endlich.

Demnach gibt es eine obere Schranke $S \in \mathbb{N}$ mit einer Abbildung $\nu : \mathbb{N} \rightarrow \mathbb{N}$ sodass
\begin{gather*}
	\forall n_0 \in \mathbb{N} : \quad S < n_0 \Rightarrow \exists \nu(n_0) \in \mathbb{N} : n_0 < \nu(n_0)\ \wedge\ a_{n_0} > a_{\nu(n_0)} \\
	S < n_0 < \nu(n_0)
\end{gather*}
Es kann so ein streng monoton steigendes $(\sigma_k)_{k \in \mathbb{N}}$ mit einem beliebigen $\sigma_1 \in \mathbb{N}$ , $\sigma_1 > S$ erzeugt werden mit
\[ \sigma_{n_0+1} = \nu(\sigma_{n_0}) \]
Demnach ist die Teilfolge $(b_{\sigma_k})_{k \in \mathbb{N}}$ monoton fallend, da
\[ \forall k \in \mathbb{N} : \sigma_k < \sigma_{k+1} \ \wedge\ a_{\sigma_k} > a_{\sigma_{k+1}} \]
\end{proof}
Ansatz: Philip Geißler, \url{http://tiny.cc/TelegramGruppe}, \url{https://github.com/AlcedoAttis}
\end{document}
