\documentclass[a4paper, 12pt]{scrartcl}

\usepackage[utf8]{inputenc}
\usepackage[T1]{fontenc}
\usepackage[ngerman]{babel}

\usepackage{amssymb}
\usepackage{amsmath}

\usepackage{mathrsfs}

\usepackage{framed}
\usepackage{float}
\usepackage{mathtools}

\usepackage{amsthm}
\usepackage{thmtools}
\usepackage{marvosym}

\usepackage{url}
\usepackage[hidelinks]{hyperref}

\usepackage[left=1.8cm, right=1.8cm, top=1cm]{geometry}

\allowdisplaybreaks
\setkomafont{paragraph}{\normalfont\itshape}

\setlength{\parindent}{0pt}

\declaretheoremstyle[%
  spaceabove=0,%
  spacebelow=6pt,%
  headfont=\normalfont\itshape,%
  postheadspace=1em,%
  headpunct={}
]{mystyle}

\declaretheorem[name={Behauptung}, style=mystyle, unnumbered]{theorem}
\declaretheorem[name={Lemma}, style=mystyle]{lemma}
\declaretheorem[name={Voraussetzung}, style=mystyle, unnumbered]{precondition}
\let\proof\oldproof
\declaretheorem[name={Beweis}, style=mystyle, qed=\qedsymbol, unnumbered]{proof}

\newcounter{taski}
\newcounter{taskii}[taski]
\newcounter{taskiii}[taskii]


\newcommand{\task}{\stepcounter{taski}\textbf{Aufgabe \arabic{taski}}\\}
\newcommand{\ttask}{\stepcounter{taskii}\textbf{(\alph{taskii})}\par}
\newcommand{\tttask}{\stepcounter{taskiii}\quad(\roman{taskiii})\par}

\newcommand{\defimpl}[1]{\stackrel{\text{Def.}\;#1}{\Longrightarrow}}
\newcommand{\defImpl}[1]{\stackrel{\text{Def.}\;#1}{\Longleftrightarrow}}
\newcommand{\txtimpl}[1]{\stackrel{\text{#1}}{\Longrightarrow}}
\newcommand{\txtImpl}[1]{\stackrel{\text{#1}}{\Longleftrightarrow}}
\newcommand{\refimpl}[1]{\txtimpl{\eqref{#1}}}
\newcommand{\refImpl}[1]{\txtImpl{\eqref{#1}}}

\begin{document}
\begin{flushright}
	Kamal\\
	Maximilian Neumann
\end{flushright}
\begin{center}
	\bfseries Analysis I Blatt 11
\end{center}
\task

\ttask

\ttask

Für beliebiges $f \in \mathscr{B}(M)$ gilt
\begin{align*}
	\forall x \in M : 0 &\leq |f(x)| \leq \sup_{x \in M} |f(x)| = \Vert f \Vert_M \\
	\Rightarrow \quad 0 &\leq \Vert f \Vert_M
\end{align*}
Ist $\Vert f \Vert_M = 0$, so
\begin{align*}
	\lVert f \rVert_M =& \sup_{x \in M} |f(x)| = 0 \\
	\Rightarrow \quad\forall x \in M &: |f(x)| \leq 0 \\
	\xRightarrow{0 \leq |f(x)|} \quad \forall x \in M &: f(x) = 0 \\
	\Rightarrow \quad f =&\ 0_M \qquad (0_M : M \rightarrow R~,~x \mapsto 0)
\end{align*}
Hat man zwei Funktionen $f,\ g \in \mathscr{B}(M)$, so gilt nach Dreiecksungleichung
\begin{align*}
	\forall x \in M : |f(x) + g(x)| \leq |f(x)| + |g(x)|
\end{align*}
\begin{theorem}
Für jede Funktion $f \in \mathscr{B}(M)$ mit beliebigem $\lambda \in \mathbb{R}$ gilt
\[ \Vert \lambda \cdot f(x) \Vert_M = \sup_{x \in M} |\lambda \cdot f(x)| = \lambda \sup_{x \in M} |f(x)| = \lambda \Vert f(x) \Vert \] 
\end{theorem}
\begin{proof}
Es sei $S \coloneqq \sup_{x \in M} |f(x)| = \Vert f \Vert_M$

Fall 1: $\exists x_m \in M : |f(x_m)| = S$
\begin{align*}
	\forall x \in M : |f(x)| &\leq S ~\Rightarrow~ |\lambda f(x)| = |(\lambda \cdot f)(x)| \leq |\lambda| S \\
	\exists x_m \in M : |f(x)| &= S ~\Rightarrow~ |\lambda|S = |(\lambda \cdot f)(x_m)| \\
	\xRightarrow{\max} \quad |\lambda| S = \max_{x \in M} |(\lambda \cdot S)(x)| &= \sup_{x \in M} |(\lambda \cdot S)(x)| = \Vert \lambda \cdot f \Vert_M
\end{align*}
Fall 2: $\exists x_m \in M : |f(x_m)| = S \quad,\quad \forall \varepsilon \in R_+ : \left\{ x \mid f(x) \in \mathcal{U}_\varepsilon(S) ~,~ x \in M \right\} \text{ unendlich}$ \\
Sei $\mathcal{H}_{S,\varepsilon}(f) \coloneqq \left\{ x \mid f(x) \in \mathcal{U}_\varepsilon(S) ~,~ x \in M \right\}$
\begin{align*}
	\forall \varepsilon \in \mathbb{R}_+ : \forall x \in &\:\mathcal{H}_{S,\varepsilon}(f) : |f(x) - S| < \varepsilon \\
	\Rightarrow \quad \forall \varepsilon \in \mathbb{R}_+ : \forall x \in &\:\mathcal{H}_{S,\varepsilon}(f) : |\lambda f(x) - \lambda S| < |\lambda|\varepsilon \
	\Rightarrow x \in \mathcal{H}_{|\lambda S|,|\lambda|\varepsilon}(\lambda \cdot f) \\
	\Rightarrow \quad \forall \varepsilon' \in \mathbb{R_+} : &\:\mathcal{H}_{|\lambda S|,\varepsilon'}(\lambda \cdot f) \text{ unendlich} \qquad \varepsilon(\varepsilon') = \frac{\varepsilon'}{|\lambda|} \tag{$\ast$} \\
	\\
	\forall x \in M : &\:|f(x)| < S \\
	\Rightarrow \quad \forall x \in M: &\:|(\lambda \cdot f)(x)| < |\lambda| S \tag{$\ast\ast$} \\\\
	\xRightarrow{\ast,\ast\ast} |\lambda|S = \sup_{x \in M} |(\lambda \cdot f)(x)| &= \Vert \lambda \cdot f \Vert_M
\end{align*}
\end{proof}
\ttask
\begin{theorem}
Jede \textsc{Cauchy}-Funktionsfolge $(a_n)_{n \in \mathbb{N}}$ unter der Norm $\Vert \cdot \Vert_M$ ist konvergent.
\end{theorem}
\begin{proof}
\begin{gather*}
	\forall \varepsilon \in \mathbb{R}_+ :\exists N \in \mathbb{N} : \forall n,\ m \in \mathbb{N} ~,~ n,\ m > N: \Vert a_n - a_m \Vert_B < \varepsilon \\
	\xRightarrow{\sup} \quad \forall \varepsilon \in \mathbb{R}_+ :\exists N \in \mathbb{N} : \forall n,\ m \in \mathbb{N} ~,~ n,\ m > N: \forall x \in M : |a_n(x) - a_m(x) | \leq \sup_{x \in M} |(a_n - a_m)(x)| < \varepsilon \\
	\Rightarrow \quad \forall x \in M : \bigg(\forall \varepsilon \in \mathbb{R}_+ :\exists N \in \mathbb{N} : \forall n,\ m \in \mathbb{N} ~,~ n,\ m > N:  |a_n(x) - a_m(x) | < \varepsilon \bigg) \\
	\xRightarrow{\textsc{Cauchy}} \quad \forall x \in M: (a_n(x))_{n \in \mathbb{N}} \text{ konvergent} \quad \Longrightarrow a \text{ punktweise konvergent}
\end{gather*}
\end{proof}
\task
\ttask

\begin{theorem}
	\[ f: \mathbb{R}_+ \rightarrow \mathbb{R}~,~ x \mapsto \sqrt{x} \qquad \text{gleichmäßig stetig} \]
\end{theorem}
\begin{proof}
Seien beliebige $x,\ y \in \mathbb{R}_+$ mit $0 \leq x \leq y$ gegeben.
\begin{align*}
	y &= x + (y-x) \leq y + \overbrace{2\sqrt{x}\sqrt{y-x}}^{\geq 0} + (y-x) = (\sqrt{x} + \sqrt{y-x})^2 \\
	\sqrt{y} &\leq \sqrt{x} + \sqrt{y-x} \\
	\sqrt{y} - \sqrt{x} &\leq \sqrt{y-x}
\intertext{mit $\varepsilon \coloneqq \delta^2$}
	\forall x,\ y \in \mathbb{R}_+ ~,~ |y-x| < \delta \ : |\sqrt{y} - \sqrt{x}|  &\leq \sqrt{|y-x|} = \sqrt{\delta} = \varepsilon
\end{align*}
\end{proof}
Gleichmäßige Stetigkeit impliziert Stetigkeit.

\ttask
\begin{theorem}
Für zwei Funktionen $f,\ g: M \rightarrow \mathbb{R}$ ist die Maximumsfunktion
\[ \mathrm{max}_{f,g} : M \rightarrow \mathbb{R} ~,~ x \mapsto \max \left\{ f(x),\ g(x) \right\} \]
stetig über $M$.
\end{theorem}
\begin{proof}
Der Definitionsbereich $M$ kann in drei Mengen $M_f,\ M_g$ partitioniert werden, sodass
\begin{gather*}
	M = M_f \cup M_g \qquad M_f \cap M_g= \emptyset \\
	M_f \coloneqq \left\{ x : f(x) \geq g(x) ~,~ x \in M \right\} \qquad M_g \coloneqq \left\{ x : f(x) < g(x) ~,~ x \in M \right\} \\
\end{gather*}
Es gilt
\[ \forall x \in M_f : \mathrm{max}_{f,g}(x) = f(x) \qquad \forall x \in M_g : \mathrm{max}_{f,g}(x) = g(x)  \]
An allen Rändern $x^*$ von kompakten Intervallen zwischen $M_f$ und $M_g$ gilt $f(x^*) = g(x^*)$, sodass
\[ \lim_{x \rightarrow x^*_-} \mathrm{max}_{f,g}(x) = g(x^*) = f(x^*) = \lim_{x \rightarrow x^*_+} \mathrm{max}_{f,g}(x) \]
Demnach ist $\mathrm{max}_{f,g}(x)$ auf $M_f \cup M_g = M$ stetig.
\end{proof}

\task

\begin{theorem}
	Eine \textsc{Lipschitz}-stetige Funktion $f: M \rightarrow \mathbb{R}$ mit $L \in \mathbb{R_+}$ sodass
	\[ \forall x,\ x' \in M: |f(x)-f(x)'| < L|x-x'| \]
	ist gleichmäßig stetig
\end{theorem}
\begin{proof}
Es sei $\varepsilon \coloneqq L\delta$
\begin{align*}
	\xRightarrow{\textsc{Lipschitz}} \quad & \forall x,\ x' \in M: |f(x)-f(x)'| < L|x-x'| \\
	\Longrightarrow \quad & \forall x,\ x' \in M ~,~ |x-x'| < \delta\ : |f(x)-f(x')| < L \delta = \varepsilon
\end{align*}
\end{proof}

In Aufgabe 2a wurde bereits gezeigt, dass die Funktion $f: \mathbb{R}_+ \rightarrow \mathbb{R}~,~ x \mapsto \sqrt{x}$ gleichmäßig stetig ist. Setzt man jedoch $x' = 0$ und betrachtet $\lim_{x \rightarrow 0} |f(x)-f(x')|$ so folgt für $L$
\[
	L > \lim_{x \rightarrow 0} \frac{|f(x)-f(x')|}{|x-x'|} = \lim_{x \rightarrow 0} \frac{\sqrt{x}}{x} = \lim_{x \rightarrow 0} \frac{1}{\sqrt{x}} = \infty
\]
sodass $L$ nicht existieren kann. Demnach ist die Funktion gleichmäßig stetig, aber nicht \textsc{Lipschitz}-stetig.
\end{document}
