\documentclass[a4paper, 12pt]{scrartcl}

\usepackage[utf8]{inputenc}
\usepackage[T1]{fontenc}
\usepackage[ngerman]{babel}

\usepackage{amssymb}
\usepackage{amsmath}
\usepackage{framed}
\usepackage{float}
\usepackage{mathtools}

\usepackage{tikz}

\usepackage{amsthm}
\usepackage{thmtools}
\usepackage{marvosym}

\usepackage[left=2cm, right=2cm, top=1cm]{geometry}

\allowdisplaybreaks
\setkomafont{paragraph}{\normalfont\itshape}

\setlength{\parindent}{0pt}

\declaretheoremstyle[%
  spaceabove=0,%
  spacebelow=6pt,%
  headfont=\normalfont\itshape,%
  postheadspace=1em,%
  headpunct={}
]{mystyle}

\declaretheorem[name={Behauptung}, style=mystyle, unnumbered]{theorem}
\declaretheorem[name={Lemma}, style=mystyle]{lemma}
\declaretheorem[name={Voraussetzung}, style=mystyle, unnumbered]{precondition}
\let\proof\oldproof
\declaretheorem[name={Beweis}, style=mystyle, qed=\qedsymbol, unnumbered]{proof}

\newcounter{taski}
\newcounter{taskii}[taski]
\newcounter{taskiii}[taskii]


\newcommand{\task}{\stepcounter{taski}\textbf{Aufgabe \arabic{taski}}\\}
\newcommand{\ttask}{\stepcounter{taskii}\textbf{(\alph{taskii})}\par}
\newcommand{\tttask}{\stepcounter{taskiii}\quad(\roman{taskiii})\par}

\newcommand{\defimpl}[1]{\stackrel{\text{Def.}\;#1}{\Longrightarrow}}
\newcommand{\defImpl}[1]{\stackrel{\text{Def.}\;#1}{\Longleftrightarrow}}
\newcommand{\txtimpl}[1]{\stackrel{\text{#1}}{\Longrightarrow}}
\newcommand{\txtImpl}[1]{\stackrel{\text{#1}}{\Longleftrightarrow}}

\DeclareMathOperator*{\argmin}{arg\,min}
\begin{document}
\begin{flushright}
	Kamal\\
	Maximilian Neumann
\end{flushright}
\begin{center}
	\bfseries Analysis I Blatt 4
\end{center}
\task
\ttask
\begin{theorem}
	\[ \forall n \in \mathbb{N}_0 : \sum_{k=0}^n \binom{n}{k} = 2^n \]
\end{theorem}
\begin{proof}
\[ 2^n = (1+1)^n = \sum_{k=0}^n \binom{n}{k} 1^k1^{n-k} = \sum_{k=0}^n \binom{n}{k} \]
\end{proof}
\begin{theorem}
	\[ \forall n \in \mathbb{N} : \sum_{k=0}^n(-1)^k\binom{n}{k} = 0 \]
\end{theorem}
\begin{proof}
	\[ 0 = 0^n = (1-1)^n = \sum_{k=0}^n \binom{n}{k}(-1)^k1^{n-k} = \sum_{k=0}^n(-1)^k\binom{n}{k} \tag{$n \in \mathbb{N}$} \]
\end{proof}
\ttask
\begin{theorem}
	\[ \forall n \in \mathbb{N} : \sum_{k=0}^n \binom{n}{k}^2 = \binom{2n}{n} \]
\end{theorem}
\begin{proof}
\[ 	\sum_{k=0}^{2n}\binom{n}{k}x^k = (1+x)^{2n} = (1+x)^n(1+x)^n = \left( \sum_{k=0}^n \binom{n}{k}x^k \right) \left( \sum_{k=0}^n \binom{n}{k}x^{n-k} \right) \]
Einvariablige Polynome bilden einen Vektorraum, sodass jedes Polynom eindeutig bestimmte Koeffizienten hat. Sind zwei Polynome gleich, so müssen demnach ihre Koeffizienten übereinstimmen.

Auf linker Seite steht der Summand $\binom{2n}{n}x^n$, welcher gleich der Summe aller Summanden vom Grad $x^n$ auf der rechten Seite ist. Es gilt
\[ \forall i,\ j \in I: \binom{n}{i}x^i \cdot \binom{n}{j}x^{n-j} = ax^n \Leftrightarrow i=j \tag{$a \in \mathbb{N}$} \]
sodass
\begin{align*}
	\binom{2n}{n}x^n  &= \sum_{k=0}^n \binom{n}{k}x^k \cdot \binom{n}{k}x^{n-k} = x^n\sum_{k=0}^n \binom{n}{k}^2 \\
	\binom{2n}{n} &= \sum_{k=0}^n \binom{n}{k}^2
\end{align*}
\end{proof}
\task
\ttask
\begin{theorem}
	\[ \forall a,\ b \in \mathbb{R}^0_+ : \left( \frac{a+b}{2} \right)^2 \geq ab  \]
\end{theorem}
\begin{proof}
\begin{align*}
	\txtimpl{Satz 2.5.1} \quad 0 &\leq \left( \frac{a-b}{2} \right)^2 = \frac{a^2+b^2}{4} - \frac{1}{2}\,ab = \left( \frac{a+b}{2} \right)^2 - ab \\
	\txtimpl{Axiom B iii} \quad ab &\leq \left( \frac{a+b}{2} \right)^2
\end{align*}
\end{proof}

\ttask
\begin{theorem}
	\[ \left( \frac{a+b+c+d}{4} \right)^4 \geq abcd  \]
\end{theorem}
\begin{proof}
\begin{align*}
	\txtimpl{Satz (a)}&&\quad\left( \frac{a+b+c+d}{4} \right)^2 &\geq \frac{a+b}{2} \cdot \frac{c+d}{2} \\
\intertext{Nach Satz 2.5.2 $\displaystyle \forall x,\ y \in \mathbb{R}^0_+ : x \geq y \Rightarrow x^2 \geq xy \geq y^2$}
	\Rightarrow&&\qquad\left( \frac{a+b+c+d}{4} \right)^4 &\geq \left(\frac{a+b}{2}\right)^2 \cdot \left(\frac{c+d}{2}\right)^2 \\
	\txtimpl{Satz (a)}&&\quad&\geq abcd
\end{align*}
\end{proof}
Die Ungleichung hält für beliebig viele Variablen, da das arithmetische Mittel immer größer oder gleich dem geometrischen Mittel ist. Gleichheit gilt genau dann, wenn alle zu mittelnden Werte untereinander gleich sind, da in diesem Fall beide Mittel genau eben diesen Wert annehmen.\\

\task
\ttask
Fall 1: $\frac{1}{2}x + 3 > 0$
\[ \left\lvert \frac{1}{2}x + 3 \right\rvert = \frac{1}{2}x + 3 = 4 \quad,\quad x = 2 \quad,\quad \frac{1}{2}\,2 + 3 > 0\]
Fall 2: $\frac{1}{2}x + 3 \leq 0$
\[ \left\lvert \frac{1}{2}x + 3 \right\rvert = -\frac{1}{2}x - 3 = 4 \quad,\quad x = -14 \quad,\quad \frac{1}{2}\,2(-14) + 3 < 0\]
\[ \mathcal{L} = \left\{ -14,\ 2 \right\} \]
\newpage
\ttask
Fall 1: $x-2 > 0$
\[ \left\lvert x-2 \right\rvert = x-2 < \frac{1}{2} \quad,\quad x < \frac{5}{2} \quad \Rightarrow \quad x \in \big(2,\tfrac{5}{2}\big) \]
Fall 2: $x-2 \leq 0$
\[ \left\lvert x-2 \right\rvert = -x + 2 < \frac{1}{2} \quad,\quad x > \frac{3}{2} \quad \Rightarrow \quad x \in \big(\tfrac{3}{2},2\big] \]
\[ \mathcal{L} = \big(\tfrac{3}{2},2\big] \cup \big(2,\tfrac{5}{2}\big) = \left( \tfrac{3}{2},\tfrac{5}{2} \right)  \]
\ttask

Fall 1: $\quad2x - 1 \geq 0 \ \wedge\ x - 1 \geq 0$
	\[2x - 1 \geq x - 1 + 1 \quad,\quad x \geq 1 \quad,\quad \forall x \geq 1 : 2x - 1 \geq 0 \ \wedge
\ x - 1 \geq 0 \quad \Rightarrow \quad x \in [1;\infty) \]
Fall 2: $\quad2x - 1 \geq 0 \ \wedge\ x - 1 < 0$
\[ 2x - 1 \geq -x + 1 + 1 \quad,\quad x \geq 1 \quad \text{\Lightning\ da}\quad x < 1\]
Fall 3: $\quad2x - 1 < 0 \ \wedge\ x - 1 \geq 0$
\[ -2x + 1 \geq x - 1 + 1\quad,\quad x \leq \frac{1}{3} \quad \text{\Lightning\ da}\quad x \geq 1\]
Fall 4: $\quad2x - 1 < 0 \ \wedge\ x - 1 < 0$
\[ -2x + 1 \geq -x + 1 + 1 \quad,\quad x \leq -1 \quad,\quad \forall x \leq -1 : \quad2x - 1 < 0 \ \wedge\ x - 1 < 0 \quad \Rightarrow \quad x \in \big(-\infty;-1\big]\]
\[ \mathcal{L} = (-\infty;-1]\ \cup\  [1;\infty) = \mathbb{R}\backslash(-1;1)\]
\end{document}
