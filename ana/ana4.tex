\documentclass[a4paper, 12pt]{scrartcl}

\usepackage[utf8]{inputenc}
\usepackage[T1]{fontenc}
\usepackage[ngerman]{babel}

\usepackage{amssymb}
\usepackage{amsmath}
\usepackage{framed}
\usepackage{float}
\usepackage{mathtools}

\usepackage{tikz}

\usepackage{amsthm}
\usepackage{thmtools}
\usepackage{marvosym}

\usepackage[left=2cm, right=2cm, top=1cm]{geometry}

\allowdisplaybreaks
\setkomafont{paragraph}{\normalfont\itshape}

\setlength{\parindent}{0pt}

\declaretheoremstyle[%
  spaceabove=0,%
  spacebelow=6pt,%
  headfont=\normalfont\itshape,%
  postheadspace=1em,%
  headpunct={}
]{mystyle}

\declaretheorem[name={Behauptung}, style=mystyle, unnumbered]{theorem}
\declaretheorem[name={Lemma}, style=mystyle]{lemma}
\declaretheorem[name={Voraussetzung}, style=mystyle, unnumbered]{precondition}
\let\proof\oldproof
\declaretheorem[name={Beweis}, style=mystyle, qed=\qedsymbol, unnumbered]{proof}

\newcounter{taski}
\newcounter{taskii}[taski]
\newcounter{taskiii}[taskii]


\newcommand{\task}{\stepcounter{taski}\textbf{Aufgabe \arabic{taski}}\\}
\newcommand{\ttask}{\stepcounter{taskii}\textbf{(\alph{taskii})}\par}
\newcommand{\tttask}{\stepcounter{taskiii}\quad(\roman{taskiii})\par}

\newcommand{\defimpl}[1]{\stackrel{\text{Def.}\;#1}{\Longrightarrow}}
\newcommand{\defImpl}[1]{\stackrel{\text{Def.}\;#1}{\Longleftrightarrow}}
\newcommand{\txtimpl}[1]{\stackrel{\text{#1}}{\Longrightarrow}}
\newcommand{\txtImpl}[1]{\stackrel{\text{#1}}{\Longleftrightarrow}}

\DeclareMathOperator*{\argmin}{arg\,min}
\begin{document}
\begin{flushright}
	Kamal\\
	Maximilian Neumann
\end{flushright}
\begin{center}
	\bfseries Analysis I Blatt 4
\end{center}
\task
\ttask
\begin{theorem}
	\[ \forall n \in \mathbb{N}_0 : \sum_{k=0}^n \binom{n}{k} = 2^n \]
\end{theorem}
\begin{proof}
\[ 2^n = (1+1)^n = \sum_{k=0}^n \binom{n}{k} 1^k1^{n-k} = \sum_{k=0}^n \binom{n}{k} \]
\end{proof}
\begin{theorem}
	\[ \forall n \in \mathbb{N} : \sum_{k=0}^n(-1)^k\binom{n}{k} = 0 \]
\end{theorem}
\begin{proof}
	\[ 0 = 0^n = (1-1)^n = \sum_{k=0}^n \binom{n}{k}(-1)^k1^{n-k} = \sum_{k=0}^n(-1)^k\binom{n}{k} \tag{$n \in \mathbb{N}$} \]
\end{proof}
\ttask
\begin{theorem}
	\[ \forall n \in \mathbb{N} : \sum_{k=0}^n \binom{n}{k}^2 = \binom{2n}{n} \]
\end{theorem}
\begin{proof}
\[ 	\sum_{k=0}^{2n}\binom{n}{k}x^k = (1+x)^{2n} = (1+x)^n(1+x)^n = \left( \sum_{k=0}^n \binom{n}{k}x^k \right) \left( \sum_{k=0}^n \binom{n}{k}x^{n-k} \right) \]
\end{proof}
\begin{lemma}
Für zwei reell wertigen Folgen $(a_k)_{k \in \mathbb{N}_0},\ (b_k)_{k \in \mathbb{N}_0}$ gilt
\[ \forall k \in \mathbb{N}_0\:,\: k \leq n : a_k = b_k \ \Longleftrightarrow\ \forall x \in \mathbb{R}^* : \sum_{k=0}^na_kx^k = \sum_{k=0}^nb_kx^k \]
\end{lemma}
\begin{proof}\ \\
1. Richtung $\Rightarrow$ trivial

2. Richtung $\Leftarrow$
Beweis durch Widerspruch
\end{proof}
\task
\ttask
\begin{theorem}
	\[ \forall a,\ b \in \mathbb{R}^0_+ : \left( \frac{a+b}{2} \right)^2 \geq ab  \]
\end{theorem}
\begin{proof}
\begin{align*}
	\left( \frac{a+b}{2} \right)^2 &= \frac{a^2+b^2}{4} + \frac{1}{2}\,ab
\intertext{Nach $\textsc{Cauchy-Schwarz}$ gilt $\quad\displaystyle a^2+b^2 \geq 2ab$}
	\geq \frac{1}{2}\,ab + \frac{1}{2}\,ab = ab
\end{align*}
\end{proof}

\ttask
\begin{align*}
	\left( \frac{a+b+c+d}{4} \right)^2 &\geq \frac{a+b}{2} \cdot \frac{c+d}{2} \\
\intertext{Nach Satz 1.5.2 $\displaystyle \forall x,\ y \in \mathbb{R}^0_+ : x \geq y \Rightarrow x^2 \geq xy \geq y\forall a,\ b \in \mathbb{R}^0_+ : a \geq b \Rightarrow a^2 \geq ab \geq b^2$}
	\Rightarrow \qquad\left( \frac{a+b+c+d}{4} \right)^4 &\geq \left(\frac{a+b}{2}\right)^2 \cdot \left(\frac{c+d}{2}\right)^2 \\
	&\geq abcd
\end{align*}

\task

\ttask
Fall 1: $\frac{1}{2}x + 3 > 0$
\[ \left\lvert \frac{1}{2}x + 3 \right\rvert = \frac{1}{2}x + 3 = 4 \quad,\quad x = 2 \]
Fall 2: $\frac{1}{2}x + 3 \leq 0$
\[ \left\lvert \frac{1}{2}x + 3 \right\rvert = -\frac{1}{2}x - 3 = 4 \quad,\quad x = -14 \]
\[ \mathcal{L} = \left\{ -14,\ 2 \right\} \]
\ttask
Fall 1: $x-2 > 0$
\[ \left\lvert x-2 \right\rvert = x-2 < \frac{1}{2} \quad,\quad x < \frac{5}{2} \]
Fall 2: $x-2 \leq 0$
\[ \left\lvert x-2 \right\rvert = -x + 2 < \frac{1}{2} \quad,\quad x > \frac{3}{2} \]
\[ \mathcal{L} = \mathbb{R}\backslash [\tfrac{3}{2};\tfrac{5}{2}] \]

\end{document}
