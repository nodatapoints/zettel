\documentclass[a4paper, 12pt]{scrartcl}

\usepackage[utf8]{inputenc}
\usepackage[T1]{fontenc}
\usepackage[ngerman]{babel}

\usepackage{amssymb}
\usepackage{amsmath}
\usepackage{framed}
\usepackage{float}
\usepackage{mathtools}

\usepackage{tikz}

\usepackage{amsthm}
\usepackage{thmtools}
\usepackage{marvosym}

\usepackage{url}
\usepackage[hidelinks]{hyperref}

\usepackage[left=1.8cm, right=1.8cm, top=1cm]{geometry}

\allowdisplaybreaks
\setkomafont{paragraph}{\normalfont\itshape}

\setlength{\parindent}{0pt}

\declaretheoremstyle[%
  spaceabove=0,%
  spacebelow=6pt,%
  headfont=\normalfont\itshape,%
  postheadspace=1em,%
  headpunct={}
]{mystyle}

\declaretheorem[name={Behauptung}, style=mystyle, unnumbered]{theorem}
\declaretheorem[name={Lemma}, style=mystyle]{lemma}
\declaretheorem[name={Voraussetzung}, style=mystyle, unnumbered]{precondition}
\let\proof\oldproof
\declaretheorem[name={Beweis}, style=mystyle, qed=\qedsymbol, unnumbered]{proof}

\newcounter{taski}
\newcounter{taskii}[taski]
\newcounter{taskiii}[taskii]


\newcommand{\task}{\stepcounter{taski}\textbf{Aufgabe \arabic{taski}}\\}
\newcommand{\ttask}{\stepcounter{taskii}\textbf{(\alph{taskii})}\par}
\newcommand{\tttask}{\stepcounter{taskiii}\quad(\roman{taskiii})\par}

\newcommand{\defimpl}[1]{\stackrel{\text{Def.}\;#1}{\Longrightarrow}}
\newcommand{\defImpl}[1]{\stackrel{\text{Def.}\;#1}{\Longleftrightarrow}}
\newcommand{\txtimpl}[1]{\stackrel{\text{#1}}{\Longrightarrow}}
\newcommand{\txtImpl}[1]{\stackrel{\text{#1}}{\Longleftrightarrow}}
\newcommand{\refimpl}[1]{\txtimpl{\eqref{#1}}}
\newcommand{\refImpl}[1]{\txtImpl{\eqref{#1}}}

\begin{document}
\begin{flushright}
	Kamal\\
	Maximilian Neumann
\end{flushright}
\begin{center}
	\bfseries Analysis I Blatt 9
\end{center}
\task
\setcounter{taskii}{1}
\ttask\ Seien $(a_n)_{n \in \mathbb{N}},\ (b_n)_{n \in \mathbb{N}}$ reell-wertige Folgen sodass
\begin{align*}
	a_n \coloneqq \frac{1}{\sqrt[n]{n^n}} = \frac{1}{n} \quad&\quad b_n \coloneqq \frac{1}{\sqrt[n]{n^{n+1}}} \\
	\frac{a_n}{b_n} &= \sqrt[n]{\frac{n^{n+1}}{n^n}} = \sqrt[n]{n} \\
	\lim_{n \rightarrow \infty}\frac{a_n}{b_n} &= \lim_{n \rightarrow \infty}\sqrt[n]{n} = 1 \\
	\sum_{n=1}^\infty a_n = \infty \quad &\txtimpl{(a)} \quad \sum_{n=1}^\infty b_n = \infty
\end{align*}
Es liegt keine Konvergenz vor.

\task
\ttask\ Nach Wurzelkriterium
\begin{align*}
	\sqrt[n]{\frac{n!}{n^n}} &= \frac{\sqrt{1}}{n} \cdot \frac{\sqrt{2}}{n} \cdot \hdots \cdot \frac{\sqrt{n}}{n} \qquad \forall i > n: \sqrt{i} > \sqrt{n} \\
	\Rightarrow \quad &<  \frac{\sqrt{n}}{n} \cdot \frac{\sqrt{n}}{n} \cdot \hdots \cdot \frac{\sqrt{n}}{n} = \left( \frac{1}{\sqrt{n}} \right)^n  \qquad \forall n \in \mathbb{N}: \frac{1}{\sqrt n} \leq 1 \\
	\Rightarrow \quad &< 1 
\end{align*}
liegt bedingte und unbedingte Konvergenz vor, da $\forall n \in \mathbb{N}: \frac{n!}{n^n} > 0$

\ttask\ Nach Quotientenkriterium
\begin{align*}
	\frac{\:\dfrac{((n+1)!)^2}{(2(n+1))!\:}}{\dfrac{(n!)^2}{(2n)!}} &= \left( \frac{(n+1)!}{n!} \right)^2 \cdot \left( \frac{(2n)!}{(2n+2)!} \right)  = (n+1)^2 \cdot \frac{1}{(2n+1)(2n+2)} \\
	&= \underbrace{\frac{n+1}{2n+1}}_{< 1} \cdot \underbrace{\frac{n+1}{2n+2}}_{< 1} < 1
\end{align*}
liegt bedingte und unbedingte Konvergenz vor, da die Folge positiv bleibt.

\newpage
\ttask\ Nach \textsc{Leibniz} ist es für bedingte Konvergenz hinreichend die Nullfolge nachzuweisen.\\
Z. z.: $\displaystyle \lim_{n \rightarrow \infty} \frac{\sqrt{n+1}-\sqrt n}{n} = 0$
\begin{align*}
	\forall n \in \mathbb{N} : \quad \frac{\sqrt{n+1}-\sqrt n}{n} &= \frac 1 n \cdot \frac{1}{\sqrt{n+1}+\sqrt n} < \frac{1}{n} \\
	\lim_{n \rightarrow \infty} \frac 1 n &= 0 \\
	\txtimpl{Majorante} \quad \lim_{n \rightarrow \infty} \frac{\sqrt{n+1}-\sqrt n}{n} &= 0
\end{align*}

\ttask
\begin{align*}
	\forall n \in \mathbb{N}: \quad \frac{2n+1}{n(n+1)} &= \frac{1}{n} + \frac{1}{n+1} \\
	\txtimpl{Rechn. Reihen} \quad \sum_{n=1}^\infty \frac{2n+1}{n(n+1)} &= \sum_{n=1}^\infty \frac{1}{n} + \sum_{n=1}^\infty \frac{1}{n+1} = \infty
\end{align*}
Es liegt demnach keine unbedingte Konvergenz vor. Mit \textsc{Leibniz} und Rechenregeln für konvergente Folgen geht hervor dass
\begin{align*}
	\lim_{n \rightarrow \infty} \frac{2n+1}{n(n+1)} &= \lim_{n \rightarrow \infty} \frac{1}{n} + \lim_{n \rightarrow \infty} \frac{1}{n+1} = 0
\end{align*}
womit es sich um eine Nullfolge und somit um bedingte Konvergenz handelt.

\task

\ttask
\begin{theorem}
	Es sei eine positiv-reellwertige Folge $(a_n)_{n \in \mathbb{N}}$ gegeben, wobei mit festem $p,\ q \in \mathbb{R}$ gilt
	\[ \forall n \in \mathbb{N}: p \leq \frac{a_{n+1}}{a_n} \leq q \]
	Dann gilt
	\[ \limsup_{n \rightarrow \infty} \sqrt[n]{a_n} \leq \limsup_{n \rightarrow \infty} \frac{a_{n+1}}{a_n} \]
\end{theorem}
\begin{proof}
1. Use lemma
\begin{align*}
	2.\ \forall n \in \mathbb{N}: \quad \sqrt[n]{a_n} \geq \sqrt[n]{a_1p^{n-1}}
\end{align*}
3. ??? \\
4. Profit
\end{proof}
\begin{lemma}
	\[ \forall n \in \mathbb{N}: a_1p^n \leq a_{n+1} \]
\end{lemma}
\begin{proof}
\begin{align*}
	\forall n \in \mathbb{N}: \quad a_1p^n &= \quad a_1 \overbrace{p \cdot \hdots \cdot p}^n \leq a_1 \frac{a_2}{a_1} \cdot \frac{a_3}{a_2} \cdot \hdots \cdot \frac{a_{n+1}}{a_n}\quad  = a_{n+1}
\end{align*}
\end{proof}
\ttask\ Sei $(a_n)_{n \in \mathbb{N}}$ als reell-wertige Folge definiert als
\begin{align*}
	a_n &\coloneqq \left( \frac{n+1}{\sqrt[n]{n!}} \right)^n = \frac{(n+1)^n}{n!} \\
	\frac{a_{n+1}}{a_n} &= \frac{(n+2)^{n+1}}{(n+1)^n} \cdot \frac{n!}{(n+1)!} = \left( \frac{n+2}{n+1} \right)^{n+1} \\
	&= \left( 1 + \frac{1}{n+1} \right)^{n+1} \\
	\Rightarrow \quad \liminf_{n \rightarrow \infty} \frac{a_{n+1}}{a_n} = \limsup_{n \rightarrow \infty} \frac{a_{n+1}}{a_n} &= \lim_{n \rightarrow \infty} \frac{a_{n+1}}{a_n} = \lim_{n \rightarrow \infty} \left( 1 + \frac 1 n \right)^n = e \\
	\txtimpl{(a)} \quad e = \liminf_{n \rightarrow \infty} \sqrt[n]{a_n} &= \limsup_{n \rightarrow \infty} \sqrt[n]{a_n} = e \\
	\lim_{n \rightarrow \infty} \frac{n+1}{\sqrt[n]{n!}} &= \lim_{n \rightarrow \infty} \sqrt[n]{a_n} = e
\end{align*}
\end{document}
