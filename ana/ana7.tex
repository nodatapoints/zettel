\documentclass[a4paper, 12pt]{scrartcl}

\usepackage[utf8]{inputenc}
\usepackage[T1]{fontenc}
\usepackage[ngerman]{babel}

\usepackage{amssymb}
\usepackage{amsmath}
\usepackage{framed}
\usepackage{float}
\usepackage{mathtools}

\usepackage{tikz}

\usepackage{amsthm}
\usepackage{thmtools}
\usepackage{marvosym}

\usepackage{url}
\usepackage[hidelinks]{hyperref}

\usepackage[left=1.8cm, right=1.8cm, top=1cm]{geometry}

\allowdisplaybreaks
\setkomafont{paragraph}{\normalfont\itshape}

\setlength{\parindent}{0pt}

\declaretheoremstyle[%
  spaceabove=0,%
  spacebelow=6pt,%
  headfont=\normalfont\itshape,%
  postheadspace=1em,%
  headpunct={}
]{mystyle}

\declaretheorem[name={Behauptung}, style=mystyle, unnumbered]{theorem}
\declaretheorem[name={Lemma}, style=mystyle]{lemma}
\declaretheorem[name={Voraussetzung}, style=mystyle, unnumbered]{precondition}
\let\proof\oldproof
\declaretheorem[name={Beweis}, style=mystyle, qed=\qedsymbol, unnumbered]{proof}

\newcounter{taski}
\newcounter{taskii}[taski]
\newcounter{taskiii}[taskii]


\newcommand{\task}{\stepcounter{taski}\textbf{Aufgabe \arabic{taski}}\\}
\newcommand{\ttask}{\stepcounter{taskii}\textbf{(\alph{taskii})}\par}
\newcommand{\tttask}{\stepcounter{taskiii}\quad(\roman{taskiii})\par}

\newcommand{\defimpl}[1]{\stackrel{\text{Def.}\;#1}{\Longrightarrow}}
\newcommand{\defImpl}[1]{\stackrel{\text{Def.}\;#1}{\Longleftrightarrow}}
\newcommand{\txtimpl}[1]{\stackrel{\text{#1}}{\Longrightarrow}}
\newcommand{\txtImpl}[1]{\stackrel{\text{#1}}{\Longleftrightarrow}}
\newcommand{\refimpl}[1]{\txtimpl{\eqref{#1}}}
\newcommand{\refImpl}[1]{\txtImpl{\eqref{#1}}}

\begin{document}
\begin{flushright}
	Mike,\ Kamal\\
	Maximilian Neumann
\end{flushright}
\begin{center}
	\bfseries Analysis I Blatt 7
\end{center}
\task
\begin{theorem}
\[ \sup M \in M \]
\end{theorem}
\begin{proof}
Seien $(c_n)_{n \in \mathbb{N}},\ (s_n)_{n \in \mathbb{N}}$ reell-wertige Folgen mit
\[ c_n \coloneqq \sup M \qquad s_n \coloneqq \frac{1}{n}\inf M + \left( 1 - \frac{1}{n} \right)\sup M  \]
Es gilt für beliebiges $n \in \mathbb{N}$
\begin{align*}
	\inf M \leq \sup M \Rightarrow s_n &\leq \frac{1}{n}\sup M + \left( 1 - \frac{1}{n} \right)\sup M = \sup M \\
	s_n &\leq c_n
\end{align*}
Weiterhin gilt nach Rechenregeln konvergenter Folgen
\[ 	\lim_{n \rightarrow \infty} s_n - c_n = \inf M \left( \lim_{n \rightarrow \infty} \frac{1}{n} \right)  + \sup M  \left(\lim_{n \rightarrow \infty} 1 - \frac{1}{n} \right) - \sup M = 0 + 1 \cdot \sup M - \sup M = 0
 \]
Es kann demnach eine Intervallschachtelung $(I_n)_{n \in \mathbb{N}}$ , $I_n = [s_n;c_n]$ konstruiert werden. \\

Fall 1: $\exists n_0 \in \mathbb{N} : I_{n_0} \cap M$ endlich
\begin{gather*}
	\max(I_{n_0} \cap M) \in M ~,~ \max(I_{n_0} \cap M) = \sup(I_{n_0} \cap M) = \sup M \quad \text{da} \quad \forall m \in M : m \leq \sup M = c_n \\
	\Longrightarrow \quad \sup M = \max(I_{n_0} \cap M) \in M
\end{gather*}
Fall 2: $\forall n_0 \in \mathbb{N} : I_{n_0} \cap M$ unendlich
\begin{align*}
	\lim_{n \rightarrow \infty} c_n -s_n = 0 \quad\defimpl{\lim}\quad& \forall \varepsilon \in \mathbb{R}^+\ \exists n_0 \in \mathbb{N}\ \forall n \in \mathbb{N}~,~n_0 <n : \sup M - s_n < \varepsilon \\
	\defimpl{I_n} \quad &\forall \varepsilon \in \mathbb{R}^+\ \exists n_0 \in \mathbb{N} : I_{n_0} \subseteq [\sup M - \varepsilon;\sup M]~,~I_{n_0}\ \text{unendlich} \\
	\Longrightarrow \quad &\forall \varepsilon \in \mathbb{R}^+ : [\sup M - \varepsilon;\sup M] \cap M\ \text{unendlich} \\
	\defimpl{M} \quad &\sup M \in M
\end{align*}
\end{proof}
\newpage
\task

\begin{theorem}
	\[ \forall n \in \mathbb{N}: 1 \leq a_n \]
\end{theorem}
\begin{proof} durch vollständige Induktion
\paragraph*{Induktionsanfang} $1 \leq a \coloneqq a_1$
\paragraph*{Induktionsbehauptung} $1 \leq a_n$
\paragraph*{Induktionsschritt} Z.z. $\quad 1 \leq a_{n+1}$
\begin{align*}
	1 &\leq a_n \\
	\txtimpl{Satz 2.5.2}\quad -\frac{1}{a_n} &\geq -1 \\
	\txtimpl{Axiome $\mathbb{R}$} \quad 2 - \frac{1}{a_n} &\geq 2 - 1 = 1 \\
	\defimpl{a_{n+1}} \quad a_{n+1} &\geq 1
\end{align*}
$a_n$ ist nach unten beschränkt.
\end{proof}
Für beliebiges $n$ gilt nach 2.5 Rechnen mit Ungleichungen
\begin{align*}
	0 &\leq (a_n - 1)^2 \\
	\Rightarrow \quad 0 &\leq a_n^2 - 2a_n + 1 \\
	\Rightarrow \quad 2 &\leq a_n + \frac{1}{a_n} \\
	\Rightarrow \quad 2 - \frac{1}{a_n} &\leq a_n \\
	\defimpl{a_{n+1}}\quad a_{n+1} &\leq a_n
\end{align*}
sodass $a_n$ monoton fallend ist und so auch nach oben durch $a_1$ beschränkt ist. Nach Satz 3.4 ist $a_n$ somit konvergent, da es beschränkt und monoton ist. Demnach existiert ein $\lim_{n \rightarrow \infty} a_n$, sodass nach Rechenregeln für konvergente Folgen gilt
\begin{align*}
	a \coloneqq \lim_{n \rightarrow \infty} a_n &= \lim_{n \rightarrow \infty} a_{n+1} = 2 - \frac{1}{\lim_{n \rightarrow \infty} a_n} = 2 - \frac{1}{a} \\
	\Rightarrow \quad 0 &= a^2 - 2a + 1 = (a-1)^2 \\
	\Rightarrow \quad a &= 1
\end{align*}
\newpage
\task
\begin{theorem}
	\[ \lim_{n \rightarrow \infty} a_n = \lim_{n \rightarrow \infty} b_n \]
\end{theorem}
\begin{proof}
Sei o.b.d.A $a < b$. Es gilt nach Rechenregeln für Ungleichungen
\begin{align*}
	a^2 &< ab \tag{$a > 0$} < b^2 & 2a &< a + b < 2b \\
	\Rightarrow \quad a &< \sqrt{ab} < b & a &< \frac{a+b}2 < b
\end{align*}
Es wurde in Blatt 4 bereits gezeigt dass bei $a \neq b$
\begin{align*}
	ab &< \left( \frac{a+b}2 \right)^2 \\
	\sqrt{ab} &< \frac{a+b}2
\end{align*}
Sodass insgesamt dach Def. $a_n,\ b_n$ gilt dass
\[ a_n < a_{n+1} < b_{n+1} < b_n \]
Es liegt eine Intervallschachtelung vor, da $a_n$ und $b_n$ monoton sind jeweils durch $b$ und $a$ beschränkt sind. So gilt nach Satz 3.5 dass $a_n$ und $b_n$ konvergent sind und
\[ \lim_{n \rightarrow \infty} a_n = \lim_{n \rightarrow \infty} b_n \]
Obige Argumentation kann analog zu $b < a$ geführt werden. Bei Gleichheit gilt
\[ a = b = \sqrt{ab} = \frac{a+b}2 = \lim_{n \rightarrow \infty} a_n = \lim_{n \rightarrow \infty} b_n  \]
\end{proof}
\end{document}
