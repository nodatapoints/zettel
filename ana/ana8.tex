\documentclass[a4paper, 12pt]{scrartcl}

\usepackage[utf8]{inputenc}
\usepackage[T1]{fontenc}
\usepackage[ngerman]{babel}

\usepackage{amssymb}
\usepackage{amsmath}
\usepackage{framed}
\usepackage{float}
\usepackage{mathtools}

\usepackage{tikz}

\usepackage{amsthm}
\usepackage{thmtools}
\usepackage{marvosym}

\usepackage{url}
\usepackage[hidelinks]{hyperref}

\usepackage[left=1.8cm, right=1.8cm, top=1cm]{geometry}

\allowdisplaybreaks
\setkomafont{paragraph}{\normalfont\itshape}

\setlength{\parindent}{0pt}

\declaretheoremstyle[%
  spaceabove=0,%
  spacebelow=6pt,%
  headfont=\normalfont\itshape,%
  postheadspace=1em,%
  headpunct={}
]{mystyle}

\declaretheorem[name={Behauptung}, style=mystyle, unnumbered]{theorem}
\declaretheorem[name={Lemma}, style=mystyle]{lemma}
\declaretheorem[name={Voraussetzung}, style=mystyle, unnumbered]{precondition}
\let\proof\oldproof
\declaretheorem[name={Beweis}, style=mystyle, qed=\qedsymbol, unnumbered]{proof}

\newcounter{taski}
\newcounter{taskii}[taski]
\newcounter{taskiii}[taskii]


\newcommand{\task}{\stepcounter{taski}\textbf{Aufgabe \arabic{taski}}\\}
\newcommand{\ttask}{\stepcounter{taskii}\textbf{(\alph{taskii})}\par}
\newcommand{\tttask}{\stepcounter{taskiii}\quad(\roman{taskiii})\par}

\newcommand{\defimpl}[1]{\stackrel{\text{Def.}\;#1}{\Longrightarrow}}
\newcommand{\defImpl}[1]{\stackrel{\text{Def.}\;#1}{\Longleftrightarrow}}
\newcommand{\txtimpl}[1]{\stackrel{\text{#1}}{\Longrightarrow}}
\newcommand{\txtImpl}[1]{\stackrel{\text{#1}}{\Longleftrightarrow}}
\newcommand{\refimpl}[1]{\txtimpl{\eqref{#1}}}
\newcommand{\refImpl}[1]{\txtImpl{\eqref{#1}}}

\begin{document}
\begin{flushright}
	Mike,\ Kamal\\
	Maximilian Neumann
\end{flushright}
\begin{center}
	\bfseries Analysis I Blatt 8
\end{center}
\task

\ttask
\begin{theorem}
Sei $(a_n)_{n \in \mathbb{N}}$ eine reell-wertige monoton fallende Nullfolge, so gilt
\[ \sum_{n=1}^\infty a_n \text{ konvergiert} \quad \Longleftrightarrow \quad \sum_{n=1}^\infty 2^na_{2^n} \text{ konvergiert} \]
\end{theorem}
\begin{proof}
``$\Leftarrow$'':
\begin{align*}
	\text{Monoton fallend} \quad \Rightarrow \quad &\forall n \in \mathbb{N} : \forall m \in \mathbb{N}~,~ n \leq m : a_m \leq a_n \\
	\Rightarrow \quad &\forall n,\ p \in \mathbb{N} : \sum_{k=n}^{n+p-1} a_k = a_n + \hdots + a_{n+p-1} \leq \underbrace{a_n + \hdots + a_n}_{\cdot\,p} = a_n \cdot p \\
	\Rightarrow \quad &\forall n \in \mathbb{N} : 2^na_{2^n} \geq \sum_{k=2^n}^{2^n+2^n-1} a_k = \sum_{k=2^n}^{2^{n+1}-1} a_k \\
	\Rightarrow \quad & \sum_{n=1}^\infty 2^na_{2^n} = 2a_2 + 4a_4 + 8a_8 + \hdots \geq \underbrace{(a_2 + a_3)}_{2a_2} + \underbrace{(a_4 + a_5 + a_6 + a_7)}_{4a_4} + \hdots \\
	\Rightarrow \quad & a_1 + \sum_{n=1}^\infty 2^na_{2^n} \geq \sum_{n=1}^\infty a_n \\
	\Rightarrow \quad & \sum_{n=1}^\infty a_n \text{ konvergiert}
\end{align*}
``$\Rightarrow$'':
\begin{align*}
	\text{Monoton fallend} \quad \Rightarrow \quad &\forall n \in \mathbb{N} : \forall m \in \mathbb{N}~,~ n \leq m : a_m \leq a_n \\
	\Rightarrow \quad &\forall n,\ p \in \mathbb{N} : \sum_{k=n-p}^{n-1} a_k = a_{n-p} + \hdots + a_{n-1} \geq \underbrace{a_{n-1} + \hdots + a_{n-1}}_{\cdot\,p} = a_{n-1} \cdot p \\
	\Rightarrow \quad &\forall n \in \mathbb{N} : 2^{n+1}a_{2^{n+1}} = 2 \cdot 2^na_{2^{n+1}} \leq 2\sum_{k=2^{n+1}+1-2^n}^{2^{n+1}} a_k = 2\sum_{k=2^n+1}^{2^{n+1}} a_k \\
	\Rightarrow \quad & \sum_{n=1}^\infty 2^na_{2^n} = 2a_2 + 4a_4 + 8a_8 + \hdots \leq 2\left( \underbrace{(a_3 + a_4)}_{4a_4} + \underbrace{(a_4 + a_5 + a_6 + a_7)}_{8a_8} + \hdots \right)  \\
	\Rightarrow \quad & 2a_1 + 2a_2 + \sum_{n=1}^\infty 2^na_{2^n} \leq 2\sum_{n=1}^\infty a_n \\
	\Rightarrow \quad & \sum_{n=1}^\infty 2^na_{2^n} \text{ konvergiert}
\end{align*}
\end{proof}

\task

\ttask\ Nach Quotientenkriterium für ein beliebiges $n \in \mathbb{N}$ , $n > 1$ gilt
\begin{align*}
	\left\lvert \frac{a_{n+1}}{a_n} \right\rvert &= \frac{n+1}{2^{n+1}}\,\frac{2^n}{n} = \frac{1}{2}\frac{n+1}{n} < 1
\end{align*}
Demnach ist $\sum_{n=2}^\infty \frac{n}{2^n}$ absolut konvergent. Da der erste Term $\frac{1}{2}$ hinzuaddiert werden kann, ist ebenfalls die Reihe ab $n=1$ konvergent.

\ttask\ Nach Wurzelkriterium für ein beliebiges $n \in \mathbb{N}$ gilt
\begin{align*}
	\sqrt[n]{\left( \frac{n+1}{2n+1} \right)^n } = \frac{n+1}{2n+1} < 1
\end{align*}
sodass die Reihe konvergent ist.

\ttask\ Für ein beliebiges $n \in \mathbb{N}$ gilt
\begin{align*}
	\frac{n+1}{n} &> 1 \\
	\Rightarrow \quad \left( \frac{n+1}{n} \right)^{n+1} &> 1
\end{align*}
Sodass nach Minorantenkriterium die Reihe divergiert, da
\[ \sum_{n=1}^\infty 1 = \infty \]

\task

\begin{align*}
	S_1 &= 1 + \frac 1 3 - \frac 1 2 - \frac 1 4 + \hdots = \sum_{n=1}^\infty \left( \frac{1}{4n-3} + \frac{1}{4n-1} - \frac{1}{4n-2} - \frac{1}{4n} \right) \\
	&=  \sum_{n=1}^\infty \left( \frac{1}{4n-3} - \frac{1}{4n-2} + \frac{1}{4n-1} - \frac{1}{4n} \right) = 1 - \frac 1 2 + \frac 1 3 - \frac 1 4 + \hdots \\
	&= \sum_{n=1}^\infty \frac{(-1)^{n-1}}{n} = a \\
	\\
	S_2 &= 1 - \frac 1 2 - \frac 1 4 + \frac 1 3 - \frac 1 6 - \frac 1 8 + \hdots + \frac{1}{2n-1} - \frac{1}{4n-2} - \frac{1}{4n} = \sum_{n=1}^\infty \left( \frac{1}{2n-1} - \frac{1}{4n-2} - \frac{1}{4n} \right) \\
	&= \sum_{n=1}^\infty \left( \frac{1}{2n-1} - \frac{1}{2}\frac{1}{2n-1} - \frac{1}{4n} \right) = \frac{1}{2}\sum_{n=1}^\infty \left( \frac{1}{2n-1} - \frac{1}{2n} \right) = \frac{1}{2} \left( 1 - \frac 1 2 + \frac 1 3 - \frac 1 4 + \hdots  \right) \\
	&= \frac{1}{2}\sum_{n=1}^\infty \frac{(-1)^{n-1}}{n} = \frac{1}{2}a
\end{align*}
\[  \ell(\alpha) = \ell_0 \frac{\alpha}{\cos \alpha} \]
\end{document}
