\documentclass[a4paper, 12pt]{scrartcl}

\usepackage[utf8]{inputenc}
\usepackage[T1]{fontenc}
\usepackage[ngerman]{babel}

\usepackage{amssymb}
\usepackage{amsmath}
\usepackage{framed}
\usepackage{float}
\usepackage{mathtools}

\usepackage{tikz}

\usepackage{amsthm}
\usepackage{thmtools}
\usepackage{marvosym}

\usepackage{url}
\usepackage[hidelinks]{hyperref}

\usepackage[left=1.8cm, right=1.8cm, top=1cm]{geometry}

\allowdisplaybreaks
\setkomafont{paragraph}{\normalfont\itshape}

\setlength{\parindent}{0pt}

\declaretheoremstyle[%
  spaceabove=0,%
  spacebelow=6pt,%
  headfont=\normalfont\itshape,%
  postheadspace=1em,%
  headpunct={}
]{mystyle}

\declaretheorem[name={Behauptung}, style=mystyle, unnumbered]{theorem}
\declaretheorem[name={Lemma}, style=mystyle]{lemma}
\declaretheorem[name={Voraussetzung}, style=mystyle, unnumbered]{precondition}
\let\proof\oldproof
\declaretheorem[name={Beweis}, style=mystyle, qed=\qedsymbol, unnumbered]{proof}

\newcounter{taski}
\newcounter{taskii}[taski]
\newcounter{taskiii}[taskii]


\newcommand{\task}{\stepcounter{taski}\textbf{Aufgabe \arabic{taski}}\\}
\newcommand{\ttask}{\stepcounter{taskii}\textbf{(\alph{taskii})}\par}
\newcommand{\tttask}{\stepcounter{taskiii}\quad(\roman{taskiii})\par}

\newcommand{\defimpl}[1]{\stackrel{\text{Def.}\;#1}{\Longrightarrow}}
\newcommand{\defImpl}[1]{\stackrel{\text{Def.}\;#1}{\Longleftrightarrow}}
\newcommand{\txtimpl}[1]{\stackrel{\text{#1}}{\Longrightarrow}}
\newcommand{\txtImpl}[1]{\stackrel{\text{#1}}{\Longleftrightarrow}}
\newcommand{\refimpl}[1]{\txtimpl{\eqref{#1}}}
\newcommand{\refImpl}[1]{\txtImpl{\eqref{#1}}}

\begin{document}
\begin{flushright}
	Mike,\ Kamal\\
	Maximilian Neumann
\end{flushright}
\begin{center}
	\bfseries Analysis I Blatt 8
\end{center}
\task

\ttask\ Nach Quotientenkriterium für ein beliebiges $n \in \mathbb{N}$ , $n > 1$ gilt
\begin{align*}
	\left\lvert \frac{a_{n+1}}{a_n} \right\rvert &= \frac{n+1}{2^{n+1}}\,\frac{2^n}{n} = \frac{1}{2}\frac{n+1}{n} < 1
\end{align*}
Demnach ist $\sum_{n=2}^\infty \frac{n}{2^n}$ absolut konvergent ist. Da der erste Term $\frac{1}{2}$ hinzuaddiert werden kann, ist ebenfalls die Reihe ab $n=1$ konvergent.

\ttask\ Nach Wurzelkriterium für ein beliebiges $n \in \mathbb{N}$ gilt
\begin{align*}
	\sqrt[n]{\left( \frac{n+1}{2n+1} \right)^n } = \frac{n+1}{2n+1} < 1
\end{align*}
sodass die Reihe konvergent ist.

\ttask\ Für ein beliebiges $n \in \mathbb{N}$ gilt
\begin{align*}
	\frac{n+1}{n} &> 1 \\
	\Rightarrow \quad \left( \frac{n+1}{n} \right)^{n+1} &> 1
\end{align*}
Sodass nach Minorantenkriterium die Reihe divergiert, da
\[ \sum_{n=1}^\infty 1 = \infty \]
\end{document}
