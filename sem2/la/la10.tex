\documentclass[a4paper, 12pt]{scrartcl}

\usepackage[utf8]{inputenc}
\usepackage[T1]{fontenc}
\usepackage[ngerman]{babel}

\usepackage{amssymb}
\usepackage{amsmath}
\usepackage{physics}
\usepackage{framed}
\usepackage{float}
\usepackage{mathtools}
\usepackage{marvosym}
\usepackage{bbm}

\usepackage{tikz}
\usepackage{chngcntr}

\usepackage{amsthm}
\usepackage{thmtools}

\usepackage[left=2cm, right=2cm, top=2cm]{geometry}

\allowdisplaybreaks

\setlength{\parindent}{0pt}

\setkomafont{paragraph}{\normalfont\itshape}


\declaretheoremstyle[%
  spaceabove=0,%
  spacebelow=6pt,%
  headfont=\normalfont\itshape,%
  postheadspace=1em,%
  headpunct={}
]{mystyle}

\declaretheorem[name={Behauptung}, style=mystyle, unnumbered]{theorem}
\declaretheorem[name={Lemma}, style=mystyle]{lemma}
\declaretheorem[name={Voraussetzung}, style=mystyle, unnumbered]{precondition}
\declaretheorem[name={Bemerkung}, style=mystyle, unnumbered]{note}
\let\proof\oldproof
\declaretheorem[name={Beweis}, style=mystyle, qed=\qedsymbol, unnumbered]{proof}

\newcounter{taski}
\newcounter{taskii}[taski]
\newcounter{taskiii}[taskii]

\newcommand{\task}{\stepcounter{taski}\textbf{Aufgabe \arabic{taski}}~}
\newcommand{\ttask}{\stepcounter{taskii}\textbf{(\alph{taskii})}~}
\newcommand{\tttask}{\stepcounter{taskiii}\quad(\roman{taskiii})~}

\DeclareMathOperator*{\Lin}{Lin}
\DeclareMathOperator*{\id}{id}

\let\widetilde\oldwidetilde
\def\widetilde#1{\overset{\resizebox{\widthof{\ensuremath{#1}}+1mm}{1.5mm}{\tikz[baseline,thick]{\draw(0,0)..controls(0.25,0.25)and(0.75,-0.25)..(1,0);}}}{\ensuremath{#1}}}

\setcounter{taski}{32}
\begin{document}
\begin{center}
    \textbf{8. Abgabeblatt}\\[2em]
	\def\arraystretch{2}
    \begin{tabular}{|l|l|l|l||p{18mm}|}
        \hline
        Aufgabe 33 & Aufgabe 34 & Aufgabe 35 & Aufgabe 36 & Summe:~ \\
        \hline &&&&\\
         \hline  
    \end{tabular}
\end{center}
\begingroup
\def\arraystretch{1.5}
\begin{tabular}{p{.5\textwidth}p{.5\textwidth}}
	\hline
    Übungsgruppe: Mo 14:15 ~~ SR B& Tutor(in): Sebastian Groß\\
    Namen: Kamal Abdellatif &\\
    \hline
\end{tabular}
\endgroup\\

\task
\ttask
\begin{align*}
    \forall w \in W: f^{ad}(w) &\overset{\text{ONB}}=
    \sum_{i=1}^n \expval{f^{ad}(w),v_i}_Vv_i
    \overset{\text{adj.}}= \sum_{i=1}^n \expval{w,f(v_i)}_Vv_i
\end{align*}
\ttask
\begin{align*}
    f^{ad} &= \Gamma_V^{-1} \circ f^* \circ \Gamma_W
    = \Gamma_V^{-1} \circ \Phi_{\mathcal{B}^*} \circ \widetilde{M_{\mathcal{B}^*}^{\mathcal{C}^*}(f^*)} \circ \Phi_{\mathcal{C}^*}^{-1} \circ \Gamma_W \\
    f^{ad} &= \Phi_\mathcal{B} \circ \widetilde{M_\mathcal{B}^{\mathcal{C}}(f^{ad})} \circ \Phi_\mathcal{C}^{-1} \\
    \widetilde{M_\mathcal{B}^{\mathcal{C}}(f^{ad})}
    &= \Phi_\mathcal{B}^{-1} \circ f^{ad} \circ \Phi_\mathcal{C}
    = \Phi_\mathcal{B}^{-1} \circ \Gamma_V^{-1} \circ \Phi_{\mathcal{B}^*} \circ \widetilde{M_{\mathcal{B}^*}^{\mathcal{C}^*}(f^*)} \circ \Phi_{\mathcal{C}^*}^{-1} \circ \Gamma_W \circ \Phi_\mathcal{C} \\ 
    &= \underbrace{\qty(\Phi_{\mathcal{B}^*}^{-1} \circ \Gamma_W \circ \Phi_\mathcal{B})^{-1}}_{\displaystyle\widetilde{M_\mathcal{B}^*\qty(\expval{\cdot,\cdot}_V)^{-1}}}
    \circ \widetilde{M_{\mathcal{B}^*}^{\mathcal{C}^*}(f^*)} \circ 
    \underbrace{\Phi_{\mathcal{C}^*}^{-1} \circ \Gamma_W \circ \Phi_\mathcal{C}}_{\displaystyle\widetilde{M_\mathcal{C}^*\qty(\expval{\cdot,\cdot}_W)}}
\end{align*}
\task
\ttask
\begin{align*}
    \expval{f(\lambda v+w), \lambda v+w} &= \expval{\lambda f(v)+f(w), \lambda v}+\expval{\lambda f(v)+f(w), w} \\
    &= \lambda\expval{f(v), \lambda v} + \expval{f(w), \lambda v} + \lambda\expval{f(v),w} + \expval{f(w), w} \\
    &= \lambda\overline{\lambda}\expval{f(v),v} + \overline{\lambda}\expval{f(w), v} + \lambda\expval{f(v),w} + \expval{f(w), w} \\
    &= |\lambda|^2\expval{f(v),v} + \lambda\expval{f(v),w} + \overline\lambda\expval{f(w), v} + \expval{f(w), w}
\end{align*}
\ttask Sei $v,\, w \in V$, $v$ fest, $\lambda \in \mathbb{C}$
\begin{align*}
    0 &= 0 + 0
    = \expval{f(v+w),v+w} + \expval{f(iv+w),iv+w} \\
    &= \overbrace{\expval{f(v),v}}^{0} +\expval{f(v),w} + \expval{f(w), v} + \overbrace{\expval{f(w),w}}^{0}
    + \overbrace{\expval{f(v),v}}^{0} +\expval{f(v),w} - \expval{f(w), v} + \overbrace{\expval{f(w),w}}^{0} \\
    &= 2\expval{f(v),w} \quad\qc \expval{f(v),w} = 0 ~\forall w \in V \xRightarrow{\text{unit}} f(v) = 0 \\\\
    &\implies \forall v \in V : f(v) = 0 \implies f = 0
\end{align*}
\ttask Sei $v \in V$
\begin{align*}
    \overline{\expval{v,f(v)}} &\overset{\text{sesq.}}= \expval{f(v),v} \overset{\text{selbstadj.}}= \expval{v,f(v)} \\
    \implies \expval{v,f(v)} &\in \mathbb{R}
\end{align*}
\ttask Eine Rotation $f$ um $\frac{\pi}{2}$ im $\mathbb{R}^2$ bildet jeden Vektor auf einen senkrecht zu ihm stehenden Vektor ab, sodass $\expval{f(v),v}=0 ~\forall v \in V$, jedoch ist eine Rotation nicht die Nullabbildung.
\end{document}
