\documentclass[a4paper, 12pt]{scrartcl}

\usepackage[utf8]{inputenc}
\usepackage[T1]{fontenc}
\usepackage[ngerman]{babel}

\usepackage{amssymb}
\usepackage{amsmath}
\usepackage{physics}
\usepackage{framed}
\usepackage{float}
\usepackage{mathtools}
\usepackage{marvosym}
\usepackage{bbm}

\usepackage{tikz}
\usepackage{chngcntr}

\usepackage{amsthm}
\usepackage{thmtools}

\usepackage[left=2cm, right=2cm, top=2cm]{geometry}

\allowdisplaybreaks

\setlength{\parindent}{0pt}

\setkomafont{paragraph}{\normalfont\itshape}


\declaretheoremstyle[%
  spaceabove=0,%
  spacebelow=6pt,%
  headfont=\normalfont\itshape,%
  postheadspace=1em,%
  headpunct={}
]{mystyle}

\declaretheorem[name={Behauptung}, style=mystyle, unnumbered]{theorem}
\declaretheorem[name={Lemma}, style=mystyle]{lemma}
\declaretheorem[name={Voraussetzung}, style=mystyle, unnumbered]{precondition}
\let\proof\oldproof
\declaretheorem[name={Beweis}, style=mystyle, qed=\qedsymbol, unnumbered]{proof}

\newcounter{taski}
\newcounter{taskii}[taski]
\newcounter{taskiii}[taskii]

\newcommand{\task}{\stepcounter{taski}\textbf{Aufgabe \arabic{taski}}~}
\newcommand{\ttask}{\stepcounter{taskii}\textbf{(\alph{taskii})}~}
\newcommand{\tttask}{\stepcounter{taskiii}\quad(\roman{taskiii})~}

\setcounter{taski}{18}
\begin{document}
\begin{center}
    \textbf{5. Abgabeblatt}\\[2em]
	\def\arraystretch{2}
    \begin{tabular}{|l|l|l|l||p{18mm}|}
        \hline
        Aufgabe 17 & Aufgabe 18 & Aufgabe 19 & Aufgabe 20 & Summe:~ \\
        \hline &&&&\\
         \hline  
    \end{tabular}
\end{center}
\begingroup
\def\arraystretch{1.5}
\begin{tabular}{p{.5\textwidth}p{.5\textwidth}}
	\hline
    Übungsgruppe: Mo 14:15 ~~ SR B& Tutor(in): Sebastian Groß\\
    Namen: Ellen Bräutigam, Kamal Abdellatif &\\
    \hline
\end{tabular}
\endgroup\\

\task
\ttask
Sei
\[ X = \mqty(1 & x_1 \\ 1 & x_2 \\ \vdots & \vdots \\ 1 & x_n)
\quad,\quad y - X\beta = \mqty(y_1 \\ y_2 \\ \vdots \\ y_n) - \mqty(a+bx_1\\a+bx_2\\\vdots \\ a+x_n) \]
Da $x_i \neq x_j ~,~ i,\,j \in \qty{1,\,\dots,\,n}$, sind die Spalten von $X$ lin. unabh. Sei der UVR $U \subseteq \mathbb{R}^n$ definiert als die Lin. Hülle der Spalten von $X$.  Aus Satz (20.5) folgt, dass $\Vert y-X\beta\Vert$ minimal wird für die Projektion $X\beta = p_U(y)$.
Unter Verwendung des üblichen Skalarprodukts folgt
\[
    \inf_{(a,b)^t\in \mathbb{R}^2} \sum_{i=1}^n (y_i - (a+bx_i))^2
    =\inf_{(a,b)^t\in \mathbb{R}^2}  \left\Vert y - X\beta\right\Vert^2 = \left\Vert y - p_U(y)\right\Vert^2  
\]
\begin{align*}
    p_U(y) &= X(X^tX)^{-1}X^ty = X\mqty(n & \sum_ix_i \\\\ \sum_ix_i & \sum_ix_i^2)^{-1}\mqty(\sum_i y_i \\\\ \sum_ix_iy_i) \\
           &= X\frac{1}{n\sum_i{x_i} - \qty(\sum_ix_i)^2}\mqty(\qty(\sum_ix_i)^2 & - n\sum_i{x_i} \\\\ - n\sum_i{x_i} & n)\mqty(\sum_i y_i \\\\ \sum_ix_iy_i) \\
\end{align*}
\end{document}
