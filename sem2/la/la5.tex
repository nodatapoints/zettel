\documentclass[a4paper, 12pt]{scrartcl}

\usepackage[utf8]{inputenc}
\usepackage[T1]{fontenc}
\usepackage[ngerman]{babel}

\usepackage{amssymb}
\usepackage{amsmath}
\usepackage{physics}
\usepackage{framed}
\usepackage{float}
\usepackage{mathtools}
\usepackage{marvosym}
\usepackage{bbm}

\usepackage{tikz}
\usepackage{chngcntr}

\usepackage{amsthm}
\usepackage{thmtools}

\usepackage[left=2cm, right=2cm, top=2cm]{geometry}

\allowdisplaybreaks

\setlength{\parindent}{0pt}

\setkomafont{paragraph}{\normalfont\itshape}


\declaretheoremstyle[%
  spaceabove=0,%
  spacebelow=6pt,%
  headfont=\normalfont\itshape,%
  postheadspace=1em,%
  headpunct={}
]{mystyle}

\declaretheorem[name={Behauptung}, style=mystyle, unnumbered]{theorem}
\declaretheorem[name={Lemma}, style=mystyle]{lemma}
\declaretheorem[name={Voraussetzung}, style=mystyle, unnumbered]{precondition}
\let\proof\oldproof
\declaretheorem[name={Beweis}, style=mystyle, qed=\qedsymbol, unnumbered]{proof}

\newcounter{taski}
\newcounter{taskii}[taski]
\newcounter{taskiii}[taskii]

\newcommand{\task}{\stepcounter{taski}\textbf{Aufgabe \arabic{taski}}~}
\newcommand{\ttask}{\stepcounter{taskii}\textbf{(\alph{taskii})}~}
\newcommand{\tttask}{\stepcounter{taskiii}\quad(\roman{taskiii})~}

\setcounter{taski}{18}
\begin{document}
\begin{center}
    \textbf{5. Abgabeblatt}\\[2em]
	\def\arraystretch{2}
    \begin{tabular}{|l|l|l|l||p{18mm}|}
        \hline
        Aufgabe 17 & Aufgabe 18 & Aufgabe 19 & Aufgabe 20 & Summe:~ \\
        \hline &&&&\\
         \hline  
    \end{tabular}
\end{center}
\begingroup
\def\arraystretch{1.5}
\begin{tabular}{p{.5\textwidth}p{.5\textwidth}}
	\hline
    Übungsgruppe: Mo 14:15 ~~ SR B& Tutor(in): Sebastian Groß\\
    Namen: Ellen Bräutigam, Kamal Abdellatif &\\
    \hline
\end{tabular}
\endgroup\\

\task
\ttask
Sei
\[ X = \mqty(1 & x_1 \\ 1 & x_2 \\ \vdots & \vdots \\ 1 & x_n)
\quad,\quad y - X\beta = \mqty(y_1 \\ y_2 \\ \vdots \\ y_n) - \mqty(a+bx_1\\a+bx_2\\\vdots \\ a+x_n) \]
Da $x_i \neq x_j ~,~ i,\,j \in \qty{1,\,\dots,\,n}$, sind die Spalten von $X$ lin. unabh. Sei der UVR $U \subseteq \mathbb{R}^n$ definiert als die Lin. Hülle der Spalten von $X$.  Aus Satz (20.5) folgt, dass $\Vert y-X\beta\Vert$ minimal wird für die Projektion $X\beta = p_U(y)$.
Unter Verwendung des üblichen Skalarprodukts folgt
\[
    \inf_{(a,b)^t\in \mathbb{R}^2} \sum_{i=1}^n (y_i - (a+bx_i))^2
    =\inf_{(a,b)^t\in \mathbb{R}^2}  \left\Vert y - X\beta\right\Vert^2 = \left\Vert y - p_U(y)\right\Vert^2  
\]
\begin{align*}
    p_U(y) &= X(X^tX)^{-1}X^ty = X\mqty(n & \sum_ix_i \\\\ \sum_ix_i & \sum_ix_i^2)^{-1}\mqty(\sum_i y_i \\\\ \sum_ix_iy_i) \\
   &= X\frac{1}{n\sum_i{x_i^2} - \qty(\sum_ix_i)^2}\mqty(\qty(\sum_ix_i)^2 & - n\sum_i{x_i} \\\\ - n\sum_i{x_i} & n)\mqty(\sum_i y_i \\\\ \sum_ix_iy_i) \\
   &= X\frac{1}{\frac{1}{n}\sum_i{x_i^2} -\bar{x}^2} \mqty(\bar{x}^2 & -\bar{x} \\ -\bar{x} & \frac{1}{n}) \mqty(n\bar{y} \\ \sum_ix_iy_i)
   = X\frac{1}{\frac{1}{n}\sum_i(x_i^2-\bar{x}^2)} \mqty(n\bar{x}^2\bar{y} -\bar{x}\sum_ix_iy_i \\
	-n\bar{x}\bar{y} + \frac{1}{n}\sum_ix_iy_i) \\
	\frac{1}{n}\sum_i(x_i^2-\bar{x}^2) &= S_{xx}+2\bar{x}\frac{1}{n}\sum_ix_i - 2\bar{x}^2 = S_{xx} \\
	b &= \frac{1}{S_{xx}} \qty(-n\bar{x}\bar{y} + \frac{1}{n}\sum_ix_iy_i)
	= \frac{1}{S_{xx}} \qty(n\bar{x}\bar{y} \overbrace{- \sum_i\bar{x}y_i - \sum_i\bar{y}x_i}^{-2n\bar{x}\bar{y}} + \frac{1}{n}\sum_ix_iy_i) \\
	  &= \frac{1}{S_{xx}} \frac{1}{n}\sum(x_iy_i - \bar{x}y_i - \bar{y}x_i + \bar{x}\bar{y}) = \frac{S_{xy}}{S_{xx}}  \\
	a &= \frac{1}{S_{xx}}\qty(n\bar{x}^2\bar{y}-\bar{x} \sum_ix_iy_i)
\end{align*}
\newpage
\ttask
\[ \hat{a} = -3 \qquad \hat{b} = \frac{13}{5}  \]
\begin{center}
	\includegraphics[width=.7\textwidth]{polyfit}
\end{center}
\task
\ttask
Mit der Billinearform $\gamma_A$ wobei $A = \mathrm{diag}(w_1,\, \dots,\, w_n)$ gilt nach \textsc{Cauchy-Schwarz} für $a,\, b \in \mathbb{R}^n$
\begin{align*}
	\gamma_A(a,\, b) &\leq \gamma_A(a,\, a) \cdot \gamma_A(b,\, b) \\
	\sum_{i=1}^nw_ia_ib_i &\leq \qty(\sum_{i=1}^nw_ia_i^2)\cdot\qty(\sum_{i=1}^nw_ib_i^2)
\end{align*}
\ttask Es sei die Bilinearform $\gamma_B$ mit $B = \frac{1}{n}E_n$.  $a = (x_1^{-1},\, \dots,\, x_n^{-1})^t$, $b = (x_1,\, \dots,\, x_n)^t$.

$\gamma_B(a,\, b) = 1$
\[  \]
\end{document}
