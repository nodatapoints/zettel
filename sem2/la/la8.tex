\documentclass[a4paper, 12pt]{scrartcl}

\usepackage[utf8]{inputenc}
\usepackage[T1]{fontenc}
\usepackage[ngerman]{babel}

\usepackage{amssymb}
\usepackage{amsmath}
\usepackage{physics}
\usepackage{framed}
\usepackage{float}
\usepackage{mathtools}
\usepackage{marvosym}
\usepackage{bbm}

\usepackage{tikz}
\usepackage{chngcntr}

\usepackage{amsthm}
\usepackage{thmtools}

\usepackage[left=2cm, right=2cm, top=2cm]{geometry}

\allowdisplaybreaks

\setlength{\parindent}{0pt}

\setkomafont{paragraph}{\normalfont\itshape}


\declaretheoremstyle[%
  spaceabove=0,%
  spacebelow=6pt,%
  headfont=\normalfont\itshape,%
  postheadspace=1em,%
  headpunct={}
]{mystyle}

\declaretheorem[name={Behauptung}, style=mystyle, unnumbered]{theorem}
\declaretheorem[name={Lemma}, style=mystyle]{lemma}
\declaretheorem[name={Voraussetzung}, style=mystyle, unnumbered]{precondition}
\let\proof\oldproof
\declaretheorem[name={Beweis}, style=mystyle, qed=\qedsymbol, unnumbered]{proof}

\newcounter{taski}
\newcounter{taskii}[taski]
\newcounter{taskiii}[taskii]

\newcommand{\task}{\stepcounter{taski}\textbf{Aufgabe \arabic{taski}}~}
\newcommand{\ttask}{\stepcounter{taskii}\textbf{(\alph{taskii})}~}
\newcommand{\tttask}{\stepcounter{taskiii}\quad(\roman{taskiii})~}

\DeclareMathOperator*{\Spur}{Spur}
\DeclareMathOperator*{\Lin}{Lin}

\setcounter{taski}{29}
\begin{document}
\begin{center}
    \textbf{8. Abgabeblatt}\\[2em]
	\def\arraystretch{2}
    \begin{tabular}{|l|l|l|l||p{18mm}|}
        \hline
        Aufgabe 29 & Aufgabe 30 & Aufgabe 31 & Aufgabe 32 & Summe:~ \\
        \hline &&&&\\
         \hline  
    \end{tabular}
\end{center}
\begingroup
\def\arraystretch{1.5}
\begin{tabular}{p{.5\textwidth}p{.5\textwidth}}
	\hline
    Übungsgruppe: Mo 14:15 ~~ SR B& Tutor(in): Sebastian Groß\\
    Namen: Ellen Bräutigam, Kamal Abdellatif &\\
    \hline
\end{tabular}
\endgroup\\

\task
\ttask Erweitert man $K$ auf den Zerfällungskörper $\mathbb{C}$, so zerfällt $\chi_\varphi$ in Linearfaktoren. Aus P0.4 folgt, dass für jede komplexe Nullstelle $z_i$ von $\varphi$ $\overline{z_i}$ ebenfalls eine Nullstelle ist. Genau dann wenn $z_i \in \mathbb{R}$ ist $\overline{z_i} = z_i$. Andernfalls geht $(t-\overline{z_i})$ als weiterer Linearfaktor in $\chi_\varphi$ ein:
\[ \chi_\varphi = (t-\lambda_1) \cdot \hdots \cdot (t-\lambda_r) \cdot (t-z_1)(t-\overline{z_1}) \cdot \hdots \cdot (t-z_m)(t-\overline{z_m}) \]
Es sei $q_i \coloneqq (t-z_i)(t-\overline{z_i}) ~,~ i \in \qty{1,\,\dots,\,m}$
\[ q_i = t^2 - \overbrace{(z_i + \overline{z_i})}^{\in~\mathbb{R}}t + \overbrace{z_i\overline{z_i}}^{\in~\mathbb{R}} \in \mathbb{R}[t] \]
Da $q_i$ zwei eindeutige komplexe Nullstellen besitzt, ist es demnach ein quadratisches Polynom aus $\mathbb{R}[t]$ ohne reelle Nullstellen.

\ttask
\begin{theorem}
    Es gibt einen $\varphi$-invarianten UVR $W \subseteq V$ mit $\dim_{\mathbb{R}} \in \qty{1,\,2}$.
\end{theorem}
\begin{proof} Nach \textbf{(a)} kann $\chi_\varphi$ mit $\lambda_1,\,\dots,\,\lambda_r \in \mathbb{R}$ und quadratischen Polynomen $q_1,\,\dots,\,q_m \in \mathbb{R}[t]$ ohne reelle Nullstellen aufgeteilt werden sodass
\[ \chi_\varphi = (t-\lambda_1) \cdot \hdots \cdot (t-\lambda_r) \cdot q_1 \cdot \hdots \cdot q_m \]
Fall 1: $r > 0$ Es gibt demnach eine reelle Nullstelle $\lambda_1$ von $\chi_\varphi$. Es wurde in der VL gezeigt dass es ein Eigenpaar $(\lambda_1, v_1)$ von $\varphi$ gibt. Demnach 
\begin{align*}
    \varphi(v) &= \lambda_1 v \quad \forall v \in \Lin(\qty{v_1}) \eqqcolon W\\
    \varphi(W) &= \Lin(\qty{\lambda_1v_1}) = W\qc \dim_{\mathbb{R}}W = 1 \qq{da} v_1 \neq 0
\end{align*}
Fall 2: $r = 0$. $\varphi$ besitzt keine reellen Nullstellen. Nach \textsc{Cayley-Hamilton} gilt
\[ \chi_\varphi(\varphi) = 0 \]
Sei $\chi_\varphi = \sum_{i=0}^{2m} a_i t^i$ mit $a_0,\,\dots,\,a_{2m} \in \mathbb{R}$. Da $W$ $\varphi$-invariant gilt $\varphi^i(W) \subseteq W$. Sei $w \in W$.
\begin{align*}
    \chi_\varphi(\varphi)(w) &= \sum_{i=0}^{2m} a_i \varphi^i(w)
\end{align*}
\end{proof}
\end{document}
