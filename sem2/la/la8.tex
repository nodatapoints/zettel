\documentclass[a4paper, 12pt]{scrartcl}

\usepackage[utf8]{inputenc}
\usepackage[T1]{fontenc}
\usepackage[ngerman]{babel}

\usepackage{amssymb}
\usepackage{amsmath}
\usepackage{physics}
\usepackage{framed}
\usepackage{float}
\usepackage{mathtools}
\usepackage{marvosym}
\usepackage{bbm}

\usepackage{tikz}
\usepackage{chngcntr}

\usepackage{amsthm}
\usepackage{thmtools}

\usepackage[left=2cm, right=2cm, top=2cm]{geometry}

\allowdisplaybreaks

\setlength{\parindent}{0pt}

\setkomafont{paragraph}{\normalfont\itshape}


\declaretheoremstyle[%
  spaceabove=0,%
  spacebelow=6pt,%
  headfont=\normalfont\itshape,%
  postheadspace=1em,%
  headpunct={}
]{mystyle}

\declaretheorem[name={Behauptung}, style=mystyle, unnumbered]{theorem}
\declaretheorem[name={Lemma}, style=mystyle]{lemma}
\declaretheorem[name={Voraussetzung}, style=mystyle, unnumbered]{precondition}
\declaretheorem[name={Bemerkung}, style=mystyle, unnumbered]{note}
\let\proof\oldproof
\declaretheorem[name={Beweis}, style=mystyle, qed=\qedsymbol, unnumbered]{proof}

\newcounter{taski}
\newcounter{taskii}[taski]
\newcounter{taskiii}[taskii]

\newcommand{\task}{\stepcounter{taski}\textbf{Aufgabe \arabic{taski}}~}
\newcommand{\ttask}{\stepcounter{taskii}\textbf{(\alph{taskii})}~}
\newcommand{\tttask}{\stepcounter{taskiii}\quad(\roman{taskiii})~}

\DeclareMathOperator*{\Spur}{Spur}
\DeclareMathOperator*{\Lin}{Lin}
\DeclareMathOperator*{\id}{id}

\setcounter{taski}{29}
\begin{document}
\begin{center}
    \textbf{8. Abgabeblatt}\\[2em]
	\def\arraystretch{2}
    \begin{tabular}{|l|l|l|l||p{18mm}|}
        \hline
        Aufgabe 29 & Aufgabe 30 & Aufgabe 31 & Aufgabe 32 & Summe:~ \\
        \hline &&&&\\
         \hline  
    \end{tabular}
\end{center}
\begingroup
\def\arraystretch{1.5}
\begin{tabular}{p{.5\textwidth}p{.5\textwidth}}
	\hline
    Übungsgruppe: Mo 14:15 ~~ SR B& Tutor(in): Sebastian Groß\\
    Namen: Ellen Bräutigam, Kamal Abdellatif &\\
    \hline
\end{tabular}
\endgroup\\

\task
\ttask Erweitert man $K$ auf den Zerfällungskörper $\mathbb{C}$, so zerfällt $\chi_\varphi$ in Linearfaktoren. Aus P0.4 folgt, dass für jede komplexe Nullstelle $z_i$ von $\varphi$ $\overline{z_i}$ ebenfalls eine Nullstelle ist. Genau dann wenn $z_i \in \mathbb{R}$ ist $\overline{z_i} = z_i$. Andernfalls geht $(t-\overline{z_i})$ als weiterer Linearfaktor in $\chi_\varphi$ ein:
\[ \chi_\varphi = (t-\lambda_1) \cdot \hdots \cdot (t-\lambda_r) \cdot (t-z_1)(t-\overline{z_1}) \cdot \hdots \cdot (t-z_m)(t-\overline{z_m}) \]
Es sei $q_i \coloneqq (t-z_i)(t-\overline{z_i}) ~,~ i \in \qty{1,\,\dots,\,m}$
\[ q_i = t^2 - \overbrace{(z_i + \overline{z_i})}^{\in~\mathbb{R}}t + \overbrace{z_i\overline{z_i}}^{\in~\mathbb{R}} \in \mathbb{R}[t] \]
Da $q_i$ zwei eindeutige komplexe Nullstellen besitzt, ist es demnach ein quadratisches Polynom aus $\mathbb{R}[t]$ ohne reelle Nullstellen.

\ttask
\begin{theorem}
    Es gibt einen $\varphi$-invarianten UVR $W \subseteq V$ mit $\dim_{\mathbb{R}} \in \qty{1,\,2}$.
\end{theorem}
\begin{proof} Nach \textbf{(a)} kann $\chi_\varphi$ mit $\lambda_1,\,\dots,\,\lambda_r \in \mathbb{R}$ und quadratischen Polynomen $q_1,\,\dots,\,q_m \in \mathbb{R}[t]$ ohne reelle Nullstellen aufgeteilt werden sodass
\[ \chi_\varphi = (t-\lambda_1) \cdot \hdots \cdot (t-\lambda_r) \cdot q_1 \cdot \hdots \cdot q_m \]
Fall 1: $r > 0$ Es gibt demnach eine reelle Nullstelle $\lambda_1$ von $\chi_\varphi$. Es wurde in der VL gezeigt dass es ein Eigenpaar $(\lambda_1, v_1)$ von $\varphi$ gibt. Demnach 
\begin{align*}
    \varphi(v) &= \lambda_1 v \quad \forall v \in \Lin(\qty{v_1}) \eqqcolon W\\
    \varphi(W) &= \Lin(\qty{\lambda_1v_1}) = W\qc \dim_{\mathbb{R}}W = 1 \qq{da} v_1 \neq 0
\end{align*}
Fall 2: $r = 0$. $\varphi$ besitzt keine reellen Nullstellen. Nach \textsc{Cayley-Hamilton} gilt
\[ \chi_\varphi(\varphi) = 0 \]
Sei $\chi_\varphi = \sum_{i=0}^{2m} a_i t^i$ mit $a_0,\,\dots,\,a_{2m} \in \mathbb{R}$. Da $W$ $\varphi$-invariant gilt $\varphi^i(W) \subseteq W$. Sei $w \in W$.
\begin{align*}
    \chi_\varphi(\varphi)(w) &= \sum_{i=0}^{2m} a_i \varphi^i(w)
\end{align*}
\end{proof}
\ttask
\begin{theorem}
    Sei $\varphi$ eine Isometrie und $W \subseteq V$ ein $\varphi$-invarianter UVR.
\end{theorem}

\ttask
\begin{theorem}
    Sei $\varphi$ eine Isometrie. Dann gibt es stets eine Basis $\mathcal{B} = (v_1,\, \dots,\, v_n)$, sodass $M^{\mathcal{B}}_{\mathcal{B}}(\varphi)$ in Isometrie-Normalform ist.
\end{theorem}
\begin{proof} durch vollständige Induktion \\
\emph{Induktionsanfang}

$n = 1$ \quad Die Isometrien auf einem $\mathbb{R}$-VRs $V$ mit $\dim_K V = 1$ sind $\id_V$ und $-\id_V$. Demnach ist $M^{\mathcal{B}}_{\mathcal{B}}(\varphi)$ entweder $E_1$ oder $-E_1$, welche beide in Isometrie-Normalform gegeben sind durch
\[ (r,\, s,\, k) \in \qty{(1,0,0),\, (0,1,0)} \]

$n = 2$ \quad Es wurde in der VL gezeigt, dass jede Isometrie eines $\mathbb{R}$-VRs $V$ mit $\dim_ \mathbb{R} = 2$ bezüglich einer ONB sich durch ein Drehkästchen $A(\alpha)$ mit $\alpha \in [0,2\pi)$ darstellen lässt. Für die zwei Spezialfälle $0, \pi$ gilt
\[ A(0) = E_2 \qquad A(\pi) = -E_2 \]
Demnach ist die Isometrie-Normalform immer gegeben durch eine der folgenden Kombinationen
\[ (r,\, s,\, k) \in \qty{(1, 0, 0),\, (0,1,0),\, (0, 0, 1)} \]

\emph{Induktionsbehauptung} Für alle $\mathbb{R}$-VR $V$ mit $\dim_ \mathbb{R} V < n$ gibt es ONB, sodass $\varphi$ in Isometrie-Normalform vorliegt.

\emph{Induktionsschritt} Für einen $\mathbb{R}$-VR $V$ mit $\dim_ \mathbb{R} V = n$ gibt es eine ONB $\mathcal{C}$, sodass $\varphi$ in Isometrie-Normalform vorliegt.

Aus \textbf{(b)} folgt die Existenz eines $\varphi$-invarianten UVRs $W \subseteq V$ mit $\dim_K W \in \qty{1,\, 2}$. Nach Induktionsanfang gibt es eine Basis $\mathcal{B}$ mit $M^{\mathcal{B}}_{\mathcal{B}}\qty(\varphi\eval_W)$ in Isometrie-Normalform.

Nach \textbf{(c)} ist $W^\perp$ ebenfalls $\varphi$-invariant. Da
\[ V = W \oplus W^\perp \qc n = \dim_ \mathbb{R} V = \overbrace{\dim_ \mathbb{R} W}^{\geq 1} + \dim_ \mathbb{R} W^\perp \qc \dim_ \mathbb{R} W^\perp < n, \]
gibt es nach Induktionsbehauptung für $W^\perp$ eine ONB $\mathcal{C}$ mit $M^{\mathcal{C}}_{\mathcal{C}}\qty(\varphi\eval_{W^\perp})$ in Isometrie-Normalform. Da $V = W \oplus W^\perp$ ergibt die Vereinigung $\mathcal{B}'$ der Basen $\mathcal{B}$ und $\mathcal{C}$ eine Basis von $V$, wobei sich die gesamte Darstellungsmatrix für $\varphi$ sich ergibt als
\[ M_{\mathcal{B}'}^{\mathcal{B}'}(\varphi) = \mqty(\diagonalmatrix{M^{\mathcal{B}}_{\mathcal{B}}\qty(\varphi\eval_{W}), M^{\mathcal{C}}_{\mathcal{C}}\qty(\varphi\eval_{W^\perp})}) \]
welches ebenfalls in Isometrie-Normalform vorliegt.
\end{proof}
\begin{note}
    Aus der Induktion geht weiterhin hervor, dass sich $V$ immer aus $\varphi$-invarianten UVRs $W_1,\,\dots,\,  W_m \subseteq V ~,~m \leq n$ mit $\dim_ \mathbb{R} W_i \in \qty{1,\, 2}$ darstellen lässt sodass
    \[ V = \bigoplus_{i=1}^m W_i \]
\end{note}
\setcounter{taski}{31}
\newpage
\task
\begin{lemma}
    Jede Isometrie $\varphi$ des euklidischen Raums $(\mathbb{R}^2, \expval{\cdot,\cdot})$ lässt sich aus höhstens zwei Spiegelungen darstellen.
\end{lemma}
\begin{proof}
Sei $v \in \mathbb{R}^2 ~,~ \varphi$ Isometrie. Man spiegele zuerst entlang $v$ und dann an $v + \varphi(v)$. 
\begin{align*}
    s_v(v) &= v - 2\frac{\expval{v,v}}{\expval{v,v}}v = -v \\
    s_{v + \varphi(v)}(-v) &= -v - 2 \cdot \frac{\expval{-v, v+\varphi(v)}}{\norm{v+\varphi(v)}^2} \cdot (v+\varphi(v))
    = -v + 2\frac{\expval{v,v} + \expval{v,\varphi(v)}}{\norm{v+\varphi(v)}^2}\cdot (v+\varphi(v)) \\
                           &= -v + \frac{\norm{v+\varphi(v)}^2 + \overbrace{\norm{v}^2 -\norm{\varphi(v)}^2}^0}{\norm{v+\varphi(v)}^2}\cdot (v+\varphi(v)) = -v + v + \varphi(v) = \varphi(v) \\\\
    \implies \varphi&= s_{v+\varphi(v)} \circ s_v 
\end{align*}
\end{proof}

\begin{theorem}
    Jede Isometrie auf einem euklidischen Raum $(\mathbb{R}^n,\, \expval{\cdot,\cdot})$ lässt sich als Verkettung von höhstens $n$ Spiegelungen darstellen.
\end{theorem}
\begin{proof} Sei $V = \mathbb{R}^n$, $\varphi$ Isometrie.
Nach der Bemerkung gibt es $\varphi$-invariante UVRs $W_1,\,\dots,\,  W_m \subseteq V ~,~m \leq n$ mit $\dim_ \mathbb{R} W_i \in \qty{1,\, 2}$ sodass
\[ V = \bigoplus_{i=1}^m W_i \tag{$\ast$} \]
Betrachte man eine Einschränkung $\varphi_i \coloneqq \varphi\eval_{W_i}$. Wenn $\dim_ \mathbb{R} W_i = 1$, so ist $\varphi_i \in \qty{\id_{W_i},\, -\id_{W_i}}$. Letzteres lässt sich darstellen durch
\[ -\textrm{id}_{W_i} = s_w \qquad w \in W_i \quad (W_i \neq \qty{0}) \]
Aus dem Lemma folgt, dass für $\dim_ \mathbb{R} W_i = 2$ sich $\varphi_i$ als Verkettung zweier Spiegelungen darstellen lässt.

Aus $(\ast)$ folgt
\[ \varphi = \varphi_1 \circ \cdots \circ \varphi_m \]
wobei jedes $\varphi_i$ keiner, einer oder zwei Spiegelungen entspricht. Die Anzahl der Spiegelungen in einem $\varphi_i$ ist dabei kleiner gleich $\dim_ \mathbb{R} W_i$, sodass insgesamt maximal $n$ Spiegelungen verkettet werden.
\end{proof}
\end{document}
