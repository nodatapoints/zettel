\documentclass[a4paper, 12pt]{scrartcl}

\usepackage[utf8]{inputenc}
\usepackage[T1]{fontenc}
\usepackage[ngerman]{babel}

\usepackage{amssymb}
\usepackage{amsmath}
\usepackage{physics}
\usepackage{framed}
\usepackage{float}
\usepackage{mathtools}
\usepackage{marvosym}
\usepackage{bbm}

\usepackage{tikz}
\usepackage{chngcntr}

\usepackage{amsthm}
\usepackage{thmtools}

\usepackage[left=2cm, right=2cm, top=2cm]{geometry}

\allowdisplaybreaks

\setlength{\parindent}{0pt}

\setkomafont{paragraph}{\normalfont\itshape}


\declaretheoremstyle[%
  spaceabove=0,%
  spacebelow=6pt,%
  headfont=\normalfont\itshape,%
  postheadspace=1em,%
  headpunct={}
]{mystyle}

\declaretheorem[name={Behauptung}, style=mystyle, unnumbered]{theorem}
\declaretheorem[name={Lemma}, style=mystyle]{lemma}
\declaretheorem[name={Voraussetzung}, style=mystyle, unnumbered]{precondition}
\let\proof\oldproof
\declaretheorem[name={Beweis}, style=mystyle, qed=\qedsymbol, unnumbered]{proof}

\newcounter{taski}
\newcounter{taskii}[taski]
\newcounter{taskiii}[taskii]

\newcommand{\task}{\stepcounter{taski}\textbf{Aufgabe \arabic{taski}}~}
\newcommand{\ttask}{\stepcounter{taskii}\textbf{(\alph{taskii})}~}
\newcommand{\tttask}{\stepcounter{taskiii}\quad(\roman{taskiii})~}

\setcounter{taski}{5}
\begin{document}
\begin{center}
    \begingroup\bfseries
    Lineare Algebra 2 \\
    2. Abgabeblatt \\[1em]
    \endgroup
    \def\arraystretch{2}
    \begin{tabular}{|c|c|c|c||p{18mm}|}
        \hline
        Aufgabe 5 & Aufgabe 6 & Aufgabe 7 & Aufgabe 8 & Summe:\\
        \hline
        ~ & ~ & ~ & ~ & ~ \\
        \hline
    \end{tabular}

    \vspace{1em}
    \def\arraystretch{1.5}
    \begin{tabular}{p{.5\textwidth}p{.5\textwidth}}
        \hline
        Übungsgruppe: Mo 14:15 ~~ SR B& Tutor(in): Sebastian Groß\\
        Namen: Ellen Bräutigam, Kamal Abdellatif & \\\\
        \hline
    \end{tabular}
\end{center}
\task\ 
\ttask
\[ M_\mathcal{B}^\mathcal{B}(\varphi_1) = \mqty(1 & 1 & 1 \\ 0 & 3 & 2 \\ 0 & 0 & 1)
\qquad M_\mathcal{B}^\mathcal{B}(\varphi_2) = \mqty(1 & 0 & 0 \\ 1 & 1 & 1 \\ 2 & 0 & 2) \]

\ttask
\begin{align*}
    \chi_{\varphi_1} &= \det\qty(M_\mathcal{B}^\mathcal{B}(\varphi_1) - t \mathbbm{1}) = (t-1)^2(t-3) \\
    \chi_{\varphi_2} &= \det\qty(M_\mathcal{B}^\mathcal{B}(\varphi_2) - t \mathbbm{1}) = (t-1)^2(t-2)
\end{align*}
\ttask
Alle Nullstellen des charakteristischen Polynoms sind auch Nullstellen des Minimalpolynoms. Durch variieren der Exponenten:
\begin{align*}
    \mu_{\varphi_1} &= (t-1)(t-3) \\
    \mu_{\varphi_2} &= (t-1)^2(t-2)
\end{align*}
\ttask
Nach VL ist eine lineare Abbildung $\varphi$ genau dann diagonalisierbar, wenn $\mu_ \varphi$ in Linearfaktoren zerfällt. Dies ist nur bei $\varphi_1$ der Fall.
\begin{align*}
    \mathbf{\lambda = 1}: \qquad \qty(M_\mathcal{B}^\mathcal{B}(\varphi_1) - \mathbbm 1)b
    &= \mqty(0 & 1 & 1 \\ 0 & 2 & 2 \\ 0 & 0 & 0)b = 0 \qquad
    &\xRightarrow{\text{Hinsehen}} \quad b_1 &= \mqty(1 \\ 0 \\ 0) \quad b_2 = \mqty(0 \\ 1 \\ -1) \\
    \mathbf{\lambda = 3}: \qquad \qty(M_\mathcal{B}^\mathcal{B}(\varphi_1) - 3\mathbbm 1)b
    &= \mqty(-2 & 1 & 1 \\ 0 & 0 & 2 \\ 0 & 0 & -2)b = 0 \qquad
    &\xRightarrow{\text{Hinsehen}} \quad b_3 &= \mqty(1 \\ 2 \\ 0) \\\\
    \mathcal{B}' &= (b_1,\,b_2,\,b_3) & M_\mathcal{B'}^\mathcal{B'}(\varphi_1) &= \mqty(\diagonalmatrix{1, 1, 3})
\end{align*}

\newpage

\task
\ttask
\begin{gather*}
    f^2 = f \quad \curvearrowright \quad \mu_f = t^2-t = t(t-1) \\
    \text{Spezialfall: } f = \text{id}_V : \quad \mu_f = t - 1
\end{gather*}
$f$ ist immer diagonalisierbar, da $\mu_f$ immer in Linearfaktoren zerfällt.

\ttask
\tttask
\begin{theorem}
    $V$ sei $K$-VR, $n = \dim V < \infty$ , $f,\,g \in \text{End}_K(V)$. So gilt für $\mu_{f \circ g},\, \mu_{g \circ f} \in K[t]$
    \[ \mu_{f \circ g} \mid t \cdot \mu_{g \circ f} \]
\end{theorem}
\vspace{-1em}
\begin{proof}
Sei $\displaystyle \mu_{g \circ f} = \sum_{k=0}^n c_kt^k \qquad c_1,\,\dots,\,c_n \in K$
\begin{align*}
    t\mu_{g \circ f}(f \circ g) &= \sum_{k=0}^n c_k\overbrace{(f \circ g)\circ\hdots\circ(f \circ g)}^{k+1}
    = \sum_{k=0}^n c_kf \circ ~ \overbrace{(g \circ f)\circ\hdots\circ(g \circ f)}^k ~ \circ g \\
                                &= f \circ \qty(\sum_{k=0}^n c_k(g \circ f)^k)\circ g = f \circ \mu_{g \circ f}(g \circ f) \circ g = 0 \\
    \xRightarrow{t\mu_{g \circ f} \in G(g \circ f)} \qquad &\mu_{f \circ g} \mid t \cdot \mu_{g \circ f}  
\end{align*}
\end{proof}

\tttask
\begin{gather*}
    A = \mqty(0&0\\0&1) \quad B = \mqty(0&1\\0&0) \qquad f \coloneqq \widetilde A ~,~g \coloneqq \widetilde B \\
    AB = 0 ~,~ \mu_{f \circ g} = t \qquad BA = B ~,~ \mu_{g \circ f} = t^2
\end{gather*}
\end{document}
