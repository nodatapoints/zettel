\documentclass[a4paper, 12pt]{scrartcl}

\usepackage[utf8]{inputenc}
\usepackage[T1]{fontenc}
\usepackage[ngerman]{babel}

\usepackage{amssymb}
\usepackage{amsmath}
\usepackage{physics}
\usepackage{framed}
\usepackage{float}
\usepackage{mathtools}
\usepackage{marvosym}
\usepackage{bbm}

\usepackage{tikz}
\usepackage{chngcntr}

\usepackage{amsthm}
\usepackage{thmtools}

\usepackage[left=2cm, right=2cm, top=2cm]{geometry}

\allowdisplaybreaks

\setlength{\parindent}{0pt}

\setkomafont{paragraph}{\normalfont\itshape}


\declaretheoremstyle[%
  spaceabove=0,%
  spacebelow=6pt,%
  headfont=\normalfont\itshape,%
  postheadspace=1em,%
  headpunct={}
]{mystyle}

\declaretheorem[name={Behauptung}, style=mystyle, unnumbered]{theorem}
\declaretheorem[name={Lemma}, style=mystyle]{lemma}
\declaretheorem[name={Voraussetzung}, style=mystyle, unnumbered]{precondition}
\declaretheorem[name={Bemerkung}, style=mystyle, unnumbered]{note}
\let\proof\oldproof
\declaretheorem[name={Beweis}, style=mystyle, qed=\qedsymbol, unnumbered]{proof}

\newcounter{taski}
\newcounter{taskii}[taski]
\newcounter{taskiii}[taskii]

\newcommand{\task}{\stepcounter{taski}\textbf{Aufgabe \arabic{taski}}~}
\newcommand{\ttask}{\stepcounter{taskii}\textbf{(\alph{taskii})}~}
\newcommand{\tttask}{\stepcounter{taskiii}\quad(\roman{taskiii})~}

\DeclareMathOperator*{\Lin}{Lin}
\DeclareMathOperator*{\id}{id}

\setcounter{taski}{40}
\begin{document}
\begin{center}
    \textbf{8. Abgabeblatt}\\[2em]
	\def\arraystretch{2}
    \begin{tabular}{|l|l|l|l||p{18mm}|}
        \hline
        Aufgabe 41 & Aufgabe 42 & Aufgabe 43 & Aufgabe 44 & Summe:~ \\
        \hline &&&&\\
         \hline  
    \end{tabular}
\end{center}
\begingroup
\def\arraystretch{1.5}
\begin{tabular}{p{.5\textwidth}p{.5\textwidth}}
	\hline
    Übungsgruppe: Mo 14:15 ~~ SR B& Tutor(in): Sebastian Groß\\
    Namen: Kamal Abdellatif &\\
    \hline
\end{tabular}
\endgroup\\

\task
\ttask
\begin{gather*}
    \mathcal{C} = (v_1,\, v_2,\, v_3) \qquad \mathcal{B} = (w_1,\, w_2,\, w_3) \\
    w_1 = \frac{v_1}{\norm{v_1}} = \frac{v_1}{1} = (1,\, 0,\, 0)^t \\
    \widetilde{w_2} = v_2 - \expval{v_2,w_1}w_1 = (0,\, i,\, 0)^t \qc \norm{\widetilde{w_2}} = \sqrt{2} \qc 
    w_2 = \frac{\sqrt 2}{2}(0,\, i,\, 0)^t \\
    \widetilde{w_3} = v_3 - \expval{v_3,w_1}w_1 - \expval{v_3,w_2}w_2 = \qty(-i,\, \frac{1}{2},\, 1)^t
    \qc \norm{\widetilde{w_3}} = \frac 1{\sqrt 2} \qc 
    w_3 = \sqrt 2\qty(-i,\, \frac{1}{2},\, 1)^t
\end{gather*}
\ttask
\begin{gather*}
    M^\mathcal{B}_\mathcal{B}(\tilde{B}) = T^{E_3}_{\mathcal{B}}B~T^{\mathcal{B}}_{E_3}
    = \sqrt 2\mqty(\sqrt{2} & 0 & -2i \\ 0 & i & 1 \\ 0 & 0 & 2)^{-1}
    \mqty( i & 2 & i-2 \\ 0 & i-1 & 0 \\ 0 & 2i & -i-1)
    \frac{\sqrt 2}{2}\mqty(\sqrt{2} & 0 & -2i \\ 0 & i & 1 \\ 0 & 0 & 2) \\
    = \mqty(i & 0 & 0 \\ 0 & -1 & 1 \\ 0 & -1 & -1) \\
    \xRightarrow{\text{Hinsehen}} \qq{nicht selbstadjungiert} \qquad
    \xRightarrow{\text{Nachrechnen}} \qq{nicht unitär} \\
    M^\mathcal{B}_\mathcal{B}(\tilde{B} \circ \tilde{B}^{ad})
    = M^\mathcal{B}_\mathcal{B}(\tilde{B})\underbrace{\overline{M^\mathcal{B}_\mathcal{B}(\tilde{B})^t}}_{\substack{M^\mathcal{B}_\mathcal{B}(\tilde{B}^{ad}) \\ \text{da $\mathcal{B}$ ONB}}}
    = \mqty(1 & 0 & 0 \\ 0 & 0 & -2 \\ 0 & 2 & 0)
    = \overline{M^\mathcal{B}_\mathcal{B}(\tilde{B})^t}M^\mathcal{B}_\mathcal{B}(\tilde{B})
    = M^\mathcal{B}_\mathcal{B}(\tilde{B}^{ad} \circ \tilde{B})\\
    \implies \tilde{B} \circ \tilde{B}^{ad} = \tilde{B}^{ad} \circ \tilde{B} \\ 
    M^{E_3}_{E_3}(\tilde{B} \circ \tilde{B}^{ad}) = B\overline{B^t}
    = \mqty(1 & 4 & 2-5i\\ 0 & 2 & 0\\ 0 & 0 & 2)
    \neq \mqty(1 & -4 & 2+5i\\ 0 & 2 & 0\\ 0 & 0 & 2)
    = \overline{B^t}B = M^{E_3}_{E_3}(\tilde{B}^{ad} \circ \tilde{B}) \\
    \implies \tilde{B} \circ \tilde{B}^{ad} \neq \tilde{B}^{ad} \circ \tilde{B}
\end{gather*}
\task
\ttask Da $f$ normal gibt es eine ONB $\mathcal{B}$ aus EV von $f$, sodass
\[
    M^\mathcal{B}_\mathcal{B}(f)
    = \mathrm{Diag}(\lambda_1,\, \dots,\, \lambda_n) \eqqcolon M \qc f^3
    = f^4 \implies \lambda_i^3 = \lambda_i^4 \quad (1 \leq i \leq n)
\]
Für ein beliebiges $\lambda_i ~,~1 \leq i \leq n$ folgt
\begin{gather*}
    \lambda_i^3 = \lambda_i^4 \qc \lambda_i^3(\lambda_i - 1) = 0 \implies \lambda_i \in \qty{0,\, 1} \quad (\ast) \\
    (\ast) \implies \lambda_i^2 = \lambda_i \implies f^2 = f \qquad
    (\ast) \implies \overline{\lambda_i} = \lambda_i \implies \overline{M^t} = M \xRightarrow{\mathcal{B}~\text{ONB}} f = f^{ad}
\end{gather*}

\newpage
\ttask $(V, \expval{\cdot,\cdot})$ unitärer Vektorraum, $\dim_ \mathbb{C} = n < \infty$, $f \in \mathrm{End}_\mathbb{C}(V)$
\begin{theorem}
    \[ f~\text{normal} ~\Longleftrightarrow~ \exists p \in \mathbb{C}[t]_{\leq n-1} : f^{ad} = p(f) \]
\end{theorem}
\begin{proof}

\glqq$\Leftarrow$\grqq~
\[
    f \circ f^{ad}
    = f \circ \sum_{k=0}^{n-1}a_kf^k
    = \sum_{k=0}^{n-1}a_kf \circ f^k
    = \sum_{k=0}^{n-1}a_kf^k \circ f
    = (\sum_{k=0}^{n-1}a_kf^k) \circ f
    = f^{ad} \circ f
\]
\glqq$\Rightarrow$\grqq~ Sei $f$ normal. So gibt es eine ONB $\mathcal{B}$ von $V$ aus EV von $f$. 
\[ M \coloneqq M_\mathcal{B}^\mathcal{B}(f) = \mathrm{Diag}(\lambda_1,\,\dots,\,\lambda_n) \]
Es soll nun ein $p \in \mathbb{C}[t]_{\leq n-1}$ geben sodass $f^{ad} = f$.
\begin{gather*}
    M_\mathcal{B}^\mathcal{B}(f^{ad}) \overset{\text{ONB}}= \overline{M^t} = p(M) \\
    \implies \overline{\lambda_i} = p(\lambda_i) = \sum_{k=0}^{n-1}a_k \lambda_i^k \qquad (a_0,\,\dots,\,a_{n-1} \in \mathbb{C} ~,~ 1 \leq i \leq n) \\
    \Lambda \coloneqq \mqty(\lambda_1^0 & \cdots & \lambda_1^{n-1} \\ 
          \vdots & \ddots & \vdots \\ 
          \lambda_n^0 & \cdots & \lambda_n^{n-1})
    \qquad
    \Lambda \mqty(a_0 \\ \vdots \\ a_{n-1}) = \mqty(\overline{\lambda_1} \\ \vdots \\ \overline{\lambda_n})
\end{gather*}
Mit gegebenen $\lambda_1,\,\dots,\lambda_n$ kann dieses inhomogene LGS nach den Koeffizienten $a_0,\,\dots,\,a_{n-1}$ gelöst werden. Damit erhält man das gewünschte $p$.
\end{proof}
\task
\ttask
Seien $x_1,\,x_2 \in I ~,~ y_1,\,y_2 \in J ~,~ r \in R$ beliebig.
\begin{gather*}
    (x_1 + y_1) + (x_2 + y_2) = \underbrace{(x_1 + x_2)}_{\in~I} + \underbrace{(y_1 + y_2)}_{\in~J} \in I + J  \\
    r(x_1 + y_1) = \underbrace{rx_1}_{\in~I} + \underbrace{ry_1}_{\in~J} \in I + J  \\
\end{gather*}
\ttask

\ttask Wenn $R$ HIR, so gibt es $p \in I ~,~ q \in J$ sodass $I = (p) ~,~ J = (q)$. 
\begin{gather*}

\end{gather*}
\end{document}
