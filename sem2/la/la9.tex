\documentclass[a4paper, 12pt]{scrartcl}

\usepackage[utf8]{inputenc}
\usepackage[T1]{fontenc}
\usepackage[ngerman]{babel}

\usepackage{amssymb}
\usepackage{amsmath}
\usepackage{physics}
\usepackage{framed}
\usepackage{float}
\usepackage{mathtools}
\usepackage{marvosym}
\usepackage{bbm}

\usepackage{tikz}
\usepackage{chngcntr}

\usepackage{amsthm}
\usepackage{thmtools}

\usepackage[left=2cm, right=2cm, top=2cm]{geometry}

\allowdisplaybreaks

\setlength{\parindent}{0pt}

\setkomafont{paragraph}{\normalfont\itshape}


\declaretheoremstyle[%
  spaceabove=0,%
  spacebelow=6pt,%
  headfont=\normalfont\itshape,%
  postheadspace=1em,%
  headpunct={}
]{mystyle}

\declaretheorem[name={Behauptung}, style=mystyle, unnumbered]{theorem}
\declaretheorem[name={Lemma}, style=mystyle]{lemma}
\declaretheorem[name={Voraussetzung}, style=mystyle, unnumbered]{precondition}
\declaretheorem[name={Bemerkung}, style=mystyle, unnumbered]{note}
\let\proof\oldproof
\declaretheorem[name={Beweis}, style=mystyle, qed=\qedsymbol, unnumbered]{proof}

\newcounter{taski}
\newcounter{taskii}[taski]
\newcounter{taskiii}[taskii]

\newcommand{\task}{\stepcounter{taski}\textbf{Aufgabe \arabic{taski}}~}
\newcommand{\ttask}{\stepcounter{taskii}\textbf{(\alph{taskii})}~}
\newcommand{\tttask}{\stepcounter{taskiii}\quad(\roman{taskiii})~}

\DeclareMathOperator*{\Lin}{Lin}
\DeclareMathOperator*{\id}{id}

\setcounter{taski}{32}
\begin{document}
\begin{center}
    \textbf{8. Abgabeblatt}\\[2em]
	\def\arraystretch{2}
    \begin{tabular}{|l|l|l|l||p{18mm}|}
        \hline
        Aufgabe 33 & Aufgabe 34 & Aufgabe 35 & Aufgabe 36 & Summe:~ \\
        \hline &&&&\\
         \hline  
    \end{tabular}
\end{center}
\begingroup
\def\arraystretch{1.5}
\begin{tabular}{p{.5\textwidth}p{.5\textwidth}}
	\hline
    Übungsgruppe: Mo 14:15 ~~ SR B& Tutor(in): Sebastian Groß\\
    Namen: Kamal Abdellatif &\\
    \hline
\end{tabular}
\endgroup\\

\task
\ttask
\begin{gather*}
    M^{E_3}_{E_1}(g^*(\varphi))
    = M^{E_3}_{E_1}(\varphi \circ g)
    = M^E_{E_1}(\varphi)\underbrace{M^{E_3}_E(g)}_{\mathbbm{1}} = \mqty(3 & 3 & 5) \\
    g^*(\varphi) = \mqty(3 \\ 3 \\ 5)^*
\end{gather*}
\ttask $v_1,\, v_2$ sind linear unabhängig und bilden daher eine Basis für $U$. Diese Basis wird ergänzt durch einen Vektor $v_3$
\[ v_3 = 2 + t + 2t^2 \]
welcher linear unabhängig zu $v_1,\, v_2$ ist. Damit bildet $(v_1,\, v_2,\, v_3)$ eine Basis von $V$. Nach Satz 22.10 ist demnach $(v_3^*)$ eine Basis von $U^0$. Dabei ist $v_3^*$
\[ v_3^* = \Phi_{E^*}\qty((2,\, 1,\, 2)^t) \]
\ttask

\end{document}
