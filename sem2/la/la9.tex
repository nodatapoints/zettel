\documentclass[a4paper, 12pt]{scrartcl}

\usepackage[utf8]{inputenc}
\usepackage[T1]{fontenc}
\usepackage[ngerman]{babel}

\usepackage{amssymb}
\usepackage{amsmath}
\usepackage{physics}
\usepackage{framed}
\usepackage{float}
\usepackage{mathtools}
\usepackage{marvosym}
\usepackage{bbm}

\usepackage{tikz}
\usepackage{chngcntr}

\usepackage{amsthm}
\usepackage{thmtools}

\usepackage[left=2cm, right=2cm, top=2cm]{geometry}

\allowdisplaybreaks

\setlength{\parindent}{0pt}

\setkomafont{paragraph}{\normalfont\itshape}


\declaretheoremstyle[%
  spaceabove=0,%
  spacebelow=6pt,%
  headfont=\normalfont\itshape,%
  postheadspace=1em,%
  headpunct={}
]{mystyle}

\declaretheorem[name={Behauptung}, style=mystyle, unnumbered]{theorem}
\declaretheorem[name={Lemma}, style=mystyle]{lemma}
\declaretheorem[name={Voraussetzung}, style=mystyle, unnumbered]{precondition}
\declaretheorem[name={Bemerkung}, style=mystyle, unnumbered]{note}
\let\proof\oldproof
\declaretheorem[name={Beweis}, style=mystyle, qed=\qedsymbol, unnumbered]{proof}

\newcounter{taski}
\newcounter{taskii}[taski]
\newcounter{taskiii}[taskii]

\newcommand{\task}{\stepcounter{taski}\textbf{Aufgabe \arabic{taski}}~}
\newcommand{\ttask}{\stepcounter{taskii}\textbf{(\alph{taskii})}~}
\newcommand{\tttask}{\stepcounter{taskiii}\quad(\roman{taskiii})~}

\DeclareMathOperator*{\Lin}{Lin}
\DeclareMathOperator*{\id}{id}
\newcommand{\same}{\ensuremath~\Longleftrightarrow~}

\setcounter{taski}{32}
\begin{document}
\begin{center}
    \textbf{8. Abgabeblatt}\\[2em]
	\def\arraystretch{2}
    \begin{tabular}{|l|l|l|l||p{18mm}|}
        \hline
        Aufgabe 33 & Aufgabe 34 & Aufgabe 35 & Aufgabe 36 & Summe:~ \\
        \hline &&&&\\
         \hline  
    \end{tabular}
\end{center}
\begingroup
\def\arraystretch{1.5}
\begin{tabular}{p{.5\textwidth}p{.5\textwidth}}
	\hline
    Übungsgruppe: Mo 14:15 ~~ SR B& Tutor(in): Sebastian Groß\\
    Namen: Kamal Abdellatif &\\
    \hline
\end{tabular}
\endgroup\\

\task
\ttask
\begin{gather*}
    M^{E_3}_{E_1}(g^*(\varphi))
    = M^{E_3}_{E_1}(\varphi \circ g)
    = M^E_{E_1}(\varphi)\underbrace{M^{E_3}_E(g)}_{\mathbbm{1}} = \mqty(3 & 3 & 5) \\
    int bla
    g^*(\varphi) = \mqty(3 \\ 3 \\ 5)^*
\end{gather*}
\ttask $v_1,\, v_2$ sind linear unabhängig und bilden daher eine Basis für $U$. Diese Basis wird ergänzt durch einen Vektor $v_3$
\[ v_3 = 2 + t + 2t^2 \]
welcher linear unabhängig zu $v_1,\, v_2$ ist. Damit bildet $(v_1,\, v_2,\, v_3)$ eine Basis von $V$. Nach Satz 22.10 ist demnach $(v_3^*)$ eine Basis von $U^0$. Dabei ist $v_3^*$
\[ v_3^* = \Phi_{E^*}\qty((2,\, 1,\, 2)^t) \]
\ttask Aus Berechnung
\begin{gather*}
    M^E_{E_3}(f) = \frac{1}{6} \mqty(6&3&2 \\ 6&-3&2 \\ 6&12&16)
    \intertext{$(v_1,\,v_2)$ ist Basis von $U$, demnach $(w_1,\,w_2) = (f_(v_1),\,f(v_2))$ Basis von $f(U)$}
    f(v_1) = M^E_{E_3}(f)\mqty(1\\-2\\0) = \mqty(0\\2\\-3) \qquad
    f(v_2) = M^E_{E_3}(f)\mqty(-3\\0\\3) = \mqty(-2\\-2\\5)
    \intertext{Die Basis $(w_1,\,w_2)$ wird mit $w_3$ zu einer Basis von $V$ ergänzt.}
    w_3 = \mqty(2\\3\\2)\qc X^0 = \Lin( \qty{w_3^*} )
\end{gather*}
Ich sehe hier keinen Unterschied zwischen (i) und (ii).
\newpage
\ttask
\begin{align*}
    (f^*)^{-1}(U^0) &= X^0 \\
    U^0 &= f^*(X^0) = \Lin(f^*(w_3^*)) = \Lin\qty(M^{E_3^*}_{E^*}(f*)\Phi^{-1}_{E_3^*}(w_3^*))\\
    &= (M^{E}_{E_3}(f)^tw_3) = \Lin\qty(\Phi_{E_3^*}\frac{7}{2}\mqty(2\\1\\2)) = \Lin(v_3^*)
\end{align*}
\task \ttask $V$ ist $K$-VR, $U,\,U_1,\,U_2 \subseteq V$ UVR.
\begin{theorem}
    \[ U_1^0 + U_2^0 \subset (U_1 \cap U_2)^0 \]
    Bei $\dim_K V < \infty$ gilt Gleichheit.
\end{theorem}
\begin{proof}
\glqq $\subset$\grqq \quad
Sei $\varphi \in U_1^0 + U_2^0$ mit $\varphi = \varphi_1 + \varphi_2 ~,~ \varphi_1 \in U_1^0 ~,~ \varphi_2 \in U_2^0$. Es folgt
\begin{gather*}
    \forall u \in U_1 \cap U_2 : \varphi(u) = \varphi_1(u) + \varphi_2(u) = 0 \qq{da} u \in U_1 ~,~ u \in U_2 \\
    \implies \varphi \in (U_1 \cap U_2)^0
\end{gather*}

\glqq $\supset$\grqq \quad $\dim_K V = n < \infty$. Sei $\mathcal{A} = (u_1,\,\dots,\,u_r)$ eine Basis von $U_1 \cap U_2$. Diese Basis wird jeweils auf eine Basis für $U_1$ und $U_2$ erweitert.
\[ \mathcal{B} = (u_1,\,\dots,\,u_r,\,v_1,\dots\,v_s) \qq{Basis von} U_1 \qquad \mathcal{C} = (u_1,\,\dots,\,u_r,\,w_1,\,\dots\,w_t) \qq{Basis von} U_2  \]
Diese Basen $\mathcal{B},\,\mathcal{C}$ werden vereint und erweitert zu einer Basis $\mathcal{V}$ von $V$ mit
\[ \mathcal{V} = (u_1,\,\dots,\,u_r,\,v_1,\,\dots,\,v_s,\,w_1,\dots\,w_t,\,x_1,\,\dots,\,x_\ell) \]
Nun sind die dualen Basen darstellbar als
\begin{gather*}
    \mathcal{B}^* = (w_1^*,\,\dots,\,w_t^*,\,x^*_1,\dots\,x_\ell^*) \qq{B. von} U_1^0 \qquad
    \mathcal{C}^* = (v_1^*,\,\dots,\,v_s^*,\,x^*_1,\dots\,x_\ell^*) \qq{B. von} U_2^0 \\
    \mathcal{A}^* = (v_1^*,\,\dots,\,v_s^*,\,w_1^*,\,\dots,\,w_t^*,\,x^*_1,\dots\,x_\ell^*) \qq{B. von} (U_1 \cap U_2)^0
\end{gather*}
Da sich $\mathcal{A}^*$ aus $\mathcal{B}^*$ und $\mathcal{C}^*$ zusammensetzt gilt demnach
\[ U_1^0 + U_2^0 = (U_1 \cap U_2)^0 \]
\end{proof}
\newpage
\ttask
\begin{theorem}
    \[ i(U) \subset \qty(U^0)^0 \]
    Gleichheit bei $\dim_K V < \infty$
\end{theorem}
\begin{proof}
\glqq$\subset$\grqq \quad Sei $u \in U$
\begin{gather*}
    i(u)(\varphi) = \varphi(u) = 0 ~\forall \varphi \in U^0 \\
    \implies i(u) \in \qty(U^0)^0
\end{gather*}
\glqq$\supset$\grqq \quad $\dim_K V = n < \infty$
\begin{align*}
    \dim_K \qty(U^0)^0 &\overset{22.10}= n - \dim_K (U^0) = n - \qty(n - \dim_K U) = \dim_K U \overset{i~\text{Iso.}}= \dim_K U^{**} \\
    \xRightarrow{\text{Aus Teil~} \subset} \qty(U^0)^0 &= i(U)
\end{align*}
\end{proof}
\ttask
\begin{theorem}
    $V,W$ sind $K$-VR, $f \in \mathrm{Hom}_K(V,\,W)$, $U \subset V$ UVR
    \[ (f^*)^{-1}(U^0) = \qty(f(U))^0 \]
\end{theorem}
\begin{proof} Sei $x \in W^*$
\begin{gather*}
    x \in \qty(f(U))^0 \same x(f(u)) = 0 ~\forall u \in U \same f^*(x)(u) = 0 ~\forall u \in U \\
    \same f^*(x) \in U^0 \same x \in (f^*)^{-1}(U^0)
\end{gather*}
\end{proof}
\end{document}
