\documentclass[a4paper, 12pt]{scrartcl}

\usepackage[utf8]{inputenc}
\usepackage[T1]{fontenc}
\usepackage[ngerman]{babel}

\usepackage{amssymb}
\usepackage{amsmath}
\usepackage{physics}
\usepackage{framed}
\usepackage{float}
\usepackage{mathtools}
\usepackage{marvosym}

\usepackage{tikz}
\usepackage{chngcntr}

\usepackage{amsthm}
\usepackage{thmtools}

\usepackage[left=2cm, right=2cm, top=2cm]{geometry}

\allowdisplaybreaks

\setlength{\parindent}{0pt}

\setkomafont{paragraph}{\normalfont\itshape}


\declaretheoremstyle[%
  spaceabove=0,%
  spacebelow=6pt,%
  headfont=\normalfont\itshape,%
  postheadspace=1em,%
  headpunct={}
]{mystyle}

\declaretheorem[name={Behauptung}, style=mystyle, unnumbered]{theorem}
\declaretheorem[name={Lemma}, style=mystyle]{lemma}
\declaretheorem[name={Voraussetzung}, style=mystyle, unnumbered]{precondition}
\let\proof\oldproof
\declaretheorem[name={Beweis}, style=mystyle, qed=\qedsymbol, unnumbered]{proof}

\newcounter{taski}
\newcounter{taskii}[taski]
\newcounter{taskiii}[taskii]

\newcommand{\task}{\stepcounter{taski}\textbf{Aufgabe \arabic{taski}}}
\newcommand{\ttask}{\stepcounter{taskii}\textbf{(\alph{taskii})}}
\newcommand{\tttask}{\stepcounter{taskiii}\quad(\roman{taskiii})}

\begin{document}
\begin{center}
    \textbf{1. Abgabeblatt}\\[2em]
	\def\arraystretch{2}
    \begin{tabular}{|l|l|l||p{18mm}|}
        \hline
         Aufgabe 1 & Aufgabe 2 & Aufgabe 3 & Summe:~ \\
         \hline &&&\\
         \hline  
    \end{tabular}
\end{center}
\begingroup
\def\arraystretch{1.5}
\begin{tabular}{p{.5\textwidth}p{.5\textwidth}}
	\hline
    Übungsgruppe: Mo 14:15 ~~ SR B& Tutor(in): Sebastian Groß\\
    Namen: Ellen, Kamal &\\
    \hline
\end{tabular}
\endgroup
\vspace{2em}
\setcounter{taski}{1}

\task

$\lambda_1,\ \lambda_2$ \emph{aus Hinweis}
\begin{align*}
	A &= \mqty(1 & 1 \\ 1 & 0) \\
	\chi_A(\lambda) &= \lambda^2 - \lambda - 1 = (\lambda - \lambda_1)(\lambda - \lambda_2) \\
	\Rightarrow \text{Eigenwerte}\ &\qty{\lambda_1,\ \lambda_2} \\\\
	\mqty( 1 - \lambda & 1 \\ 1 & -\lambda)v &= 0 \qquad \curvearrowright \quad \text{Hom. LGS} \\
	\mqty( 1 & \frac 1 {1-\lambda} \\ 1 & - \lambda) &\rightarrow \mqty( 1 & -\lambda \\ 1 & - \lambda)
	\rightarrow \mqty(1 & -\lambda \\ 0 & 0) \qquad (\lambda = \lambda_1,\ \lambda_2) \\
	\Rightarrow \text{Eigenvektoren} &\quad (\lambda_1,\ 1)^\top ~,~ (\lambda_2,\ 1)^\top \quad,\quad \text{sei } \mathcal E = (v_1,\ v_2) \text{ Basis von $V$}\\\\
	M_{\mathcal E}^{\mathcal E}(\widetilde A\,) &= \textrm{diag}(\lambda_1,\ \lambda_2) \\
	M_{\mathcal E}^{\mathcal E}(\widetilde A^n) &= \textrm{diag}(\lambda_1^n,\ \lambda_2^n) \qquad n \in \mathbb{N}_0 \\
	A^n &= T^{\mathcal E}_{e_1,e_2} M^{\mathcal E}_{\mathcal E}(\widetilde A^n) T_{\mathcal E}^{e_1,e_2} = \mqty(\lambda_1 & \lambda_2 \\ 1 & 1) \mqty(\diagonalmatrix{\lambda_1^n, \lambda_2^n}) \mqty(\lambda_1 & \lambda_2 \\ 1 & 1)^{-1} \\
	&= \mqty(\lambda_1 & \lambda_2 \\ 1 & 1) \mqty(\diagonalmatrix{\lambda_1^n, \lambda_2^n}) \mqty(\diagonalmatrix{\frac 1 {\lambda_1^2+1}, \frac 1 {\lambda_2^2+1}})\mqty(\lambda_1 & 1 \\ \lambda_2 & 1) \\
	\mqty(a_n \\ b_n) = A^n \mqty(1 \\ 1) &= \mqty(\lambda_1 & \lambda_2 \\ 1 & 1) \mqty(\diagonalmatrix{\frac{\lambda_1^n}{\lambda_1^2+1}, \frac{\lambda_2^n}{\lambda_2^2+1}}) \mqty(\lambda_1^2 \\ \lambda_2^2) = \mqty(\frac{\lambda_1^{n+3}}{\lambda_1^2+1} + \frac{\lambda_2^{n+3}}{\lambda_2^2+1} \\ \cdots) \\
	a_n &= \frac{\lambda_1^{n+3}}{\lambda_1^2+1} + \frac{\lambda_2^{n+3}}{\lambda_2^2+1} = \frac{\lambda_1^{n+3}}{\lambda_1^2-\lambda_1\lambda_2} + \frac{\lambda_2^{n+3}}{\lambda_2^2-\lambda_1\lambda_2} = \frac{\lambda_1^{n+2}}{\lambda_1-\lambda_2}+ \frac{\lambda_2^{n+2}}{\lambda_2-\lambda_1} \\
	&= \frac{\lambda_2^{n+2}-\lambda_1^{n+2}}{\lambda_2-\lambda_1}
\end{align*}
\newpage

\task

\ttask\ 
Es sei ein beliebiges $\varphi \in \mathrm{End}_K(V)$ und $v \in V \backslash \{0\} ~,~ \lambda \in K$ ein Eigenpaar. So folgt
\begin{align*}
	\lambda v = \varphi(v) &= \varphi^3(v) = \lambda^3v \\
	\xRightarrow{v \neq 0} \qquad \lambda(\lambda - 1)(\lambda+1) &= 0 \\
	\Rightarrow \quad \lambda &\in \{-1,\ 0,\ 1\}
\end{align*}

\ttask\ 
Sei $p = \sum_{k=0}^n c_kt^k ~,~ c_0,\ \dots,\ c_n \in K ~,~ n = \mathrm{Grad}\  p$
\[ p(\varphi)(v) = \sum_{k=0}^n c_k\varphi^k(v) = v\sum_{k=0}^n c_k\lambda^k = p(\lambda)v \]

\ttask\ 
Es sei ein festes $v \in V \backslash \qty{0} ~,~ \lambda \in K$ mit $\varphi(v) = \lambda v$. Für jedes $w \in V \backslash \qty{0}$ gibt es ein $\mu \in K$ sodass $\varphi(w) = \mu w$. \\
Fall 1: $v,\ w$ lin. abhängig \\
\begin{align*}
	\Rightarrow \exists c \in K : w &= cv \\
	\mu w = \varphi(w) &= \varphi(cv) = c \lambda v = \lambda w \qquad (w \neq 0) \\
	\mu &=\lambda
\end{align*}
Fall 2: $v,\ w$ lin. unabhängig \\
\begin{gather*}
	\Rightarrow (v,\ w) \text{ Basis von Lin} (\qty{v,\ w}) \\
	v+w \in V \quad \Rightarrow \exists \quad \gamma \in K : \varphi(v+w) = \gamma (v+w) \\
	\lambda v + \mu w = \varphi(v+w) = \gamma v + \gamma w \\
	\xRightarrow{\text{Basis}} \qquad \Phi_{v,w}^{-1}(\lambda v + \mu w) = \mqty(\lambda \\ \mu) = \mqty(\gamma \\ \gamma) = \Phi_{v,w}^{-1}(\gamma(v+w))\quad \Rightarrow \quad \mu = \gamma = \lambda
\end{gather*}
Somit gilt
\begin{gather*}
	\forall w \in V: \varphi(w) = \lambda w = (\lambda \cdot \textrm{id}_V)(w)
\end{gather*}
\end{document}

