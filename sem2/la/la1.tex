\documentclass[a4paper, 12pt]{scrartcl}

\usepackage[utf8]{inputenc}
\usepackage[T1]{fontenc}
\usepackage[ngerman]{babel}

\usepackage{amssymb}
\usepackage{amsmath}
\usepackage{physics}
\usepackage{framed}
\usepackage{float}
\usepackage{mathtools}
\usepackage{marvosym}

\usepackage{tikz}
\usepackage{chngcntr}

\usepackage{amsthm}
\usepackage{thmtools}

\usepackage[left=2cm, right=2cm, top=2cm]{geometry}

\allowdisplaybreaks

\setlength{\parindent}{0pt}

\setkomafont{paragraph}{\normalfont\itshape}


\declaretheoremstyle[%
  spaceabove=0,%
  spacebelow=6pt,%
  headfont=\normalfont\itshape,%
  postheadspace=1em,%
  headpunct={}
]{mystyle}

\declaretheorem[name={Behauptung}, style=mystyle, unnumbered]{theorem}
\declaretheorem[name={Lemma}, style=mystyle]{lemma}
\declaretheorem[name={Voraussetzung}, style=mystyle, unnumbered]{precondition}
\let\proof\oldproof
\declaretheorem[name={Beweis}, style=mystyle, qed=\qedsymbol, unnumbered]{proof}

\newcounter{taski}
\newcounter{taskii}[taski]
\newcounter{taskiii}[taskii]

\newcommand{\task}{\stepcounter{taski}\textbf{Aufgabe \arabic{taski}}}
\newcommand{\ttask}{\stepcounter{taskii}\textbf{(\alph{taskii})}}
\newcommand{\tttask}{\stepcounter{taskiii}\quad(\roman{taskiii})}

\begin{document}
\begin{center}
    \textbf{1. Abgabeblatt}\\[2em]
	\def\arraystretch{2}
    \begin{tabular}{|l|l|l||p{18mm}|}
        \hline
         Aufgabe 1 & Aufgabe 2 & Aufgabe 3 & Summe:~ \\
         \hline &&&\\
         \hline  
    \end{tabular}
\end{center}

\hline
\bgroup
\def\arraystretch{1.5}
\begin{tabular}{p{.5\textwidth}p{.5\textwidth}}
    Übungsgruppe: & Tutor(in): \\
    Namen: &\\
\end{tabular}
\egroup
\hline
\vspace{2em}
\setcounter{\taski}{2}

\task

\ttask
\begin{align*}
	A &= \mqty(1 & 1 \\ 1 & 0) \\
	\chi_A(\lambda) &= \lambda^2 - \lambda - 1 = (\lambda - \lambda_1)(\lambda - \lambda_2) \\

\ttask
\[ \mqty(2 & 1 \\ 1 & 1) = A^2 &= A + 1 \]

\task

\ttask\ 
Es sei ein beliebiges $\varphi \in \mathrm{End}_K(V)$ und $v \in V \backslash \{0\} ~,~ \lambda \in K$ ein Eigenpaar. So folgt
\begin{align*}
	\lambda v = \varphi(v) &= \varphi^3(v) = \lambda^3v \\
	\xRightarrow{v \neq 0} \qquad \lambda(\lambda - 1)(\lambda+1) &= 0 \\
	\Rightarrow \quad \lambda &\in \{-1,\ 0,\ 1\}
\end{align*}

\ttask\ 
Sei $p = \sum_{k=0}^n c_kt^k ~,~ c_0,\ \dots,\ c_n \in K ~,~ n = \mathrm{Grad}\  p$
\[ p(\varphi)(v) &= \sum_{k=0}^n c_k\varphi^k(v) = v\sum_{k=0}^n c_k\lambda^k = p(\lambda)v \]

\ttask\ 
Da alle $v \in V$ Eigenvektoren sind, existiert ein $s: V \rightarrow K$ sodass 
\[ \varphi(v) = s(v)v \qquad \forall v \in V \]
\end{document}

