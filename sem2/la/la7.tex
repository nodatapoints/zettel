\documentclass[a4paper, 12pt]{scrartcl}

\usepackage[utf8]{inputenc}
\usepackage[T1]{fontenc}
\usepackage[ngerman]{babel}

\usepackage{amssymb}
\usepackage{amsmath}
\usepackage{physics}
\usepackage{framed}
\usepackage{float}
\usepackage{mathtools}
\usepackage{marvosym}
\usepackage{bbm}

\usepackage{tikz}
\usepackage{chngcntr}

\usepackage{amsthm}
\usepackage{thmtools}

\usepackage[left=2cm, right=2cm, top=2cm]{geometry}

\allowdisplaybreaks

\setlength{\parindent}{0pt}

\setkomafont{paragraph}{\normalfont\itshape}


\declaretheoremstyle[%
  spaceabove=0,%
  spacebelow=6pt,%
  headfont=\normalfont\itshape,%
  postheadspace=1em,%
  headpunct={}
]{mystyle}

\declaretheorem[name={Behauptung}, style=mystyle, unnumbered]{theorem}
\declaretheorem[name={Lemma}, style=mystyle]{lemma}
\declaretheorem[name={Voraussetzung}, style=mystyle, unnumbered]{precondition}
\let\proof\oldproof
\declaretheorem[name={Beweis}, style=mystyle, qed=\qedsymbol, unnumbered]{proof}

\newcounter{taski}
\newcounter{taskii}[taski]
\newcounter{taskiii}[taskii]

\newcommand{\task}{\stepcounter{taski}\textbf{Aufgabe \arabic{taski}}~}
\newcommand{\ttask}{\stepcounter{taskii}\textbf{(\alph{taskii})}~}
\newcommand{\tttask}{\stepcounter{taskiii}\quad(\roman{taskiii})~}

\DeclareMathOperator*{\Spur}{Spur}
\DeclareMathOperator*{\Lin}{Lin}

\setcounter{taski}{21}
\begin{document}
\begin{center}
    \textbf{7. Abgabeblatt}\\[2em]
	\def\arraystretch{2}
    \begin{tabular}{|l|l|l|l||p{18mm}|}
        \hline
        Aufgabe 25 & Aufgabe 26 & Aufgabe 27 & Aufgabe 28 & Summe:~ \\
        \hline &&&&\\
         \hline  
    \end{tabular}
\end{center}
\begingroup
\def\arraystretch{1.5}
\begin{tabular}{p{.5\textwidth}p{.5\textwidth}}
	\hline
    Übungsgruppe: Mo 14:15 ~~ SR B& Tutor(in): Sebastian Groß\\
    Namen: Ellen Bräutigam, Kamal Abdellatif &\\
    \hline
\end{tabular}
\endgroup\\

\task
\ttask
\begin{theorem}
    \[ (U+W)^\perp = U^\perp \cap W^\perp \]
\end{theorem}
\begin{proof}
''$\subseteq$"' : $\displaystyle v \in (U+W)^\perp \implies v \in U^\perp \:\wedge\: v \in W^\perp \qquad \forall v \in V$
Sei $v \in (U+W)^\perp$. Es gilt demnach
\[ 0 = \expval{v,\,u+w} \qquad \forall u \in U ~,~ w \in W. \]
Mit $u = 0$ bzw. $w = 0$ folgt
\[ 0 = \expval{v,\,u} \quad \forall u \in U \qquad\qquad 0 = \expval{v,\,w} \quad \forall w \in W, \]
womit nach Definition $v \in U^\perp$ und $v \in W^\perp$

''$\supseteq$"' : $ v \in U^\perp \:\wedge\: v \in W^\perp \implies \displaystyle v \in (U+W)^\perp\qquad \forall v \in V$

Sei $v \in V$.
\begin{gather*}
    0 = \expval{v,\,u} \quad \forall u \in U \qquad,\qquad 0 = \expval{v,\,w} \quad \forall w \in W\\
    \begin{aligned}
        \Rightarrow\quad 0 &= \expval{v,\,u} + \expval{v,\,w} = \expval{v,\,u+w} \qquad \forall u \in U ~,~ w \in W \\
        \Rightarrow\quad v &\in (U+W)^\perp
    \end{aligned}
\end{gather*}
\end{proof}
\ttask
\begin{align*}
    \qty(U^\perp + W^\perp)^\perp &= \qty(U^\perp)^\perp \cap \qty(W^\perp)^\perp \subseteq U \cap W \\
    U^\perp + W^\perp \subseteq \qty(\qty(U^\perp + W^\perp)^\perp)^\perp &\subseteq (U \cap W)^\perp \\
\end{align*}
\ttask
Sei $g \in U^\perp$, so gilt 
\begin{align*}
    \forall f \in U: 0 &= \expval{f,g} = \int_0^1f(x)g(x)\dd x = \int_0^cf(x)g(x)\dd x +\overbrace{\int_c^1f(x)g(x)\dd x}^{=~0 \text{ nach Def. } f} \\
    0 &= \int_0^cf(x)g(x)\dd x
\end{align*}
Da $f$ beliebig auf dem Intervall $[0,c)$ ist, muss $g\eval_{[0,c)} \equiv 0$. Wäre dies nicht der Fall, so könnte ein $f \in U$ konstruiert werden mit $g(x) \geq 0 \Leftrightarrow f(x) \geq 0 ~\forall x \in [0,c)$, sodass $f(x)g(x) \geq 0$. Unter der Annahme $g \not\equiv 0$ wäre durch Stetigkeit $\int_0^1f(x)g(x)\dd x > 0$, was zum Widerspruch führen würde.

Es ergibt sich
\[ U^\perp = \qty{g \in V \mid \forall x \in [0,c) : g(x) = 0} \]
Nach der selben Argumentation erhält man $\qty(U^\perp)^\perp = U$.

\ttask
Angenommen $1 \in U + U^\perp$, so gibt es $f \in U~,~g\in U^\perp$ sodass $f(x)+g(x) = 1 ~\forall x \in [0,1]$. Aus den Definitionen folgt:
\[ g(x) = 0 ~\forall x \in [0,c) \implies f(x) = 1 ~\forall x \in [0,c) \]
Da jedoch $f(c) = 0$, ist $f$ in der Stelle $c$ nicht stetig, womit $f \notin V$ \quad {\large\Lightning}\\
\ttask Nach obiger Lösung
\begin{gather*}
    1 \notin U^\perp + \qty(U^\perp)^{\perp} = U + U^\perp \qquad 1 \in V = \qty{0}^\perp = (U \cap U^\perp)^\perp \\
    \implies U + U^\perp \neq (U \cap U^\perp)^\perp
\end{gather*}
\setcounter{taski}{27}
\task
\ttask Es sei $u,\,v \in V~,~ \lambda \in K$ beliebig
\begin{align*}
    &~ \norm{f(u+v)-f(u)-f(v)}^2_W \\
    &= \expval{f(u+v)-f(u)-f(v)~,~f(u+v)-f(u)-f(v)}_W \\
    &= \norm{f(u+v)}^2_W - 2\expval{f(u+v),f(u)+f(v)}_W + \norm{f(u)+f(v)}^2_W \\
    &= \norm{f(u+v)}^2_W - 2\expval{f(u+v),f(u)}_W - 2\expval{f(u+v),f(v)}_W + \norm{f(u)}^2_W + 2\expval{f(u),f(v)}_W + \norm{f(v)}^2_W \\
    &= \norm{u+v}^2_V - 2\expval{u+v,u}_V - 2\expval{u+v,v}_V + \norm{u}^2_V + 2\expval{u,v}_V + \norm{v}^2_V \\
    &= 2\norm{u+v}^2_V - 2\expval{u+v,u+v}_V = 0 \\
    &\Rightarrow f(u+v)-f(u)-f(v) = 0\qc f(u+v) = f(u) + f(v) \tag{$\ast$}
\end{align*}
\begin{align*}
    \norm{f(\lambda v) - \lambda f(v)}^2_W &= \norm{f(\lambda v)}^2_W - 2 \expval{f(\lambda v),\lambda f(v)}_W + \norm{\lambda f(v)}^2_W \\
    &= \norm{f(\lambda v)}^2_W - 2\lambda\expval{f(\lambda v),f(v)}_W + \lambda^2\norm{f(v)}^2_W \\
    &= \norm{\lambda v}^2_V - 2\lambda\expval{\lambda v,v}_V + \lambda^2\norm{v}^2_V \\
    &= 2\lambda^2\norm{v}^2_V - 2\lambda^2\expval{v,v}_V = 0 \\
    & \Rightarrow f(\lambda v) - \lambda f(v) = 0 \qc f(\lambda v) = \lambda f(v)\tag{$\ast\ast$}
\end{align*}
Aus $(\ast)$ und ($\ast\ast$) folgt die $K$-Linearität von $f$.

\ttask Sei $u,\,v \in V$ beliebig
\[ \norm{f(u) - f(v)} = \norm{u+w - v-w} = \norm{u-v}.  \]
\ttask Sei $u,\,v \in V$ beliebig. Da $g$ eine Isometrie ist, ist $g$ linear
\[ \norm{f(u) - f(v)} = \norm{f(u-v)} = \norm{u-v}.  \]
\ttask
Aus \textbf{(e)} folgt die eindeutige Darstellung $f = T_w \circ g$ mit $g$ als Isometrie und $T_w$ als Translation.
\[ 0 = f(0) = T_w(g(0)) = T_w(0) = w\qc f = T_w\circ g = T_0\circ g = \mathrm{id}_V\circ g = g, \]
wonach nach Voraussetzung $f$ eine Isometrie ist.
\newpage
\ttask
\begin{theorem}
    Zu jeder Bewegung $f$ von $V$ existiert eine eindeutige Darstellung durch eine Isometrie $g \in \mathrm{End}_\mathbb{R}(V)$ und eine Translation $w \in V$ mit $f = T_w \circ g$. 
\end{theorem}
\begin{proof} Durch Widerspruch.
Es sei eine Bewegung $f = T_w \circ g = T_{w'} \circ g'$ mit $w \neq w' ~,~ g\neq g'$ nach den Voraussetzungen gegeben. Da $g(0) = g'(0) = 0$ gilt
\[ w' = T_{w'}(0) = T_{w'}(g'(0)) = f(0) = T_w(g(0)) = T_w(0) = w. \qq{\large\Lightning}\]
Da alle Translationen invertierbar sind mit $T_w^{-1} = T_{-w}$ folgt aus der Eindeutigkeit von $w$ folgt die Eindeutigkeit von $g$
\[ g' = T_{w'}^{-1} \circ f = T_{w'}^{-1}\circ T_w \circ g = T_w^{-1} \circ T_w \circ g = g \qq{\large\Lightning} \]
\end{proof}

\end{document}
