\documentclass[a4paper, 12pt]{scrartcl}

\usepackage[utf8]{inputenc}
\usepackage[T1]{fontenc}
\usepackage[ngerman]{babel}

\usepackage{amssymb}
\usepackage{amsmath}
\usepackage{physics}
\usepackage{framed}
\usepackage{float}
\usepackage{mathtools}
\usepackage{marvosym}

\usepackage{tikz}
\usepackage{chngcntr}

\usepackage{amsthm}
\usepackage{thmtools}

\usepackage[left=2cm, right=2cm, top=2cm]{geometry}

\allowdisplaybreaks

\setlength{\parindent}{0pt}

\setkomafont{paragraph}{\normalfont\itshape}


\declaretheoremstyle[%
  spaceabove=0,%
  spacebelow=6pt,%
  headfont=\normalfont\itshape,%
  postheadspace=1em,%
  headpunct={}
]{mystyle}

\declaretheorem[name={Behauptung}, style=mystyle, unnumbered]{theorem}
\declaretheorem[name={Lemma}, style=mystyle]{lemma}
\declaretheorem[name={Voraussetzung}, style=mystyle, unnumbered]{precondition}
\let\proof\oldproof
\declaretheorem[name={Beweis}, style=mystyle, qed=\qedsymbol, unnumbered]{proof}

\newcounter{taski}
\newcounter{taskii}[taski]
\newcounter{taskiii}[taskii]

\newcommand{\task}{\stepcounter{taski}\textbf{Aufgabe \arabic{taski}}~}
\newcommand{\ttask}{\stepcounter{taskii}\textbf{(\alph{taskii})}~}
\newcommand{\tttask}{\stepcounter{taskiii}\quad(\roman{taskiii})~}

\pagenumbering{gobble}

\begin{document}
\begin{center}
    \textbf{Analysis II\\2. Abgabeblatt}\\[2em]
	\def\arraystretch{2}
    \begin{tabular}{|l|l|l||p{18mm}|}
        \hline
         Aufgabe 1 & Aufgabe 2 & Aufgabe 3 & Summe:~ \\
         \hline &&&\\
         \hline  
    \end{tabular}
\end{center}
\begingroup
\def\arraystretch{1.5}
\begin{tabular}{p{.5\textwidth}p{.5\textwidth}}
	\hline
    Übungsgruppe: Do 14:00 ~~ SR B& Tutor: Levin Maler \\
    Namen: Ellen Bräutigam, Kamal Abdellatif &\\
    \hline
\end{tabular}
\endgroup
\\

\task
\begin{theorem}
	Sei $I = [a,b] ~,~ a<b$ ein Intervall mit $x_0 \in I$, $f: I \rightarrow \mathbb{R}$ stetig diff'bar auf $I \setminus \qty{x_0}$ mit 
	\[ \lim_{\substack{x \rightarrow x_0 \\ x \neq x_0}} f'(x) \coloneqq \alpha \]
	so ist $f$ in $x_0$ diff'bar mit $f'(x_0) = \alpha$.
\end{theorem}
\begin{proof}
\begin{align*}
	\lim_{\varepsilon \rightarrow 0} \varepsilon &= \lim_{\varepsilon \rightarrow 0} f(x_0+\varepsilon)-f(x_0) = 0 \\
	\lim_{\varepsilon \rightarrow 0} \frac{f(x_0+\varepsilon)-f(x_0)}{\varepsilon} &\overset{\textsc{L'Hospital}}{=}
    = \lim_{\varepsilon \rightarrow 0} \frac{\dv{\varepsilon}(f(x_0+\varepsilon)-f(x_0))}{\dv{\varepsilon}\varepsilon}
    \overset{\text{Kettenregel}}{=} \lim_{\varepsilon \rightarrow 0} \frac{f'(x_0+\varepsilon)\cdot 1 - 0}{1} \\
    &= \lim_{\varepsilon \rightarrow 0} f'(x_0+\varepsilon) = \lim_{\substack{x \rightarrow x_0 \\ x \neq x_0}} f'(x) = \alpha 
\end{align*}
Analog für den linksseitigen Differentialquotienten. Da beide Limes existieren und gleich $\alpha$ sind, ist $f$ in $x_0$ diff'bar und es gilt
\[ f'(x_0) = \alpha \]
\end{proof}
\addtocounter{taski}{1}
\task
\begin{align*}
	\dv{\Phi(v)}{v} &= \dv v \frac 1 2 \qty(A(v) + v\sqrt{1-v^2})
	= \frac 1 2 \qty(\dv v \int_0^v \dd u\, \frac 1 {\sqrt{1-u^2}} + \sqrt{1-v^2} - \frac{v^2}{\sqrt{1-v^2}}) \\
	\xRightarrow{\text{HS}} \qquad &= \frac 1 2 \qty(\frac{1}{\sqrt{1-0^2}} + \sqrt{1-v^2} - \frac{v^2}{\sqrt{1-v^2}} ) = \frac 1 2 \qty(\frac{1-v^2}{\sqrt{1-v^2}} + \sqrt{1-v^2} ) = \sqrt{1-v^2} \tag{$v \in [0,1)$} \\
	\intertext{Sei $\varepsilon > 0$}
	4 \int_0^{1-\varepsilon} \dd v\,\sqrt{1-v^2} &= 2 (A(v) + v\sqrt{1-v^2}) \eval_0^{1-\varepsilon} = 2(A(1)-A(0)) = 2\int_0^{1-\varepsilon} \dd u\,\frac{1}{\sqrt{1-u^2}} \\
	4 \int_0^{1\phantom{-\varepsilon}} \dd v\,\sqrt{1-v^2} &= 4 \cdot \lim_{\varepsilon \rightarrow 0}\int_0^{1-\varepsilon} \dd v\,\sqrt{1-v^2}
	= 2 \cdot \lim_{y \rightarrow 1^-} \int_0^y \dd u\,\frac{1}{\sqrt{1-u^2}} = \pi
\end{align*}
\end{document}
