\documentclass[a4paper, 12pt]{scrartcl}

\usepackage[utf8]{inputenc}
\usepackage[T1]{fontenc}
\usepackage[ngerman]{babel}

\usepackage{amssymb}
\usepackage{amsmath}
\usepackage{physics}
\usepackage{framed}
\usepackage{float}
\usepackage{mathtools}
\usepackage{marvosym}
\usepackage[shortlabels]{enumitem}

\usepackage{tikz}
\usepackage{chngcntr}

\usepackage{amsthm}
\usepackage{thmtools}

\usepackage[left=2cm, right=2cm, top=2cm]{geometry}

\allowdisplaybreaks

\setlength{\parindent}{0pt}

\setkomafont{paragraph}{\normalfont\itshape}


\declaretheoremstyle[%
  spaceabove=0,%
  spacebelow=6pt,%
  headfont=\normalfont\itshape,%
  postheadspace=1em,%
  headpunct={}
]{mystyle}

\declaretheorem[name={Behauptung}, style=mystyle, unnumbered]{theorem}
\declaretheorem[name={Lemma}, style=mystyle]{lemma}
\declaretheorem[name={Bemerkung}, style=mystyle, unnumbered]{note}
\declaretheorem[name={Voraussetzung}, style=mystyle, unnumbered]{precondition}
\let\proof\oldproof
\declaretheorem[name={Beweis}, style=mystyle, qed=\qedsymbol, unnumbered]{proof}

\newcounter{taski}
\newcounter{taskii}[taski]
\newcounter{taskiii}[taskii]

\newcommand{\task}{\stepcounter{taski}\textbf{Aufgabe \arabic{taski}}~}
\newcommand{\ttask}{\stepcounter{taskii}\textbf{(\alph{taskii})}~}
\newcommand{\tttask}{\stepcounter{taskiii}\quad(\roman{taskiii})~}

\begin{document}
\begin{center}
    \textbf{Analysis II\\9. Abgabeblatt}\\[2em]
	\def\arraystretch{2}
    \begin{tabular}{|l|l|l||p{18mm}|}
        \hline
         Aufgabe 1 & Aufgabe 2 & Aufgabe 3 & Summe:~ \\
         \hline &&&\\
         \hline  
    \end{tabular}
\end{center}
\begingroup
\def\arraystretch{1.5}
\begin{tabular}{p{.5\textwidth}p{.5\textwidth}}
	\hline
    Übungsgruppe: Do 14:00 ~~ SR B& Tutor: Levin Maier \\
    Namen: Ellen Bräutigam, Kamal Abdellatif &\\
    \hline
\end{tabular}
\endgroup\\

\setcounter{taski}{2}
\task
Sei $\mathfrak{F} = \qty{A_i}_{i \in I}$ eine Familie von zusammenhängenden Teilmengen $A_i \subset X ~(i \in I)$ einer Topologie $X$ mit einem gemeinsamen Punkt $z \in X$
\[ \forall i \in I: z \in A_i \quad. \]
\begin{lemma}
    Sind $P,\,Q \subset \bigcup_{i \in I} A_i$ offene nichtleere Teilmengen mit $P \cup Q = \bigcup_{i \in I} A_i ~,~ P \cap Q = \emptyset$, so gilt
    \[ \forall i \in I: A_i \cap P \in \qty{\emptyset,\,A_i} \]
\end{lemma}
\begin{proof} durch Widerspruch

Es sei ein $A_i ~ i \in I$ sodass $P' \coloneqq A_i \cap P \neq \emptyset$ und $Q' \coloneqq A_i \cap Q \neq \emptyset$. Letzteres ist äquivalent zu $A_i \cap P \neq A_i$ da $A_i \subset P \cup Q$.
\begin{gather*}
    A_i = A_i \cap \bigcup_{i \in I} A_i = A_i \cap (P \cup Q) = P' \cap Q' \\
    P' \cap Q' = A_i \cap (P \cap Q) = \emptyset
\end{gather*}
Es wurde in Aufgabe \textbf{(2)} des vorherigen Blattes bewiesen, dass die Schnittmenge einer offenen Menge mit einer Teilmenge offen bezüglich dieser ist. Demnach sind $P',\,Q'$ offen bezüglich $A_i$. Dann würde nach Definition $A_i$ allerdings nicht zusammenhängend sein, was ein Widerspruch mit der Voraussetzung wäre.
\end{proof}
\begin{theorem}
    $\bigcup_{i \in I} A_i$ ist zusammenhängend.
\end{theorem}
\begin{proof} durch Widerspruch

Annahme: Sei $\bigcup_{i \in I} A_i$ nicht zusammenhängend. Demnach gibt es offene nichtleere Teilmengen $P,\,Q \subset \bigcup_{i \in I} A_i$ mit $P \cup Q = \bigcup_{i \in I} A_i ~,~ P \cap Q = \emptyset$.

Sei o.B.d.A $z \in P$. Aus dem Lemma folgt so für alle $i \in I$ aus der anfänglichen Bedingung für $z$
\[ z \in A_i \implies A_i \cap P \neq \emptyset \implies A_i \subset P \]
Da $P$ und $Q$ nach Voraussetzung disjunkt folgt
\[ \forall i \in I: A_i \cap Q = \emptyset \implies Q \cap \bigcup_{i \in I} A_i = \emptyset\qc Q \cap (P \cup Q) = \emptyset \qc Q = \emptyset \quad \text{\Large\Lightning} \]
\end{proof}
\end{document}
