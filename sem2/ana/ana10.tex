\documentclass[a4paper, 12pt]{scrartcl}

\usepackage[utf8]{inputenc}
\usepackage[T1]{fontenc}
\usepackage[ngerman]{babel}

\usepackage{amssymb}
\usepackage{amsmath}
\usepackage{physics}
\usepackage{framed}
\usepackage{float}
\usepackage{mathtools}
\usepackage{marvosym}
\usepackage[shortlabels]{enumitem}

\usepackage{tikz}
\usepackage{chngcntr}

\usepackage{amsthm}
\usepackage{thmtools}

\usepackage[left=2cm, right=2cm, top=2cm]{geometry}

\allowdisplaybreaks
\setlength{\parindent}{0pt}
\setkomafont{paragraph}{\normalfont\itshape}
\pagenumbering{gobble}

\declaretheoremstyle[%
  spaceabove=0,%
  spacebelow=6pt,%
  headfont=\normalfont\itshape,%
  postheadspace=1em,%
  headpunct={}
]{mystyle}

\declaretheorem[name={Behauptung}, style=mystyle, unnumbered]{theorem}
\declaretheorem[name={Lemma}, style=mystyle]{lemma}
\declaretheorem[name={Bemerkung}, style=mystyle, unnumbered]{note}
\declaretheorem[name={Voraussetzung}, style=mystyle, unnumbered]{precondition}
\let\proof\oldproof
\declaretheorem[name={Beweis}, style=mystyle, qed=\qedsymbol, unnumbered]{proof}

\newcounter{taski}
\newcounter{taskii}[taski]
\newcounter{taskiii}[taskii]

\newcommand{\task}{\stepcounter{taski}\textbf{Aufgabe \arabic{taski}}~}
\newcommand{\ttask}{\stepcounter{taskii}\textbf{(\alph{taskii})}~}
\newcommand{\tttask}{\stepcounter{taskiii}\quad(\roman{taskiii})~}

\begin{document}
\begin{center}
    \textbf{Analysis II\\10. Abgabeblatt}\\[2em]
	\def\arraystretch{2}
    \begin{tabular}{|l|l|l||p{18mm}|}
        \hline
         Aufgabe 1 & Aufgabe 2 & Aufgabe 3 & Summe:~ \\
         \hline &&&\\
         \hline  
    \end{tabular}
\end{center}
\begingroup
\def\arraystretch{1.5}
\begin{tabular}{p{.5\textwidth}p{.5\textwidth}}
	\hline
    Übungsgruppe: Do 14:00 ~~ SR B& Tutor: Levin Maier \\
    Namen: Ellen Bräutigam, Kamal Abdellatif &\\
    \hline
\end{tabular}
\endgroup\\

\setcounter{taski}{2}
\task
\ttask Sei $(x_n)_{n \in \mathbb{N}} \in \qty(\mathbb{R}^2)^{\mathbb{N}}$ eine beliebige Folge mit $\lim_{n \rightarrow \infty} x_n = (0,\,0)$. Es gilt
\begin{gather*}
    g: \mathbb{R}^2 \rightarrow \mathbb{R} ~,~ (x,\,y) \mapsto -\qty|\sin x| \qquad h: \mathbb{R}^2 \rightarrow \mathbb{R} ~,~ (x,\,y) \mapsto \qty|\sin x| \\
    g(x_n) \leq f(x_n) \leq h(x_n) \quad \forall n \in \mathbb{N} 
\end{gather*}
Da $g$ als Komposition von $C^0(\mathbb{R}^2)$-Funktionen stetig ist gilt
\[ \lim_{n \rightarrow \infty} g(x_n) = g(0,\,0) = 0 \qquad \lim_{n \rightarrow \infty} h(x_n) = -\lim_{n \rightarrow \infty} g(x_n) = 0 \]
Aus Sandwich-Kriterium folgt
\[ \lim_{n \rightarrow \infty} f(x_n) = 0 = f(0,\,0) \]
Demnach ist nach Definition $f$ in $(0,\,0)$ stetig.

\ttask
\begin{lemma}
    Jede Gerade der Form 
\[ \qty{(x,\,y) \in \mathbb{R}^2 \mid ax + by = 0} \quad a,\,b \in \mathbb{R} ~,~ (a,\,b) \neq (0,\,0) \]
    hat maximal zwei Schnittpunkte mit der Normalparabel $\qty{(t,\,t^2) \mid t \in \mathbb{R}}$.
\end{lemma}
\begin{proof} Seien $a,\,b \in \mathbb{R}$ mit $(a,\,b) \neq (0,\,0)$ gegeben, $(x,\,y) \in \mathbb{R}^2$ ein Schnittpunkt. So gilt
\begin{align*}
    ax + by &= ax + bx^2 = 0 \\
    0 &= x(bx + a)
\end{align*}
Fall 1: $b = 0: \implies a \neq 0 \qc 0 = ax \implies x = 0$. $y = x^2 = 0$ \quad\qc $(x,\,y) = (0,\,0)$ \\
Fall 2: $a = 0: \implies b \neq 0 \qc 0 = bx^2 = by \implies x = y = 0$ \quad\qc $(x,\,y) = (0,\,0)$ \\
Fall 3: $a,\,b \neq 0: \implies (x,\,y) \in \qty{(0,\,0),~\qty(-\frac{a}{b}, \frac{a^2}{b^2})}$
\end{proof}

Sei der Einheitsvektor $\nu \in \mathbb{R}^2$ beliebig und $(t_n)_{n \in \mathbb{N}} \in \mathbb{R}_*^{\mathbb{N}}$ eine Nullfolge.

Da nach Lemma die Folge $(t_n\nu)_{n \in \mathbb{N}}$ endlich viele Schnittpunkte mit der Normalparabel hat, gilt nach Definition von $f$
\[ \exists N \in \mathbb{N}: \forall n \in \mathbb{N}~,~n>N: f(t_n\nu) = \frac{f(t_n\nu)}{t_n} = 0 \]
Damit folgt nach Definition des Limes sofort
\[ \lim_{n \rightarrow \infty} \frac{f(x_0 + t_n\nu) - f(x_0)}{t_n} = \lim_{n \rightarrow \infty} \frac{f(t_n\nu)}{t_n} = 0 \]
Da $t_n$ von Anfang beliebig war, existiert die Richtungsableitung $\displaystyle\lim_{\substack{s \rightarrow 0\\s \in \mathbb{R}}}\frac{f(x_0 + s\nu) - f(x_0)}{s} = 0$ \\
für beliebige Richtungen $\nu \in \mathbb{R}^2$ im Punkte $x_0 = (0,\,0)$.
\end{document}
