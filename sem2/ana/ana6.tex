\documentclass[a4paper, 12pt]{scrartcl}

\usepackage[utf8]{inputenc}
\usepackage[T1]{fontenc}
\usepackage[ngerman]{babel}

\usepackage{amssymb}
\usepackage{amsmath}
\usepackage{physics}
\usepackage{framed}
\usepackage{float}
\usepackage{mathtools}
\usepackage{marvosym}

\usepackage{tikz}
\usepackage{chngcntr}

\usepackage{amsthm}
\usepackage{thmtools}

\usepackage[left=2cm, right=2cm, top=2cm]{geometry}

\allowdisplaybreaks

\setlength{\parindent}{0pt}

\setkomafont{paragraph}{\normalfont\itshape}


\declaretheoremstyle[%
  spaceabove=0,%
  spacebelow=6pt,%
  headfont=\normalfont\itshape,%
  postheadspace=1em,%
  headpunct={}
]{mystyle}

\declaretheorem[name={Behauptung}, style=mystyle, unnumbered]{theorem}
\declaretheorem[name={Lemma}, style=mystyle]{lemma}
\declaretheorem[name={Voraussetzung}, style=mystyle, unnumbered]{precondition}
\let\proof\oldproof
\declaretheorem[name={Beweis}, style=mystyle, qed=\qedsymbol, unnumbered]{proof}

\newcounter{taski}
\newcounter{taskii}[taski]
\newcounter{taskiii}[taskii]

\newcommand{\task}{\stepcounter{taski}\textbf{Aufgabe \arabic{taski}}~}
\newcommand{\ttask}{\stepcounter{taskii}\textbf{(\alph{taskii})}~}
\newcommand{\tttask}{\stepcounter{taskiii}\quad(\roman{taskiii})~}

\pagenumbering{gobble}

\begin{document}
\begin{center}
    \textbf{Analysis II\\6. Abgabeblatt}\\[2em]
	\def\arraystretch{2}
    \begin{tabular}{|l|l|l||p{18mm}|}
        \hline
         Aufgabe 1 & Aufgabe 2 & Aufgabe 3 & Summe:~ \\
         \hline &&&\\
         \hline  
    \end{tabular}
\end{center}
\begingroup
\def\arraystretch{1.5}
\begin{tabular}{p{.5\textwidth}p{.5\textwidth}}
	\hline
    Übungsgruppe: Do 14:00 ~~ SR B& Tutor: Levin Maier \\
    Namen: Ellen Bräutigam, Kamal Abdellatif &\\
    \hline
\end{tabular}
\endgroup
\\
\setcounter{taski}{1}

\task
\ttask
\tttask
Es sei ein beliebiges $\alpha \in \mathbb{R}$. Die Koeffizientenfolge $(a_n)_{n \in \mathbb{N}}$ ist gegeben als
\begin{align*}
    a_n &= n^\alpha \quad n \in \mathbb{N} ~,~ \\
    \lim_{n \rightarrow \infty} \sqrt[n]{|a_n|} 
        &=\lim_{n \rightarrow \infty} n^{\frac{\alpha}{n}}
        = \lim_{n \rightarrow \infty} \exp\qty(\frac{\alpha \ln n}{n})\qquad (n^\alpha > 0 ~\forall n \in \mathbb{N}) \\
        &\overset{\exp \in C^\infty}= \exp\qty(\lim_{n \rightarrow \infty} \frac{\alpha \ln n}{n}) \\
        \intertext{Mit $n \neq 0 ~\forall n \in \mathbb{N} ~,~ \ln, \mathrm{id}_\mathbb{R} \in C^\infty$ und $\lim_{n \rightarrow \infty} \ln n = \lim_{n \rightarrow \infty} n = \infty$ gilt nach \textsc{L'Hospital}}
        &= \exp\qty(\lim_{n \rightarrow \infty} \frac{\dv{n} \alpha \ln n}{\dv{n} n})
        = \exp\qty(\lim_{n \rightarrow \infty} \frac{\alpha \frac 1n}{1}) = e^0 = 1, \\\\
      r &= \frac{1}{\limsup_{n \rightarrow \infty}\sqrt[n]{|a_n|}} = \frac{1}{\lim_{n \rightarrow \infty} \sqrt[n]{|a_n|}} = \frac{1}{1} = 1.
\end{align*}
\tttask
Die Koeffizientenfolge $(a_n)_{n \in \mathbb{N}}$ ist gegeben als
\[ a_n = \frac{n^n}{n!} \qquad (n \in \mathbb{N}). \]
Es gilt demnach
\begin{align*}
    \frac{a_{n+1}}{a_n} &= \frac{(n+1)^{n+1}}{(n+1)!} \frac{n!}{n^n} = \frac{(n+1)^n}{n^n} = \qty(1+\frac{1}{n})^n, \\
    \rho &= \lim_{n \rightarrow \infty} \abs{\frac{a_{n+1}}{a_n}}
    = \lim_{n \rightarrow \infty} \qty(1+\frac{1}{n})^n = e~.
\end{align*}
Nach dem Satz der im vorherigen Blatt in Aufgabe 2 gelöst wurde für den Konvergenzradius $r$
\[ r = \frac{1}{\rho} = \frac{1}{e}~.  \]
\newpage
\ttask
\tttask $\arctan \in C^\infty{\mathbb R}$. Nach Satz von \textsc{Taylor} gilt um die Stelle $x_0 = 0$ für $x \in \mathbb{R}$
\begin{align*}
    \arctan x &= \sum_{n=0}^\infty \frac{\arctan^{(n)}0}{n!}x^n \qquad (\arctan^{(2n)}0 = 0 ~\forall n \in \mathbb{N}) \\
              &= \sum_{n=0}^\infty \frac{\arctan^{(2n+1)}0}{(2n+1)!}x^{(2n+1)}
              = \sum_{n=0}^\infty \frac{(-1)^n(2n)!}{(2n+1)!}x^{2n+1}
              = \sum_{n=0}^\infty \frac{(-1)^n}{2n+1}x^{2n+1} \\
              &= 1 - \frac{1}{3}x^3 + \frac{1}{5}x^5 \mp \dots
\end{align*}
Für $x = 1$ erhält man
\[ \arctan 1 = \sum_{n=0}^\infty \frac{(-1)^n}{2n+1}  \]
Da $(\frac{1}{2n+1})_{n \in \mathbb{N}}$ eine Nullfolge ist, ist nach \textsc{Leibniz} die alterniernde Folge konvergent. Nach \textsc{Abel}'schen Grenzwertsatz folgt somit
\[  1 - \frac{1}{3}x^3 + \frac{1}{5}x^5 \mp \dots = \lim_{x \rightarrow 1} \arctan x = \arctan 1 = \frac{\pi}{4} \]
\begin{lemma}
    \[ \arctan^{(n)}(0) = \begin{cases}
            0 \quad& 2 \mid n \\
            (-1)^{\frac{n-1}2}(n-1)! \quad& 2 \nmid n
        \end{cases} \qquad (n \in \mathbb{N})
    \]
\end{lemma}
\begin{proof}
Durch Induktion.
\paragraph*{IA}\quad $n = 0 :~ \arctan^{(2n)} 0 = 0\quad\qc \arctan^{(2n+1)} 0 = \frac{1}{1+0^2} = 1 = (-1)^0(1-1)!$
\paragraph*{IB}\quad $n:~ \arctan^{(2n)} 0 = 0 \quad\qc \arctan^{(2n+1)} 0 = (-1)^n(2n)!$
\paragraph*{IS} Sehr krampfig formal zu schreiben und aus Zeitgründen ausgelassen.
\end{proof}
\end{document}
