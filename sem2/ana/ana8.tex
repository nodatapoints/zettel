\documentclass[a4paper, 12pt]{scrartcl}

\usepackage[utf8]{inputenc}
\usepackage[T1]{fontenc}
\usepackage[ngerman]{babel}

\usepackage{amssymb}
\usepackage{amsmath}
\usepackage{physics}
\usepackage{framed}
\usepackage{float}
\usepackage{mathtools}
\usepackage{marvosym}
\usepackage[shortlabels]{enumitem}

\usepackage{tikz}
\usepackage{chngcntr}

\usepackage{amsthm}
\usepackage{thmtools}

\usepackage[left=2cm, right=2cm, top=2cm]{geometry}

\allowdisplaybreaks

\setlength{\parindent}{0pt}

\setkomafont{paragraph}{\normalfont\itshape}


\declaretheoremstyle[%
  spaceabove=0,%
  spacebelow=6pt,%
  headfont=\normalfont\itshape,%
  postheadspace=1em,%
  headpunct={}
]{mystyle}

\declaretheorem[name={Behauptung}, style=mystyle, unnumbered]{theorem}
\declaretheorem[name={Lemma}, style=mystyle]{lemma}
\declaretheorem[name={Bemerkung}, style=mystyle, unnumbered]{note}
\declaretheorem[name={Voraussetzung}, style=mystyle, unnumbered]{precondition}
\let\proof\oldproof
\declaretheorem[name={Beweis}, style=mystyle, qed=\qedsymbol, unnumbered]{proof}

\newcounter{taski}
\newcounter{taskii}[taski]
\newcounter{taskiii}[taskii]

\newcommand{\task}{\stepcounter{taski}\textbf{Aufgabe \arabic{taski}}~}
\newcommand{\ttask}{\stepcounter{taskii}\textbf{(\alph{taskii})}~}
\newcommand{\tttask}{\stepcounter{taskiii}\quad(\roman{taskiii})~}

\begin{document}
\begin{center}
    \textbf{Analysis II\\8. Abgabeblatt}\\[2em]
	\def\arraystretch{2}
    \begin{tabular}{|l|l|l||p{18mm}|}
        \hline
         Aufgabe 1 & Aufgabe 2 & Aufgabe 3 & Summe:~ \\
         \hline &&&\\
         \hline  
    \end{tabular}
\end{center}
\begingroup
\def\arraystretch{1.5}
\begin{tabular}{p{.5\textwidth}p{.5\textwidth}}
	\hline
    Übungsgruppe: Do 14:00 ~~ SR B& Tutor: Levin Maier \\
    Namen: Ellen Bräutigam, Kamal Abdellatif &\\
    \hline
\end{tabular}
\endgroup\\

\setcounter{taski}{2}
\task
\begin{align*}
    f(t) &= (\cos 2\pi t,\,\sin 2\pi t) \quad \forall t \in [0,1) \\
    \cos^2 2\pi t + \sin^2 2\pi t &\overset{\text{Trig. Pyth.}}= 1 \implies f(t) \in \mathcal{S}^1 \quad \forall t \in [0,1)
\end{align*}
Demnach ist $f$ wohldefiniert.
\begin{theorem}
    $f: [0,1) \rightarrow \mathcal{S}^1 ~,~ t \mapsto (\cos 2\pi t,\,\sin 2\pi t)$ ist stetig.
\end{theorem}
\begin{proof}
    Sei für ein beliebiges $t_0$ für den Wert $f(t) = x_0 \in \mathcal{S}^1$ die Umgebung $U_\varepsilon(x_0)$ mit $\varepsilon > 0$ gegeben. Es sei
\[ \delta \coloneqq \frac{\varepsilon}{4\pi}  \]
Da $[0,1) \subset \mathbb{R}$ gilt $U_\delta(t_0) = (t_0-\delta,t_0+\delta)$. Sei $t = t_0 + d$ mit $\norm{d} < \delta$. Unter Verwendung der Additionstheoreme gilt 
\begin{align*}
    f(t) &= f(t+d) = \qty(\cos(2\pi t)\cos(2\pi d) - \sin(2\pi t)\sin(2\pi d), \sin(2\pi t)\cos(2\pi d) + \cos(2\pi t)\sin(2\pi d)) \\
\end{align*}
[Rest wegen Zeitnot nicht fertiggestellt]
\end{proof}

\end{document}
