\documentclass[a4paper, 12pt]{scrartcl}

\usepackage[utf8]{inputenc}
\usepackage[T1]{fontenc}
\usepackage[ngerman]{babel}

\usepackage{amssymb}
\usepackage{amsmath}
\usepackage{physics}
\usepackage{framed}
\usepackage{float}
\usepackage{mathtools}
\usepackage{marvosym}
\usepackage[shortlabels]{enumitem}

\usepackage{tikz}
\usepackage{chngcntr}

\usepackage{amsthm}
\usepackage{thmtools}

\usepackage[left=2cm, right=2cm, top=2cm]{geometry}

\allowdisplaybreaks
\setlength{\parindent}{0pt}
\setkomafont{paragraph}{\normalfont\itshape}
\pagenumbering{gobble}

\declaretheoremstyle[%
  spaceabove=0,%
  spacebelow=6pt,%
  headfont=\normalfont\itshape,%
  postheadspace=1em,%
  headpunct={}
]{mystyle}

\declaretheorem[name={Behauptung}, style=mystyle, unnumbered]{theorem}
\declaretheorem[name={Lemma}, style=mystyle]{lemma}
\declaretheorem[name={Bemerkung}, style=mystyle, unnumbered]{note}
\declaretheorem[name={Voraussetzung}, style=mystyle, unnumbered]{precondition}
\let\proof\oldproof
\declaretheorem[name={Beweis}, style=mystyle, qed=\qedsymbol, unnumbered]{proof}

\newcounter{taski}
\newcounter{taskii}[taski]
\newcounter{taskiii}[taskii]

\newcommand{\task}{\stepcounter{taski}\textbf{Aufgabe \arabic{taski}}~}
\newcommand{\ttask}{\stepcounter{taskii}\textbf{(\alph{taskii})}~}
\newcommand{\tttask}{\stepcounter{taskiii}\quad(\roman{taskiii})~}

\begin{document}
\begin{center}
    \textbf{Analysis II\\11. Abgabeblatt}\\[2em]
	\def\arraystretch{2}
    \begin{tabular}{|l|l|l||p{18mm}|}
        \hline
         Aufgabe 1 & Aufgabe 2 & Aufgabe 3 & Summe:~ \\
         \hline &&&\\
         \hline  
    \end{tabular}
\end{center}
\begingroup
\def\arraystretch{1.5}
\begin{tabular}{p{.5\textwidth}p{.5\textwidth}}
	\hline
    Übungsgruppe: Do 14:00 ~~ SR B& Tutor: Levin Maier \\
    Namen: Ellen Bräutigam, Kamal Abdellatif &\\
    \hline
\end{tabular}
\endgroup\\

\setcounter{taski}{2}
\task Denotiere $\partial_ig(x_0)$ die partielle Ableitung von $g$ an der Stelle $x_0$ nach der $i$-ten Komponente.

Sei $(x,\,y) = 0$
\begin{align*}
    \partial_1g(x,\,y) &= \pdv{x}f(ax+by) \overset{\text{Kettenr.}}= a\cdot f'(ax+by) \\
    \partial_2g(x,\,y) &= \pdv{y}f(ax+by) \overset{\text{Kettenr.}}= b\cdot f'(ax+by)
\end{align*}
Da $f \in C^{q+1}(0)$ gilt dies auch für höhere Ableitungen von $g$ bis zur Ordnung $q$. Mit jeder partiellen Ableitung erhöht sich die Ordnung der Ableitung und der entsprechende Faktor kommt hinzu.

Seien $i_1,\,\dots,\,i_m \in \qty{1,\,2} ~,~ m \leq q$ eine beliebige Permutation von $m$ Komponenten, zusammengesetzt aus $j$ Einsen und $m-j$ Zweien. So gilt mit $a = c_1 ~,~ b = c_2$ und $h = (x,\,y) \in U_\delta{(0)}$
\begin{align*}
    \partial_{i_1}\partial_{i_2}\cdots\partial_{i_m}f(0) &= c_1c_2\cdots c_mf^{(m)}(0) = a^j b^{m-j}f^{(m)}(0) \\
    \partial_{i_1}\partial_{i_2}\cdots\partial_{i_m}f(0)h^{(i_1)}h^{(i_2)}\cdots h^{(i_m)} &= f^{(m)}(0)(ax)^j(by)^{m-j}
\end{align*}
Es gibt $\binom m k$ Permutationen dieser Art für gegebenes $m,\,j$. Wählt man $m \leq q$ fest und summiert den unten gegeben Term, erhält man
\[
    \sum_{i_1,\dots,i_m=1}^2 \partial_{i_1}\partial_{i_2}\cdots\partial_{i_m}f(0)h^{(i_1)}h^{(i_2)}\cdots h^{(i_m)} =
    \sum_{j=0}^m\binom m jf^{(m)}(0)(ax)^j(by)^{m-j}
\]
Eingesetzt in die Definition der Taylor-Entwicklung von $g$ um die Stelle $0$ erhält man schlussendlich
\[ 
    g(x,\,y) = \sum_{m=0}^q \frac{f^{(m)}(0)}{m!}\sum_{j=0}^m\binom m j(ax)^j(by)^{m-j} + R_{q+1}(x,\,y)
\]
Das Restglied $R_{q+1}(x,\,y)$ ist nach Definition gegeben als
\begin{align*}
    R_{q+1}(x,\,y) &= 
    \sum_{i_1,\dots,i_{q+1}=1}^2 \partial_{i_1}\partial_{i_2}\cdots\partial_{i_{q+1}}f(asx+bsy)h^{(i_1)}h^{(i_2)}\cdots h^{(i_{q+1})} \\
    &= \frac{1}{(q+1)!}\sum_{j=0}^{q+1}\binom {q+1} jf^{(q+1)}(s(ax+by))(sax)^j(sby)^{q+1-j} \\
    &= \frac{f^{(q+1)}(s(ax+by))}{(q+1)!}\qty(s(ax+by))^{q+1} \qquad s \in (0,\,1)
\end{align*}
\end{document}
