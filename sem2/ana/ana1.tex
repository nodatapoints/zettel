\documentclass[a4paper, 12pt]{scrartcl}

\usepackage[utf8]{inputenc}
\usepackage[T1]{fontenc}
\usepackage[ngerman]{babel}

\usepackage{amssymb}
\usepackage{amsmath}
\usepackage{physics}
\usepackage{framed}
\usepackage{float}
\usepackage{mathtools}
\usepackage{marvosym}

\usepackage{tikz}
\usepackage{chngcntr}

\usepackage{amsthm}
\usepackage{thmtools}

\usepackage[left=2cm, right=2cm, top=2cm]{geometry}

\allowdisplaybreaks

\setlength{\parindent}{0pt}

\setkomafont{paragraph}{\normalfont\itshape}

\declaretheoremstyle[%
  spaceabove=0,%
  spacebelow=6pt,%
  headfont=\normalfont\itshape,%
  postheadspace=1em,%
  headpunct={}
]{mystyle}

\declaretheorem[name={Behauptung}, style=mystyle, unnumbered]{theorem}
\declaretheorem[name={Lemma}, style=mystyle]{lemma}
\declaretheorem[name={Voraussetzung}, style=mystyle, unnumbered]{precondition}
\let\proof\oldproof
\declaretheorem[name={Beweis}, style=mystyle, qed=\qedsymbol, unnumbered]{proof}

\newcounter{taski}
\newcounter{taskii}[taski]
\newcounter{taskiii}[taskii]

\newcommand{\task}{\stepcounter{taski}\textbf{Aufgabe \arabic{taski}}~}
\newcommand{\ttask}{\stepcounter{taskii}\textbf{(\alph{taskii})}~}
\newcommand{\tttask}{\stepcounter{taskiii}\quad(\roman{taskiii})~}

\begin{document}
\setcounter{taski}{1}
\task
\begin{lemma}
Es sei eine Funktion $f: U \rightarrow \mathbb{R} ~,~ U \subseteq \mathbb{R}$ mit $x_0 \in U$ an der Stelle $x_0$ diff'bar. So gilt
\[ f'(x_0) > 0 \quad \Rightarrow \quad \exists \varepsilon > 0 : \forall x \in (x_0,x_0+\varepsilon] : f(x) > f(x_0) \]
\end{lemma}
\begin{proof} durch Widerspruch. Gegenbehauptung:
\[ f'(x_0) > 0 \quad \wedge \quad \forall \varepsilon > 0: \exists x \in (x_0,x_0+\varepsilon] : f(x) \leq f(x_0) \]
Es kann demnach eine Folge $(a_n)_{n \in \mathbb{N}} \in \mathbb{R}^\mathbb{N}$ konstruiert werden: Man nehme eine beliebige positive Nullfolge $(\delta_n)_{n \in \mathbb{N}} \in \mathbb{R}_+^\mathbb{N}$. Nach Beh. mit $\varepsilon = \delta_n$ existiert für beliebiges $n$ ein $a_n \in (x_0,x_0+\delta_n]$ mit $f(x_0 + a_n) \leq f(x_0)$. Nach Majorantenkriterium
\[ \lim_{n \rightarrow \infty} \delta_n = 0 \quad \wedge \quad \forall n \in \mathbb{N}: 0 < a_n \leq \delta_n \qquad \Longrightarrow \qquad \lim_{n \rightarrow \infty} a_n = 0^+\]
Betrachten wir den Limes
\[ \lim_{n \rightarrow \infty} \frac{f(x_0+a_n)-f(x_0)}{a_n} = f'(x_0) \]
Da nach Beh. $f(x_0+a_n) \leq f(x_0)$ und $a_n > 0$ gilt
\[ f'(x_0) \leq 0 \quad \text{\Large\Lightning} \]
\end{proof}
\task
\ttask
\begin{align*}
	\dv{x} \qty(-\frac 1 {2(1+x^2)}) &= \frac x {(1+x^2)^2} \\
	\int \dd x \frac x {(1+x^2)^2} &= \int \dd x \dv{x} \qty(-\frac 1 {2(1+x^2)}) \\
	\xRightarrow{\text {Hauptsatz}} \qquad &= -\frac 1 {2(1+x^2)} + C
\end{align*}
\ttask
\begin{align*}
	\int_0^1 \dd x \frac{2x^2}{(1+x^2)^2} &= \int_0^1 \dd x\ {x} \cdot \frac{2x}{(1+x^2)^2} \\
	\xRightarrow{\text{part. Int.}} \qquad
	&= x \qty(-\frac 1 {1+x^2}) \Bigg|_0^1 + \int_0^1 \dd x\ \frac 1 {1+x^2} = - \frac 1 2 + \int_0^1 \dd x\ \frac 1 {1+x^2}
\end{align*}
\begin{align*}
	\int_0^1 \dd x \frac{x^2-1}{(1+x^2)^2} &= \int_0^1 \dd x \frac{2x^2}{(1+x^2)^2} - \int_0^1 \dd x \frac 1 {1+x^2} = - \frac 1 2 + \int_0^1 \dd x\ \frac 1 {1+x^2} -\int_0^1 \dd x\ \frac 1 {1+x^2} \\
	&= - \frac 1 2
\end{align*}
\end{document}
