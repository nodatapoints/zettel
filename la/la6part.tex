\documentclass[a4paper, 12pt]{scrartcl}

\usepackage[utf8]{inputenc}
\usepackage[T1]{fontenc}
\usepackage[ngerman]{babel}

\usepackage{amssymb}
\usepackage{amsmath}
\usepackage{framed}
\usepackage{float}
\usepackage{mathtools}
\usepackage{marvosym}

\usepackage{tikz}
\usepackage{chngcntr}

\usepackage{amsthm}
\usepackage{thmtools}

\usepackage[left=2cm, right=2cm, top=2cm]{geometry}

\allowdisplaybreaks

\setlength{\parindent}{0pt}

\setkomafont{paragraph}{\normalfont\itshape}


\declaretheoremstyle[%
  spaceabove=0,%
  spacebelow=6pt,%
  headfont=\normalfont\itshape,%
  postheadspace=1em,%
  headpunct={}
]{mystyle}

\declaretheorem[name={Behauptung}, style=mystyle, unnumbered]{theorem}
\declaretheorem[name={Lemma}, style=mystyle]{lemma}
\declaretheorem[name={Voraussetzung}, style=mystyle, unnumbered]{precondition}
\let\proof\oldproof
\declaretheorem[name={Beweis}, style=mystyle, qed=\qedsymbol, unnumbered]{proof}

\newcounter{taski}
\newcounter{taskii}[taski]
\newcounter{taskiii}[taskii]

\newcommand{\task}{\stepcounter{taski}\textbf{Aufgabe \arabic{taski}}}
\newcommand{\ttask}{\stepcounter{taskii}\textbf{(\alph{taskii})}}
\newcommand{\tttask}{\stepcounter{taskiii}\quad(\roman{taskiii})}

\newcommand{\defimpl}[1]{\stackrel{\text{Def.}\;#1}{\Longrightarrow}}
\newcommand{\defImpl}[1]{\stackrel{\text{Def.}\;#1}{\Longleftrightarrow}}
\newcommand{\txtimpl}[1]{\stackrel{\text{#1}}{\Longrightarrow}}
\newcommand{\txtImpl}[1]{\stackrel{\text{#1}}{\Longleftrightarrow}}
\newcommand{\refimpl}[1]{\txtimpl{\eqref{#1}}}
\setcounter{taski}{20}

\begin{document}
\hfill \textit{Ellen, Kamal}\\
\task

\ttask

$(v_1,\ v_2,\ v_3)$ ist nach Definition von $U$ ein ES von $U$.
Es seien $\lambda_1,\ \lambda_2,\ \lambda_3 \in \mathbb{Q}$ sodass
\begin{align}
	 \lambda_1 - 3\lambda_2 + \lambda_3 &= 0 \label{x1} \\ 
	-2\lambda_1 + 4\lambda_2 &= 0 \label{x2} \\ 
	 \lambda_1 - \lambda_2 + 5\lambda_3 &= 0 \label{x3}
\end{align}
\begin{gather}
	\refimpl{x2} \lambda_1 = 2 \lambda_2  \label{fehler2}\\
	\refimpl{x1} \quad -\lambda_2 + \lambda_3 = 0\ ,\ \lambda_2 = \lambda_3 \nonumber\\
	\refimpl{x3} 6 \lambda_2 = 0  \nonumber\\
	\Longrightarrow \quad \lambda_1 = 0\ ,\  \lambda_2 = \lambda_3 = 0 \label{fehler6}
\end{gather}

Somit sind $(v_1,\ v_2,\ v_3)$ linear unabhängig und bilden eine Basis der Länge $3 = \dim_\mathbb{Q} U$.

\ttask

Wieder ist $(v_1,\ v_2,\ v_3)$ nach Definition von $U$ ein ES von $U$. Diesmal seien $\lambda_1,\ \lambda_2,\ \lambda_3 \in \mathbb{F}_p$. Zur linearen Unabhängigkeit wird wie oben vorgegangen. Jedoch sind bei Schritt \eqref{fehler2} und \eqref{fehler6} Einschränkungen vorzunehmen. Die Vorgehensweise in Schritt \eqref{fehler2} ist genau genommen
\begin{gather*}
	-2 \lambda_1 + 4 \lambda_2 = 0 \quad,\quad 2 \lambda_1 = 4 \lambda_2 \quad,\quad 2^{-1}2 \lambda_1 = 2^{-1}4 \lambda_2 \\
	\lambda_1 = 2 \lambda_2
\end{gather*}
Jedoch existiert $2^{-1}$ in $\mathbb{F}_p$ nur dann  wenn $p \neq 2$. Nach gleicher Argumentation ist Schritt \eqref{fehler6} auf $p \neq 2,\ 3$ beschränkt. Für alle Primzahlen $p \neq 2,\ 3$ bildet $(v_1,\ v_2,\ v_3)$ so eine Basis.

Für $p = 2$ ist
	\[ U = \left\{ \begin{pmatrix} 0 \\ 0 \\ 0 \end{pmatrix},\ \begin{pmatrix} 1 \\ 0 \\ 1 \end{pmatrix} \right\} \]
sodass $(v_1)$ eine Basis bildet.
\\

Für $p = 3$ gilt $v_1 = 2v_2 + v_3$. $(v_2,\ v_3)$ sind linear unabhängig, da $\nexists \lambda \in \mathbb{F}_3 : \lambda v_{2_1} = \lambda \cdot 0 = 1 = v_{3_1} $. Somit bildet $(v_2,\ v_3)$ eine Basis.

\[ 
\begin{cases}
	(v_1) \quad \dim_{\mathbb{F}_p} = 1 &\qquad p = 2 \\
	(v_2,\ v_3) \quad \dim_{\mathbb{F}_p} = 2 &\qquad p = 3 \\
	(v_1,\ v_2,\ v_3) \quad \dim_{\mathbb{F}_p} = 3 &\qquad \text{sonst}
\end{cases}
 \]
\newpage

\task

\ttask
\begin{theorem}
\[ \forall f \in P_D\ \exists \lambda_1,\ \dots,\ \lambda_D \in \mathbb{Q} : f(x) = \sum_{i=0}^D \lambda_ip_i(x) \]
\end{theorem}
\begin{proof}
Es sei ein beliebiges $f \in P_D$ gegeben.
\begin{gather*}
	\defimpl{P_D} \exists c_0,\ \dots,\ c_D \in \mathbb{Q} : f(x) = \sum_{i=0}^D c_ix^i \\
	\lambda_i \coloneqq c_i \\
	\defimpl{p_i} f(x) = \sum_{i=0}^D \lambda_ip_i(x) 
\end{gather*}
\end{proof}

\begin{theorem}
	$(q_0,\ \dots,\ q_D)$ ist linear unabhängig.
\end{theorem}
\begin{proof} durch vollständige Induktion
\paragraph*{Induktionsanfang} $(q_0)$ ist linear unabhängig.
\paragraph*{Induktionsbehauptung} $(q_0,\ \dots,\ q_k)$ ist linear unabhängig
\paragraph*{Induktionsschritt} Z. z. $\quad(q_0,\ \dots,\ q_{k+1})$ ist linear unabhängig
Es seien $\lambda_0,\ \dots,\ \lambda_{k+1} \in \mathbb{Q}$ sodass
\begin{align*}
	\forall x \in \mathbb{Q}: 0 &= \sum_{i=0}^{k+1} \lambda_iq_i(x) = \lambda_{k+1}q_{k+1}(x) + \sum_{i=0}^k \lambda_iq_i(x)
\intertext{Sei $x = 0$, sodass $q_{k+1}(x) = 0$ da $k+1>0$}
	0 &= \lambda_0p_0(x) = \lambda_0 \\
	\Longrightarrow \quad \forall x \in \mathbb{Q} : 0 &= \sum_{i=1}^{k+1} \lambda_iq_i(x) = \sum_{i=0}^{k} \lambda_{i+1}q_{i+1}(x)
\intertext{Aus Def. $q_i$ folgt $\forall i \in \mathbb{N}\ \forall x \in \mathbb{Q}^* : q_{i+1}(x) = \frac{1}{x+1}q_i(x+1)$}
	\Longrightarrow \quad \forall x \in \mathbb{Q}^* : 0 &= \frac{1}{x+1}\sum_{i=0}^{k} \lambda_{i+1}q_i(x+1) \tag{$x \mapsto x-1$}\\
	\Longrightarrow \quad \forall x \in \mathbb{Q}^* : 0 &= \sum_{i=0}^{k} \lambda_{i+1}q_i(x) \\
	\txtimpl{IB}\defimpl{\text{Lin. Unabh.}} \quad 0 &= \lambda_1 = \dots = \lambda_{k+1} \\
\end{align*}
Somit folgt $\lambda_0 = \dots = \lambda_{k+1} = 0$ sodass $(q_0,\ \dots,\ q_{k+1})$ linear unabhängig ist.
\end{proof}
\ttask

\begin{theorem}
$(p_0,\ \dots,\ p_D)$ sind linear unabhängig.
\end{theorem}
\begin{proof} durch vollständige Induktion
\paragraph*{Induktionsanfang} $(p_0)$ ist linear unabhängig.
\paragraph*{Induktionsbehauptung} $(p_0,\ \dots,\ p_k)$ ist linear unabhängig
\paragraph*{Induktionsschritt} Z. z. $\quad(p_0,\ \dots,\ p_{k+1})$ ist linear unabhängig

Es seien $\lambda_0,\ \dots,\ \lambda_{k+1} \in \mathbb{Q}$ sodass
\begin{align*}
	\forall x \in \mathbb{Q}: 0 &= \sum_{i=0}^{k+1} \lambda_ip_i(x) = \lambda_{k+1}x^{k+1} + \sum_{i=0}^k \lambda_ip_i(x)
\intertext{Sei $x = 0$, sodass $x^{k+1} = 0$ da $k+1>0$}
	0 &= \lambda_0p_0(x) = \lambda_0 \\
	\Longrightarrow \quad 0 &= \sum_{i=1}^{k+1} \lambda_ix^i = x \sum_{i=0}^k \lambda_{i+1}p_i(x) \\
\intertext{Bei $x = 0$ ist $\displaystyle \sum_{i=0}^{k+1}\lambda_ip_i(x) = 0 + \sum_{i=1}^{k+1}\lambda_ix^i = 0$ \qquad Für $x \neq 0$ kann durch $x$ geteilt werden.}
	\Longrightarrow \quad \forall x \in \mathbb{Q}: 0 &= \sum_{i=0}^k \lambda_{i+1}p_i(x) \\
	\txtimpl{IB}\defimpl{\text{Lin. Unabh.}} \quad 0 &= \lambda_1 = \dots = \lambda_{k+1}
\end{align*}
Somit folgt $\lambda_0 = \dots = \lambda_{k+1} = 0$ sodass $(p_0,\ \dots,\ p_{k+1})$ linear unabhängig ist.
\end{proof}
Mit (a) bildet $(p_0,\ \dots,\ p_D)$ eine Basis von $P_D$, sodass $\dim_\mathbb{Q}P_D = D+1$

\ttask
\begin{theorem}
	\[ \dim_\mathbb{Q} P = \infty \]
\end{theorem}
\begin{proof} durch Widerspruch: Sei $P$ endlich erzeugbar durch ein ES $(v_1,\ \dots,\ v_n)$, sodass
nach Def. $P$ gilt $\forall f \in P\ \exists D(f) \in \mathbb{N} : f \in P_D$ . So kann eine endliche Menge $M$ konstruiert werden:
\[ M \coloneqq \left\{ D(f) \mid f \in P \right\} \]
Da $M$ endlich ist, gibt es eine obere Schranke $S \in \mathbb{N}$ sodass
\[ \forall f \in P: D(f) < S \]
$\{p_0,\ \dots,\ p_S\} = \bigcup_{f \in P} \{p_0,\ \dots,\ p_{D(f)}\}$ bildet ein ES, da es eine Basis für jedes $f \in P$ enthält. Da $\{p_0,\ \dots,\ p_{S+1}\}$ jedoch linear unabhängig ist, kann $p_{S+1} \in P$ nicht erzeugt werden. \quad\large \Lightning
\end{proof}
% Detexify
\end{document}