\documentclass[a4paper, 12pt]{scrartcl}

\usepackage[utf8]{inputenc}
\usepackage[T1]{fontenc}
\usepackage[ngerman]{babel}

\usepackage{amssymb}
\usepackage{amsmath}
\usepackage{framed}
\usepackage{float}
\usepackage{mathtools}

\usepackage{tikz}
\usepackage{amsthm}
\usepackage{chngcntr}

\allowdisplaybreaks

\setlength{\parindent}{0pt}

\theoremstyle{plain}
\newtheorem{theorem}{Theorem}
\newtheorem{lemma}[theorem]{Lemma}

\begin{document}
\textbf{Aufgabe 1}\\
(a)\\
(i)
\[ \mathcal{P}(X) = \left\{ \emptyset,
\ \left\{  8 \right\},
\ \left\{  6 \right\},
\ \left\{  6,\ 8 \right\},
\ \left\{ 4 \right\},
\ \left\{ 4,\ 8 \right\},
\ \left\{ 4,\ 6 \right\},
\ \left\{ 4,\ 6,\ 8 \right\}
 \right\} \]\\
(ii)
\begin{align*}
	X \cup Y &= \left\{ 4,\ 6,\ 7, \ 8,\ 9 \right\} & Y \cap Z &= \left\{ 7,\ 8,\ 9 \right\} \\ Y \backslash X &= \left\{ 7,\ 9 \right\} & Z \backslash Y &= \left\{ n \:\vert\: n \in \mathbb{N} \wedge n > 9 \right\} \\
\end{align*}
(iii) \[ (Y \cap Z) \times (X \cap Y) =  \left\{ (7, \ 6),\ (8, \ 6),\ (9, \ 6),\ (7, \ 8),\ (8, \ 8),\ (9, \ 8) \right\} \]
(b)\\
\begin{theorem}
	\[ C \backslash (A \cap B) = (C \backslash A) \cup (C \backslash B) \]
\end{theorem}
\begin{lemma}
Sei $ m \coloneqq (x \in M)$ und  $n \coloneqq (x \in N)$, dann
\[ x \in (M \cup N) \Leftrightarrow m \vee n \qquad x \in (M \cap N) \Leftrightarrow m \wedge n \qquad x  \in (M \backslash N) \Leftrightarrow m \wedge \neg n \]
\end{lemma}
\begin{lemma}
\[ \forall x ((x \in A) \Leftrightarrow (x \in B)) \:\Leftrightarrow\; A = B \]
\end{lemma}

\begin{proof}
	
Nun sei $a \coloneqq (x \in A), \ b \coloneqq (x \in B), \ c \coloneqq (x \in C)$

\begin{align*}
	&x \in (C \backslash (A \cap B)) \Leftrightarrow c \wedge \neg(x \in (A \cap B)) \Leftrightarrow c \wedge \neg(a \wedge b) \\
	&\Leftrightarrow c \wedge (\neg a \vee \neg b) \Leftrightarrow (c \wedge \neg a) \vee (c \wedge \neg b) \Leftrightarrow (x \in (C \backslash A)) \vee (x \in (C \backslash B)) \\
	&\Leftrightarrow x \in ((C \backslash A) \cup (C \backslash B))
\end{align*}
\end{proof}

\textbf{Aufgabe 2}\\
(a)\\
\[ f(X) = \{ 5,\ 6,\ 7,\ 8,\ 9 \} \qquad h^{-1}\left( \{11,\ 12\} \right) = \{11,\ 12\} \qquad h^{-1}(\{1\}) = \{2\} \]
(b)\\
$f$ injektiv, da nach \textsc{Peano} gilt $n+1 = m+1 \Rightarrow n = m\ (n,\ m \in \mathbb{N})$. \\
$f$ nicht surjektiv, da
\[ f(x) > x \ \wedge\ \nexists n \in \mathbb{N}\ (1 > n) \quad \Rightarrow \quad \nexists x \in \mathbb{N}\ (f(x) = 1) \]
$g$ nicht injektiv, da $g(1) = g(2)$ \\
$g$ surjektiv, da mit $m(n) = n+1$ gilt dass $\forall n \in \mathbb{N}\ \exists m \in \mathbb{N}\ (g(m) = n)$
\newpage
(c)
\begin{theorem}
Die Funktion $g : \mathbb{N} \rightarrow \mathbb{N}$ mit
\[ g(n) = \begin{cases} n+1 &\quad 2 \nmid n \\ n-1 &\quad 2 \mid n \\ \end{cases} \]
ist bijektiv.
\end{theorem}
\begin{proof}
Da $2 \mid (n \pm 1) \:\Leftrightarrow\: 2 \nmid n$ gilt
\[ (g(n) \mid 2) \Leftrightarrow (n \nmid 2) \]
womit
\[ g(g(n)) = n \pm 1 \mp 1 = n \]
sodass $g \circ g = \mathrm{id}_M$ erfüllt ist. Nach Lemma \ref{klemma} muss $g$ bijektiv, und somit sowohl injektiv als auch surjektiv sein.
\end{proof}
\begin{lemma}\label{klemma}
Gilt für ein $f:M \rightarrow M$
\begin{equation}\label{def}
	f \circ f = \mathrm{id}_M
\end{equation}
so ist $f$ bijektiv.
\end{lemma}
\begin{proof}
% \subparagraph*{Surjektivität}
Mit $x \coloneqq f(y)$ gilt
\[ \forall y \in M\ \exists x \in M\ (f(x) = y) \]
da nach \eqref{def}
\[ f(x) = f(f(y)) = y \]
gilt, ist $f$ surjektiv.
% \subparagraph*{Injektivität}
Wenn
\begin{equation}\label{gegenbeh}
	\exists m,\ n \in M\ (m \neq n \wedge f(m) = f(n))
\end{equation}
, dann muss nach \eqref{def}
\begin{align*}
	f(m) &= f(n) \\
	f(f(m)) &= f(f(n)) \\
	m &= n \\ % TODO Widerspruch \\
\end{align*}
Somit gilt das Gegenteil von \eqref{gegenbeh}, sodass $f$ injektiv ist.
\end{proof}
(c)
\begin{align*}
(f \circ h)(n) &= \begin{cases}(n+1)+1 &\quad 2  \nmid (n+1) \\ n+1-1 &\quad 2 \mid (n+1) \end{cases} & (f \circ g)(n) &= \begin{cases} f(g(1)) = 2 &\quad n = 1\\ n-1+1 &\quad n > 1\end{cases}\\
&= \begin{cases} n + 2 &\quad 2 \mid n\\n &\quad 2 \nmid n  \end{cases} & &= \begin{cases}2 &\quad n=1\\n &\quad n > 1\end{cases}
\end{align*}
\newpage
\textbf{Aufgabe 3}\\

\newcommand{\defimpl}[1]{\stackrel{\text{Def.}\;#1}{\Longrightarrow}}
(a)\\
Sei $y \in f(A \cap B)$\\
$\defimpl{f,\;y\;=\;f(x)}$ Es gibt ein $x \in A \cap B$ mit $y = f(x)$ \\	
$ \defimpl{\cap} x \in A$ und $x \in B$\\
$\stackrel{\text{Def.}\;f}{\Longrightarrow} y \in f(A)\ \text{und}\ y \in f(B)$\\
$ \defimpl{\cap} y \in f(A) \cap f(B)$ \\
(b)\\
Sei $x \in f^{-1}(C) \cup f^{-1}(D)$ \\
$\defimpl{\cup} x \in f^{-1}(C)$ oder $x \in f^{-1}(D)$ \\
Fall 1: $x \in f^{-1}(C) \defimpl{f^{-1}}$ Es gibt $y \in C$ mit $f(x) = y \Rightarrow y \in C \cup D \defimpl{f(x) = y} x \in f^{-1}(C \cup D)$ \\
Fall 2: $x \in f^{-1}(D) \defimpl{f^{-1}}$ Es gibt $y \in D$ mit $f(x) = y \Rightarrow y \in C \cup D \defimpl{f(x) = y} x \in f^{-1}(C \cup D)$
\end{document}