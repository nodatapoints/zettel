\documentclass[a4paper, 12pt]{scrartcl}

\usepackage[utf8]{inputenc}
\usepackage[T1]{fontenc}
\usepackage[ngerman]{babel}

\usepackage{amssymb}
\usepackage{amsmath}
\usepackage{framed}
\usepackage{float}
\usepackage{mathtools}

\usepackage{tikz}
\usepackage{chngcntr}

\usepackage{amsthm}
\usepackage{thmtools}

\usepackage[left=2cm, right=2cm, top=2cm]{geometry}

\allowdisplaybreaks

\setlength{\parindent}{0pt}

\declaretheoremstyle[%
  spaceabove=0,%
  spacebelow=6pt,%
  headfont=\normalfont\itshape,%
  postheadspace=1em,%
  headpunct={}
]{mystyle}

\declaretheorem[name={Behauptung}, style=mystyle, unnumbered]{theorem}
\declaretheorem[name={Lemma}, style=mystyle]{lemma}
\declaretheorem[name={Voraussetzung}, style=mystyle, unnumbered]{precondition}
\let\proof\oldproof
\declaretheorem[name={Beweis}, style=mystyle, qed=\qedsymbol, unnumbered]{proof}

\newcounter{taski}
\newcounter{taskii}[taski]
\newcounter{taskiii}[taskii]

\newcommand{\task}{\stepcounter{taski}\textbf{Aufgabe \arabic{taski}}\\}
\newcommand{\ttask}{\stepcounter{taskii}\textbf{(\alph{taskii})}\par}
\newcommand{\tttask}{\stepcounter{taskiii}\quad(\roman{taskiii})\par}

\begin{document}
\hfill Ellen, Kamal
\begin{center}
	\bfseries Lineare Algebra Blatt 1
\end{center}
\task
\ttask
\tttask
\[ \mathcal{P}(X) = \left\{ \emptyset,
\ \left\{  8 \right\},
\ \left\{  6 \right\},
\ \left\{  6,\ 8 \right\},
\ \left\{ 4 \right\},
\ \left\{ 4,\ 8 \right\},
\ \left\{ 4,\ 6 \right\},
\ \left\{ 4,\ 6,\ 8 \right\}
 \right\} \]
\tttask
\begin{align*}
	X \cup Y &= \left\{ 4,\ 6,\ 7, \ 8,\ 9 \right\} & Y \cap Z &= \left\{ 7,\ 8,\ 9 \right\} \\ Y \backslash X &= \left\{ 7,\ 9 \right\} & Z \backslash Y &= \left\{ n \:\vert\: n \in \mathbb{N} \wedge n > 9 \right\}
\end{align*}
\tttask \[ (Y \cap Z) \times (X \cap Y) =  \left\{ (7, \ 6),\ (8, \ 6),\ (9, \ 6),\ (7, \ 8),\ (8, \ 8),\ (9, \ 8) \right\} \]
\ttask
\begin{theorem}
Für alle Mengen $A,\ B,\ C$ gilt
\[ C \backslash (A \cap B) = (C \backslash A) \cup (C \backslash B) \]
\end{theorem}
\begin{precondition}
Seien $M,\ N$ Mengen und sei $ m \coloneqq (x \in M)\ ,\ n \coloneqq (x \in N)$\\ für ein beliebiges $x$, dann
\begin{gather*}
x \in (M \cup N) \Leftrightarrow m \vee n \qquad x \in (M \cap N) \Leftrightarrow m \wedge n \qquad x  \in (M \backslash N) \Leftrightarrow m \wedge \neg n \\
\forall x (m \Leftrightarrow n) \:\Leftrightarrow\; M = N
\end{gather*}
\end{precondition}
\begin{proof}
Nun sei $a \coloneqq (x \in A), \ b \coloneqq (x \in B), \ c \coloneqq (x \in C)$
\begin{align*}
	&x \in (C \backslash (A \cap B)) \Leftrightarrow c \wedge \neg(a \wedge b) \Leftrightarrow c \wedge (\neg a \vee \neg b) \\
	&\Leftrightarrow (c \wedge \neg a) \vee (c \wedge \neg b) \Leftrightarrow (x \in (C \backslash A)) \vee (x \in (C \backslash B)) \\
	&\Leftrightarrow x \in ((C \backslash A) \cup (C \backslash B))
\end{align*}
\end{proof}
\task
\ttask
\[ f(X) = \{ 5,\ 6,\ 7,\ 8,\ 9 \} \qquad h^{-1}\left( \{11,\ 12\} \right) = \{11,\ 12\} \qquad h^{-1}(\{1\}) = \{2\} \]
\ttask
$f$ injektiv, da nach \textsc{Peano} gilt $n+1 = m+1 \Rightarrow n = m\ (n,\ m \in \mathbb{N})$. \\
$f$ nicht surjektiv, da
\[ f(x) > x \ \wedge\ \nexists n \in \mathbb{N}\ (1 > n) \quad \Rightarrow \quad \nexists x \in \mathbb{N}\ (f(x) = 1) \]
$g$ nicht injektiv, da $g(1) = g(2)$ \\
$g$ surjektiv, da mit $m = n+1$ gilt dass $\forall n \in \mathbb{N}\ \exists m \in \mathbb{N}\ (g(m) = n)$
\newpage
\begin{theorem}
Die Funktion $g : \mathbb{N} \rightarrow \mathbb{N}$ mit
\[ g(n) = \begin{cases} n+1 &\quad 2 \nmid n \\ n-1 &\quad 2 \mid n \\ \end{cases} \]
ist bijektiv.
\end{theorem}
\begin{proof}
Da $2 \mid (n \pm 1) \:\Leftrightarrow\: 2 \nmid n$ gilt
\[ (2 \mid g(n)) \Leftrightarrow (2 \nmid n) \]
womit
\[ g(g(n)) = n \pm 1 \mp 1 = n \]
sodass $g \circ g = \mathrm{id}_\mathbb{N}$ erfüllt ist. Damit existiert ein $g^{-1} = g$. Demnach ist $g$ bijektiv.
% Mit $x \coloneqq f(y)$ gilt
% \[ \forall y \in M\ \exists x \in M\ (f(x) = y) \]
% da nach \eqref{def}
% \[ f(x) = f(f(y)) = y \]
% gilt, ist $f$ surjektiv.
% Wenn
% \begin{equation}\label{gegenbeh}
% 	\exists m,\ n \in M\ (m \neq n \wedge f(m) = f(n))
% \end{equation}
% , dann muss nach \eqref{def}
% \begin{align*}
% 	f(m) &= f(n) \\
% 	f(f(m)) &= f(f(n)) \\
% 	m &= n \\ % TODO Widerspruch \\
% \end{align*}
% Somit gilt das Gegenteil von \eqref{gegenbeh}, sodass $f$ injektiv ist.
\end{proof}
\ttask
\begin{align*}
(h \circ f)(n) &= \begin{cases}(n+1)+1 &\quad 2  \nmid (n+1) \\ n+1-1 &\quad 2 \mid (n+1) \end{cases} & (f \circ g)(n) &= \begin{cases} f(g(1)) = 2 &\quad n = 1\\ n-1+1 &\quad n > 1\end{cases}\\
&= \begin{cases} n + 2 &\quad 2 \mid n\\n &\quad 2 \nmid n  \end{cases} & &= \begin{cases}2 &\quad n=1\\n &\quad n > 1\end{cases}
\end{align*}
\task
Es seien $X,\ Y,\ A,\ B,\ C,\ D$ Mengen mit $A,\ B \subseteq X\ ,\ C,\ D \subseteq Y$ und $f:X \rightarrow Y$ eine Abbildung.

\ttask
\newcommand{\defimpl}[1]{\stackrel{\text{Def.}\;#1}{\Longrightarrow}}
Sei $y \in f(A \cap B)$\\
$\defimpl{f,\;y\;=\;f(x)}$ Es gibt ein $x \in A \cap B$ mit $y = f(x)$ \\	
$ \defimpl{\cap} x \in A$ und $x \in B$\\
$\stackrel{\text{Def.}\;f}{\Longrightarrow} y \in f(A)\ \text{und}\ y \in f(B)$\\
$ \defimpl{\cap} y \in f(A) \cap f(B)$\\

\ttask
Sei $x \in f^{-1}(C) \cup f^{-1}(D)$ \\
$\defimpl{\cup} x \in f^{-1}(C)$ oder $x \in f^{-1}(D)$ \\
Fall 1: $x \in f^{-1}(C) \defimpl{f^{-1}}$ Es gibt $y \in C$ mit $f(x) = y \Rightarrow y \in C \cup D \defimpl{f(x) = y} x \in f^{-1}(C \cup D)$ \\
Fall 2: $x \in f^{-1}(D) \defimpl{f^{-1}}$ Es gibt $y \in D$ mit $f(x) = y \Rightarrow y \in C \cup D \defimpl{f(x) = y} x \in f^{-1}(C \cup D)$ \\

\newpage
\ttask
\begin{theorem}
	\[ f^{-1}(C \cap D) = f^{-1}(C) \cap f^{-1}(D) \]
\end{theorem}
\begin{precondition}
	\[ x \in f^{-1}(M) \Leftrightarrow f(x) \in M \tag*{$M \subseteq Y$}  \]
\end{precondition}
\begin{proof}
\begin{align*}
	x \in f^{-1}(C \cap D) &\Leftrightarrow f(x) \in C \cap D \Leftrightarrow f(x) \in C \wedge f(x) \in D \\
	&\Leftrightarrow f^{-1}(C) \cap f^{-1}(D)
\end{align*}
\end{proof}
Es gilt keine Gleichheit für (a), da mit $A = \{1,\ 2\}\ ,\ B = \{ -1,\ 2\}$ und $f:\mathbb{R} \rightarrow \mathbb{R}\ ,\ x \mapsto x^2$
\[ f(A \cap B) = f(\{2\}) = \{4\} \qquad f(A) \cap f(B) = \{ 1,\ 4 \} \cap \{1,\ 4\} = \{1,\ 4\} \]
\task
Seien $A,\ B,\ C$ Mengen und $f:A \rightarrow B\ ,\ g:B \rightarrow C$ Abbildungen.

\ttask
Ist $g \circ f$ surjektiv, so muss $f$ nicht notwendigerweise surjektiv sein. Sei $A = B = C = \mathbb{R}$ und $f(x) = \arctan x\ ,\ g(x) = \tan x $
\[ \arctan(\mathbb{R}) = \left( -\tfrac{\tau}{4};\tfrac{\tau}{4} \right) \qquad f \circ g = \mathrm{id}_\mathbb{R}  \]
So ist $f$ nicht surjektiv, obwohl $g \circ f$ bijektiv und somit auch surjektiv ist.

\ttask
\begin{theorem}
\[ g \circ f\ \text{surjektiv} \Rightarrow g\ \text{surjektiv} \]
\end{theorem}
\begin{proof}
Ist $g \circ f$ surjektiv, dann muss
\begin{align*}
	&\forall y \in C\ \exists x \in A\ (g(f(x)) = y)
\intertext{Mit $b = f(x)$ , $b \in B$ gilt}
	&\forall y \in C\ \exists b \in B\ (g(b) = y)
\end{align*}
sodass $g$ surjektiv ist.
\end{proof}
\end{document}