\documentclass[a4paper, 12pt]{scrartcl}

\usepackage[utf8]{inputenc}
\usepackage[T1]{fontenc}
\usepackage[ngerman]{babel}

\usepackage{amssymb}
\usepackage{amsmath}
\usepackage{framed}
\usepackage{float}
\usepackage{mathtools}
\usepackage{marvosym}

\usepackage{tikz}
\usepackage{chngcntr}

\usepackage{amsthm}
\usepackage{thmtools}

\usepackage[left=2cm, right=2cm, top=2cm]{geometry}

\allowdisplaybreaks

\setlength{\parindent}{0pt}

\declaretheoremstyle[%
  spaceabove=0,%
  spacebelow=6pt,%
  headfont=\normalfont\itshape,%
  postheadspace=1em,%
  headpunct={}
]{mystyle}

\declaretheorem[name={Behauptung}, style=mystyle, unnumbered]{theorem}
\declaretheorem[name={Lemma}, style=mystyle]{lemma}
\declaretheorem[name={Voraussetzung}, style=mystyle, unnumbered]{precondition}
\let\proof\oldproof
\declaretheorem[name={Beweis}, style=mystyle, qed=\qedsymbol, unnumbered]{proof}

\newcounter{taski}
\newcounter{taskii}[taski]
\newcounter{taskiii}[taskii]

\newcommand{\task}{\stepcounter{taski}\textbf{Aufgabe \arabic{taski}}\\}
\newcommand{\ttask}{\stepcounter{taskii}\textbf{(\alph{taskii})}}
\newcommand{\tttask}{\stepcounter{taskiii}\quad(\roman{taskiii})}

\newcommand{\defimpl}[1]{\stackrel{\text{Def.}\;#1}{\Longrightarrow}}
\newcommand{\defImpl}[1]{\stackrel{\text{Def.}\;#1}{\Longleftrightarrow}}
\newcommand{\txtimpl}[1]{\stackrel{\text{#1}}{\Longrightarrow}}
\newcommand{\txtImpl}[1]{\stackrel{\text{#1}}{\Longleftrightarrow}}

\setcounter{taski}{13}

\begin{document}
\hfill \textit{Ellen, Kamal}\\
\task

\ttask
\begin{gather*}
	U_1 = \mathrm{Lin}\left( \left\{ \begin{pmatrix} 1 \\ 2 \\ 1 \end{pmatrix},\ \begin{pmatrix} -2 \\ 1 \\ 3 \end{pmatrix} \right\} \right) \qquad
	U_2 = \mathrm{Lin}\left( \left\{ \begin{pmatrix} 2 \\ -1 \\ 0 \end{pmatrix},\ \begin{pmatrix} 0 \\ 3 \\ 2 \end{pmatrix} \right\} \right) \qquad
	U_3  = \mathrm{Lin}\left( \left\{ \begin{pmatrix} 1 \\ 1 \\ 2 \end{pmatrix} \right\} \right)
\end{gather*}
\ttask
\begin{gather*}
	\begin{pmatrix} 1 \\ 4 \\ 3 \end{pmatrix} = \frac{9}{5}\begin{pmatrix} 1 \\ 2 \\ 1 \end{pmatrix} + \frac{2}{5} \begin{pmatrix} -2 \\ 1 \\ 3 \end{pmatrix} = \frac{1}{2}\begin{pmatrix} 2 \\ -1 \\ 0 \end{pmatrix} + \frac{3}{2}\begin{pmatrix} 0 \\ 3 \\ 2 \end{pmatrix}
\end{gather*}
\setcounter{taski}{14}
\task
\setcounter{taskii}{1}
\ttask
\begin{theorem}
\[ \mathrm{Lin}(\{v_0,\ \dots,\ v_n\}) = \mathrm{Lin}(\{v_0,\ v_1 - v_0,\ \dots,\ v_r - v_0\}) \]
\end{theorem}
\begin{proof}\ \\
1. Richtung \quad $\mathrm{Lin}(\{v_0,\ \dots,\ v_n\}) \supseteq \mathrm{Lin}(\{v_0,\ v_1 - v_0,\ \dots,\ v_r - v_0\})$
\begin{gather}
	\begin{gathered}\label{subsetbed}
	\defImpl{\subseteq} \forall v \in \mathrm{Lin}(\{v_0,\ v_1 - v_0,\ \dots,\ v_r - v_0\}) : v \in \mathrm{Lin}(\{v_0,\ \dots,\ v_n\})
	\end{gathered}\\
	\begin{gathered}\label{lindiff}
	\forall v \in V : v \in \mathrm{Lin}(\{v_0,\ v_1 - v_0,\ \dots,\ v_r - v_0\}) \txtImpl{Def. Lin} \exists a_0,\ \dots,\ a_r \in \mathbb{R}: \\
		a_0v_0 + a_1(v_1-v_0) + \dots + a_r(v_r - v_0) = v
	\end{gathered}\\\nonumber\\
	\begin{gathered}\label{lin}
	\forall v \in V : v \in \mathrm{Lin}(\{v_0,\ \dots,\ v_n\}) \txtImpl{Def. Lin} \exists a_0',\ \dots,\ a_r' \in \mathbb{R}: \\
		a_0'v_0 + a_1'v_1 + \dots + a_r'v_r 
	\end{gathered}
\end{gather}
Es wird angenommen dass $v \in \mathrm{Lin}(\{v_0,\ v_1 - v_0,\ \dots,\ v_r - v_0\})$
\begin{gather*}
	\txtimpl{\eqref{lindiff}} \exists a_0,\ \dots,\ a_r \in \mathbb{R}: a_0v_0 + a_1(v_1-v_0) + \dots + a_r(v_r - v_0) = v \\
	\text{nach \eqref{subsetbed} z. Z.} \quad v \in \mathrm{Lin}(\{v_0,\ \dots,\ v_n\}) \\
\end{gather*}
Es seien $a_0',\ \dots,\ a_r' \in \mathbb{R}$
\[ a_i' \coloneqq \begin{cases}
	a_0 - a_1 - \dots - a_r &\qquad i = 0 \\
	a_i &\qquad\text{sonst}
\end{cases} \tag{$i \in \mathbb{N}_0 \wedge 0 \leq i \leq r$} \]
sodass
\begin{gather*}
	 a_0v_0 + a_1(v_1-v_0) + \dots + a_r(v_r - v_0) = (a_0 - a_1 - \dots - a_r)v_0 + a_1v_1 + \dots + a_rv_r \\
	 = a_0'v_0 + a_1'v_1 + \dots + a_r'v_r \\
	 \txtimpl{\eqref{lin}} v \in \mathrm{Lin}(\{v_0,\ \dots,\ v_n\})
\end{gather*}
2. Richtung \quad $\mathrm{Lin}(\{v_0,\ \dots,\ v_n\}) \subseteq \mathrm{Lin}(\{v_0,\ v_1 - v_0,\ \dots,\ v_r - v_0\})$
\begin{gather}\label{subsetbed2}
	\defImpl{\subseteq} \forall v \in \mathrm{Lin}(\{v_0,\ \dots,\ v_n\}) : v \in \mathrm{Lin}(\{v_0,\ v_1 - v_0,\ \dots,\ v_r - v_0\})
\end{gather}
Es wird angenommen dass $v \in \mathrm{Lin}(\{v_0,\ \dots,\ v_n\})$
\begin{gather*}
	\txtimpl{\eqref{lin}} \exists a_0,\ \dots,\ a_r \in \mathbb{R}: a_0v_0 + a_1v_1 + \dots + a_rv_r = v\\
	\text{nach \eqref{subsetbed2} z. Z.} \quad v \in \mathrm{Lin}(\{v_0,\ v_1 - v_0,\ \dots,\ v_r - v_0\}) \\
\end{gather*}
Es seien $a_0',\ \dots,\ a_r' \in \mathbb{R}$
\[ a_i' \coloneqq \begin{cases}
	a_0 + a_1 + \dots + a_r &\qquad i = 0 \\
	a_i &\qquad\text{sonst}
\end{cases} \tag{$i \in \mathbb{N}_0 \wedge 0 \leq i \leq r$} \]
sodass
\begin{gather*}
	 a_0v_0 + a_1v_1 + \dots + a_rv_r = (a_0 + a_1 + \dots + a_r)v_0 + a_1(v_1 - v_0) + \dots + a_r(v_r - v_0) \\
	 = a_0'v_0 + a_1'v_1 + \dots + a_r'v_r \\
	 \txtimpl{\eqref{lindiff}} v \in \mathrm{Lin}(\{v_0,\ v_1 - v_0,\ \dots,\ v_r - v_0\})
\end{gather*}
\end{proof}
\ttask

Es gilt nicht. Man betrachte den Vektorraum $(V,\ +) = (\mathbb{R},\ +)$ über den Körper $K = \mathbb{R}$ mit üblicher Multiplikation.
\[ \mathrm{Lin}(\{1\}) = \mathrm{Lin}(\{2\}) = \mathrm{Lin}(\{1\}) \cap \mathrm{Lin}(\{2\}) = \mathbb{R} \neq \{0\} = \mathrm{Lin}(\emptyset) = \mathrm{Lin}(\{1\}\cap\{2\}) \]
\task
\setcounter{taskii}{1}
\ttask

\begin{theorem}
\[ \nexists M \subseteq U_1 : M\ \text{endlich}\ \wedge\ \mathrm{Lin}(M) = U_1 \]
\end{theorem}
\begin{proof}
durch Widerspruch

Es sei ein endliches $M \subseteq U_1$ mit $\mathrm{Lin}(M) = U_1$. Da nach Def. $U_1$
\[ \forall (F_n)_{n \in \mathbb{N}} \in M\ \exists s \in \mathbb{N}\ \forall n \in \mathbb{N} : n > s \Rightarrow F_n = 0 \]
existiert eine Menge
\[ S = \left\{ s \in \mathbb{N} \mid F \in M \right\} \]
sodass
\begin{equation}\label{schrank}
	\forall (F_n)_{n \in \mathbb{N}} \in M\ \exists s \in S\ \forall n \in \mathbb{N} : n > s \Rightarrow F_n = 0
\end{equation}
\newpage
Nach Lemma \ref{max} gibt es ein $z \in \mathbb{N}$ sodass
\begin{gather*}
	\forall s \in S : s < z \\
	\Rightarrow \forall (F_n)_{n \in \mathbb{N}} \in M\ \forall n \in \mathbb{N} : n > z \Rightarrow F_n = 0 \tag*{nach \eqref{schrank}} \\
	\Rightarrow \forall (F_n)_{n \in \mathbb{N}} \in M : F_{z+1} = 0
\end{gather*}
So ist für jede beliebige Reihe $(\xi_n)_{n \in \mathbb{N}} \in \mathrm{Lin}(M)$
\[ \xi_{z+1} = \sum_{F \in M} a(F)F_{z+1} = 0 \tag{$a:M \rightarrow \mathbb{R}$} \]
Sei $(\xi_n)_{n \in \mathbb{N}}$
\[ \xi_n \coloneqq \begin{cases}
	1 &\qquad n = z+1 \\
	0 &\qquad \text{sonst}
\end{cases} \]
$(\xi_n)_{n \in \mathbb{N}} \in U_1$ da nach Konstuktion $\forall n \in N : n > z+1 \Rightarrow \xi_n = 0$

Jedoch ist $\xi_{n+1} = 1 \neq 0$, womit $(\xi_n)_{n \in \mathbb{N}} \notin \mathrm{Lin}(M)$
\[ \Rightarrow \mathrm{Lin}(M) \neq U_1 \]
\end{proof}

\begin{lemma}\label{max}
Sei $N \subseteq \mathbb{N}$ eine endliche Menge
	\[ \exists s \in \mathbb{N} : \forall n \in N : n < s \]
\end{lemma}
\begin{proof}
durch vollständige Induktion
\paragraph*{Induktionsanfang} $N \eqqcolon \{e\} \subseteq \mathbb{N}$ , $N \sim A_1$ \hfill ($e \in \mathbb{N}$)
\[ \forall i \in N : i < e+1 \]
\paragraph*{Induktionsbehauptung} $\forall N \subseteq \mathbb{N}\:\wedge N \sim A_n\ \exists s \in \mathbb{N} : \forall n \in N : n < s$

Es sei ein $N' = N \cup \{e\}$ mit $e \in \mathbb{N} \:\wedge\:e \notin N$. Nach Konstruktion gilt $N' \sim A_{n+1}\:\wedge\:N' \subseteq \mathbb{N}$.
Es sei $s' \in \mathbb{N}$ definiert als
\[ s' \coloneqq \begin{cases}
	e+1 &\qquad e > s\\
	s &\qquad \text{sonst}
\end{cases} \]
Nach Konstruktion gilt
\[ s' > e \ \wedge\ \forall n \in N : n < s \leq s'\]
sodass
\[ \forall n \in N' : n < s' \]
\end{proof}
\end{document}