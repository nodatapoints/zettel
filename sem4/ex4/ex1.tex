\documentclass[a4paper, 11pt]{scrartcl}

\usepackage[utf8]{inputenc}
\usepackage[T1]{fontenc}

\usepackage{amsmath}
\usepackage{amssymb}

\usepackage{physics}
\usepackage{siunitx}
\usepackage{graphicx}
\usepackage{float}
\usepackage[left=2cm, right=2cm, top=1cm]{geometry}

\let\blattno1

\newcounter{taski}
\newcounter{taskii}[taski]

\newcommand{\task}{\stepcounter{taski}\textbf{\blattno.\arabic{taski})}\\}
\newcommand{\ttask}{\stepcounter{taskii}\par\alph{taskii})\quad}

\def\MeV{\mega\electronvolt}
\def\GeV{\giga\electronvolt}

\let\vec\textbf
\setlength\parindent{0pt}

\begin{document}

\parbox{.25\textwidth}{\today}
\parbox{.5\textwidth}{\centering\bfseries PEP 4 Exercise Sheet \blattno}
\parbox{.25\textwidth}{\hfill Kamal Abdellatif}
\vspace{1em}

\task
\begin{align*}
    \Delta E &= E_2 - E_1
    = m_e(\gamma_2 - \gamma_1)
    = m_e\qty(\frac1{\sqrt{1-\beta_2^2}} - \frac1{\sqrt{1-\beta_1^2}}) \\
    &= \SI{0.511}{\MeV} \cdot \qty(\frac1{\sqrt{1-0.999^2}} - \frac1{\sqrt{1-0.990^2}})
    \approx \SI{7.8}\MeV
\end{align*}
\task \ttask
\begin{align*}
    E_{cm}^2 &= \qty(\vec{k}_1 + \vec{k}_2)^2
    = \vec{k}_1^2 + \vec{k}_2^2 + 2\vec{k}_1\vec{k}_2
    = 2m_e^2 + 2\qty(-\overbrace{\sqrt{E_1^2-m_e^2}}^{p_1}\cdot\overbrace{\qty(-\sqrt{E_2^2-m_e^2})}^{p_2} + E_1E_2) \\
    &\approx 4E_1E_2 \qquad\qquad (m_e \ll E_1, E_2) \\
    E_{cm} &= 2\sqrt{E_1E_2} \approx \SI{10.6}\GeV
\end{align*}
\ttask
\begin{gather*}
\gamma\mqty(1&-\beta\\-\beta&1)\mqty(p_{cm}\\E_{cm}) = \mqty(0\\E_{cm}^*)
\qquad,\qquad
p_{cm} - \beta E_{cm} \stackrel!= 0 ~\qcomma \beta = \frac{p_{cm}}{E_{cm}} \\
E_i^2 = m_e^2 + p_i^2 ~\qcomma p_i = \pm \sqrt{E_i^2-m_e^2} \approx \pm E_i \\
\beta = \frac{p_1+p_2}{E_1+E_2} = \frac{E_1-E_2}{E_1+E_2} = 0.49
\quad,\quad v = 0.49 c
\end{gather*}
\task \ttask In the center of mass system, the mass and absolute value of the momentum of both electron and positron are the same. Therefore the total energy is given as
\[ E_{cm}^* = 2\sqrt{m_e^2+|p*|^2} \approx 2|p*| = \SI{1.02}\GeV  \]
Before the split, the particle A rests in the COM system. Because energy is conserved, the resting mass $m_A$ of A is equal to the total energy in the COM system.
\[ m_A = 1.02\:\frac{\si\GeV}{c^2} \]
\ttask Transforming the momentum vector from the COM to the laboratory system yields
\[ 
    \mqty(p_x\\p_y\\E) =
    \mqty(
        \gamma&&\gamma\beta\\
        &1& \\
        \gamma\beta&&\gamma
        )
    \mqty(p_x^*\\p_y^*\\E^*) 
\]
As stated, $p_x^* = p_e\cos\theta$. Since $m_e \ll p_e = 0.51\:\si{GeV}/c$, one can assume $E^* \approx |p^*| = p_e$.
\[ E = \gamma(\beta p_x^* + E^*) = \gamma(1+\beta\cos\theta)p_e \]
From $m_A$ and $p_A$ one can determine $\beta$ as
\[ \beta = \frac{p_A}{E_A} = \frac{p_A}{\sqrt{m_A^2+p_A^2}} = \frac1{\sqrt{1+\frac{m_A^2}{p_A^2}}} \approx 0.995 \]
\begin{figure}[H]
    \centering
    \includegraphics[width=.7\textwidth]{Esplit.pdf}
    \caption{$E/p_e$ versus $\theta$ in polar coordinates}
\end{figure}
\begin{figure}[H]
    \centering
    \includegraphics[width=.7\textwidth]{multiple.pdf}
    \caption{Polar energy dependence for different $\beta$}
\end{figure}
Note that $\beta = 0$ creates a circle, while $\beta \rightarrow 1$ approaches a cardioid.

\task One can calculate the center-of-mass energy $E_{cm}$ like before
\[
    E_{cm}^2 = \qty(\vec{k}_1 + \vec{k}_2)^2
    = \vec{k}_1^2 + \vec{k}_2^2 + 2\vec{k}_1\vec{k}_2
    = 2m_p^2 + 2\qty(-0\cdot p_1 + E_1m_p)
    = 2m_p(E_1+m_p)
\]
this to be at least the resting energy of the reaction products, i. e. $4m_p$.
\[ (4m_p)^2 = 16m_p^2 \stackrel!= 2m_p(E_1+m_p) \quad,\quad
   E_1 = 7m_p
\]
The kinetic energy is therefore $E_{kin} = E_1 - m_p = 6m_p$.
\end{document}
