\documentclass[a4paper, 12pt]{scrartcl}

\usepackage[utf8]{inputenc}
\usepackage[T1]{fontenc}
\usepackage[ngerman]{babel}

\usepackage{amssymb}
\usepackage{amsmath}
\usepackage{framed}
\usepackage{float}
\usepackage{mathtools}

\usepackage{tikz}

\usepackage{amsthm}
\usepackage{thmtools}
\usepackage{marvosym}

\usepackage[left=2cm, right=2cm, top=1cm]{geometry}

\allowdisplaybreaks
\setkomafont{paragraph}{\normalfont\itshape}

\setlength{\parindent}{0pt}

\declaretheoremstyle[%
  spaceabove=0,%
  spacebelow=6pt,%
  headfont=\normalfont\itshape,%
  postheadspace=1em,%
  headpunct={}
]{mystyle}

\declaretheorem[name={Behauptung}, style=mystyle, unnumbered]{theorem}
\declaretheorem[name={Lemma}, style=mystyle]{lemma}
\declaretheorem[name={Voraussetzung}, style=mystyle, unnumbered]{precondition}
\let\proof\oldproof
\declaretheorem[name={Beweis}, style=mystyle, qed=\qedsymbol, unnumbered]{proof}

\newcounter{taski}
\newcounter{taskii}[taski]
\newcounter{taskiii}[taskii]


\newcommand{\task}{\stepcounter{taski}\textbf{Aufgabe \arabic{taski}}\\}
\newcommand{\ttask}{\stepcounter{taskii}\textbf{(\alph{taskii})}\par}
\newcommand{\tttask}{\stepcounter{taskiii}\quad(\roman{taskiii})\par}

\newcommand{\defimpl}[1]{\stackrel{\text{Def.}\;#1}{\Longrightarrow}}
\newcommand{\defImpl}[1]{\stackrel{\text{Def.}\;#1}{\Longleftrightarrow}}
\newcommand{\txtimpl}[1]{\stackrel{\text{#1}}{\Longrightarrow}}
\newcommand{\txtImpl}[1]{\stackrel{\text{#1}}{\Longleftrightarrow}}

\DeclareMathOperator*{\argmin}{arg\,min}
\begin{document}
\begin{flushright}
	Kamal\\
	Maximilian Neumann
\end{flushright}
\begin{center}
	\bfseries Analysis I Blatt 3
\end{center}
\task
\begin{theorem}
	Für alle $n$ gibt es genau $n!$ bijektive Abbildungen in $f:A_n \rightarrow A_n$
\end{theorem}
\begin{proof}
Es sei $\pi_n \coloneqq \left\{ f:A_n \rightarrow A_n \mid f\ \text{bijektiv} \right\}$
\paragraph*{Induktionsanfang} $n = 1$ , $\pi_1 = \{ \mathrm{id}_{A_1}\}$
\paragraph*{Induktionsbehauptung} $|\pi_n| = n!$
\paragraph*{Induktionsschritt} z. z. $|\pi_{n+1}| = (n+1)!$

Es sei $\mathrm{Ins}: \pi_n \times A_{n+1} \rightarrow \pi_{n+1}$ definiert als
\[ \mathrm{Ins}(\varphi,\ k)(i) \coloneqq \begin{cases}
	n+1 &\qquad i = k \\
	\varphi(k) &\qquad i = n+1 \\
	\varphi(i) &\qquad \text{sonst}
\end{cases} \]
Jeder Wert von $\mathrm{Ins}(\varphi,\ k)$ ist nach Definition bijektiv und bildet $A_{n+1}$ auf $A_{n+1}$ ab, und ist somit in $\pi_{n+1}$ enthalten.

Nach Lemma \ref{lemma1} und Lemma \ref{lemma2} ist $\mathrm{Ins}$ bijektiv, womit
\[ |\pi_{n+1}| = |\pi_n \times A_{n+1}| \stackrel{\text{Lemma \ref{cart}}}{=} |\pi_n| \cdot |A_{n+1}| = n! \cdot (n+1) = (n+1)! \]
\end{proof}
\begin{lemma}\label{lemma1}
$\mathrm{Ins}$ ist injektiv.
\end{lemma}
\begin{proof}
\[ \mathrm{Ins}\ \text{injektiv} \txtImpl{Def. Inj.} \left( \forall\varphi,\ \varphi' \in \pi_n\ \forall k,\ k' \in A_{n+1} : \varphi \neq \varphi' \vee k \neq k' \Rightarrow \mathrm{Ins}(\varphi,\ k) \neq \mathrm{Ins}(\varphi',\ k') \right) \]
Fall 1: $\varphi \neq \varphi' \Leftrightarrow \exists i \in A_n : \varphi(i) \neq \varphi'(i)$

Wenn $k = k'$, so trifft Fall 2 zu, und es ist nichts zu zeigen $
\curvearrowright$ Es sei $k \stackrel{!}{\neq} k'$
\begin{gather*}
	i = k \defimpl{\mathrm{Ins}} \mathrm{Ins}(\varphi,\ k)(i) = n+1 \\
	i = k \stackrel{k \neq k'}{\Longrightarrow} i \neq k' \defimpl{\mathrm{Ins}} \mathrm{Ins}(\varphi',\ k')(i) = \varphi'(i)\\\\
	\defimpl{\varphi'} \varphi'(i) \neq n+1 \Rightarrow \mathrm{Ins}(\varphi',\ k')(i) \neq 
	i = k \defimpl{\mathrm{Ins}} \mathrm{Ins}(\varphi,\ k)(i)
\end{gather*}
Fall 2: $k \neq k'$
\[  \mathrm{Ins}(\varphi,\ k)(n+1) = k \neq k' = \mathrm{Ins}(\varphi',\ k')(n+1) \]
\end{proof}
\newpage
\begin{lemma}\label{lemma2}
$\mathrm{Ins}$ ist surjektiv.
\end{lemma}
\begin{proof}
\[ \Leftrightarrow \forall \varphi' \in \pi_{n+1}\ \exists \varphi \in \pi_n\ \exists k \in A_{n+1} : \varphi' = \mathrm{Ins}(\varphi,\ k)\]
Es sei ein beliebiges $\varphi' \in \pi_{n+1}$ gegeben. Nach Def. $\pi$ folgt die Existenz eines $\varphi'^{-1}$.

Sei $k \coloneqq \varphi'^{-1}(n+1)$ und $\varphi: A_n \rightarrow A_n$
\[ \varphi(i) \coloneqq \begin{cases}
	\varphi'(n+1) &\qquad i = k \\
	\varphi'(i) &\qquad \text{sonst}
\end{cases} \]
$\varphi$ ist nach Konstruktion bijektiv und erfüllt
\[ \varphi' = \mathrm{Ins}(\varphi,\ k) \]
\end{proof}

\begin{lemma}\label{cart}
Für zwei endliche Mengen $M,\ N$ gilt
\[ | M \times N| = |M| \cdot |N| \]
\end{lemma}
\begin{proof}
Sei $m \coloneqq |M|,\ n \coloneqq |N|$. Nach Def. Endlichkeit gibt es bijektive $\alpha,\ \beta$
\[ \alpha: M \rightarrow A_m \qquad \beta : N \rightarrow A_n \]
Es sei nun eine Abbildung $\varphi : M \times N \rightarrow A_{n \cdot m}$
\[ \varphi(a,\ b) \coloneqq n \cdot (\alpha(a) - 1) + \beta(b) \]
Die Größte Zahl wird bei $a = \alpha^{-1}(m),\ b = \beta^{-1}(n)$ erreicht mit
\[ \varphi_{max} = n \cdot (m - 1) + n = n \cdot m\]
\[ \forall a \in M\ \forall b \in N : \alpha(a) \leq m \wedge \beta(b) \leq n \ \Rightarrow\  \varphi(a,\ b) \leq m \cdot n \]
Jede Zahl lässt sich eindeutig in ein Vielfaches von $n$ und einen Rest aufteilen, sodass
\[ i = \varphi(a,\ b) \Leftrightarrow b = \beta^{-1}(i \!\!\!\!\mod n) \ \wedge\ a = \alpha^{-1}\left( 1+\frac{i - i \!\!\!\!\mod n}{n} \right)  \]
Es existiert demnach ein $\varphi^{-1}$, sodass $\varphi$ bijektiv.
\[  M \times N \sim A_{n \cdot m} \defimpl{\#} |M \times N| = |M| \cdot |N| \]
\end{proof}

\task

\ttask
\begin{theorem}
\[ \forall y \in X : \left(  \forall x \in X : x \neq y \Rightarrow x > y \right) \Leftrightarrow \nexists x \in X : x < y\]	
\end{theorem}
\begin{proof}
Für ein beliebiges $y \in X$
\begin{align*}
	&\quad\forall x \in X : x \neq y \Rightarrow x > y \\
	\defImpl{\Rightarrow}	&\quad\forall x \in X : x = y \vee x > y \\
	\defImpl{\geq}&\quad\forall x \in X : x \geq y \\
	\Leftrightarrow&\quad\nexists x \in X : x < y
\end{align*}
\end{proof}
\newpage
\ttask

Keine der beiden Richtungen hält ohne das Anordungsaxiom, da im vorletzten Schritt überhaupt kein logischer Zusammenhang zwischen den Aussagen $x \geq y$ und $x < y$ besteht.
\\

\begin{theorem}
Für eine Folge $(M_n)_{n \in \mathbb{N}}$
	\[ \forall n \in \mathbb{N} : M_n\ \text{abzählbar} \Rightarrow \bigcup_{n \in \mathbb{N}}M_n\quad\text{abzählbar}\]
\end{theorem}
\begin{proof}
\[ \txtimpl{Def. abz.} \forall n \in \mathbb{N}\ \exists \varphi_n:\mathbb{N} \rightarrow M_n\ ,\ \varphi_n\ \text{bijektiv} \]
Es sei
\[ \sigma : \mathbb{N}^2 \rightarrow \bigcup_{n \in \mathbb{N}}M_n\ ,\ (i,\ j) \mapsto \varphi_i(j) \]
$\sigma$ ist nach Konstruktion surjektiv. Da nach Satz 1.5.3 $\mathbb{N}^2$ abzählbar ist, sodass eine Bijektion $\tau : \mathbb{N} \rightarrow \mathbb{N}^2$ existiert.
\[ \Rightarrow \sigma \circ \tau : \mathbb{N} \rightarrow \bigcup_{n \in \mathbb{N}}M_n \quad \text{surjektiv} \]
Nach Lemma \ref{surj} ist $\bigcup_{n \in \mathbb{N}}M_n$ somit abzählbar.
\end{proof}

\begin{lemma}\label{surj}
	Gibt es für eine Menge $M$ eine surjektive Abbildung $\varphi:\mathbb{N} \rightarrow M$, so ist $M$ abzählbar.	
\end{lemma}
\begin{proof}
\renewcommand*{\thefootnote}{\textdagger}
\begin{gather*}
	\txtimpl{Def. Surj.} \forall x \in M \ \exists n \in \mathbb{N} : \varphi(n) = s \\
	\Rightarrow \forall x \in M : \varphi^{-1}(\{x\}) \neq \emptyset \tag*{\footnotemark}
\end{gather*}
\footnotetext{Es handelt sich bei $\varphi^{-1}(\{x\})$ nicht eine Umkehrfunktion, sondern lediglich um das Urbild.}
Es sei nun eine Abbildung $\theta: M \rightarrow \mathbb{N}$, welche ein $x \in M$ auf das kleinste Element in seinem Urbild abbildet. Es existiert ein eindeutiges kleinstes Element, da $\varphi^{-1}(\{x\}) \subseteq \mathbb{N}$.
\[ \theta(x) \coloneqq \min\big(\varphi^{-1}(\{x\})\big) \]
Surjektivität bleibt nach Konstruktion erhalten, und aus der Eindeutigkeit des kleinsten Elements folgt
\[ \forall x,\ y \in M : \min\big(\varphi^{-1}(\{x\})\big) = \min\big(\varphi^{-1}(\{y\})\big) \eqqcolon n \Rightarrow x = \varphi(n) = y\]
sodass $\theta$ injektiv und so bijektiv ist.
\[ \defimpl{\sim} M \sim \mathbb{N} \txtimpl{Def. abz.} M\ \text{abzählbar} \]
\end{proof}
\newpage
\task

\begin{theorem}
	\[ \sqrt{2} + \sqrt{3}\ \text{irrational} \]
\end{theorem}
\begin{proof} durch Widerspruch\\
\begin{align*}
	&&\sqrt{2} + \sqrt{3} \in \mathbb{Q} &\defImpl{\mathbb{Q}} \exists p,\ q \in \mathbb{Z} : \sqrt{2} + \sqrt{3} = \frac{p}{q} \\
	\Rightarrow&& p &= (\sqrt{2} + \sqrt{3})q \\
	\Rightarrow&& p^2 &= (5+2\sqrt{6})q^2 \\
	\Rightarrow&& \sqrt{6} &= \frac{1}{2}\left( \left( \frac{p}{q} \right)^2 - 5 \right)  \in \mathbb{Q} \qquad \text{da $\mathbb{Q}$ abgeschlossen} \\
	\\
	\Rightarrow&& \exists r,\ s \in \mathbb{Z} : \sqrt{6} &= \frac{r}{s} \\
	\Rightarrow&& r^2 &= 6s^2 = 2^13^1s^2 \\ 
\end{align*}
Es gibt eine eindeutige Primzahlfaktorzerlegung für $|r|,\ |s|$. Die Primzahlfaktorzerlegung von $r^2,\ s^2$ besteht nach Potenzgesetzen nur aus geraden Expontenten. Jedoch sind nach obiger Formel die Exponenten von 2 und 3 in $r^2$ jeweils um eins größer als in $s^2$. \Large\Lightning
\end{proof}
\vfill
\begin{flushright}
	\begin{tabular}{c|c|c|c}
	1 & 2 & 3 & $\Sigma$ \\
	\hline
	\hspace*{4em}& \hspace*{4em}& \hspace*{4em}&\hspace*{4em} \\
	&&&\\
	\end{tabular}
\end{flushright}
\end{document}
