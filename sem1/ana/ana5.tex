\documentclass[a4paper, 12pt]{scrartcl}

\usepackage[utf8]{inputenc}
\usepackage[T1]{fontenc}
\usepackage[ngerman]{babel}

\usepackage{amssymb}
\usepackage{amsmath}
\usepackage{framed}
\usepackage{float}
\usepackage{mathtools}

\usepackage{tikz}

\usepackage{amsthm}
\usepackage{thmtools}
\usepackage{marvosym}

\usepackage[left=2cm, right=2cm, top=1cm]{geometry}

\allowdisplaybreaks
\setkomafont{paragraph}{\normalfont\itshape}

\setlength{\parindent}{0pt}

\declaretheoremstyle[%
  spaceabove=0,%
  spacebelow=6pt,%
  headfont=\normalfont\itshape,%
  postheadspace=1em,%
  headpunct={}
]{mystyle}

\declaretheorem[name={Behauptung}, style=mystyle, unnumbered]{theorem}
\declaretheorem[name={Lemma}, style=mystyle]{lemma}
\declaretheorem[name={Voraussetzung}, style=mystyle, unnumbered]{precondition}
\let\proof\oldproof
\declaretheorem[name={Beweis}, style=mystyle, qed=\qedsymbol, unnumbered]{proof}

\newcounter{taski}
\newcounter{taskii}[taski]
\newcounter{taskiii}[taskii]


\newcommand{\task}{\stepcounter{taski}\textbf{Aufgabe \arabic{taski}}\\}
\newcommand{\ttask}{\stepcounter{taskii}\textbf{(\alph{taskii})}\par}
\newcommand{\tttask}{\stepcounter{taskiii}\quad(\roman{taskiii})\par}

\newcommand{\defimpl}[1]{\stackrel{\text{Def.}\;#1}{\Longrightarrow}}
\newcommand{\defImpl}[1]{\stackrel{\text{Def.}\;#1}{\Longleftrightarrow}}
\newcommand{\txtimpl}[1]{\stackrel{\text{#1}}{\Longrightarrow}}
\newcommand{\txtImpl}[1]{\stackrel{\text{#1}}{\Longleftrightarrow}}
\newcommand{\refimpl}[1]{\txtimpl{\eqref{#1}}}
\newcommand{\refImpl}[1]{\txtImpl{\eqref{#1}}}

\DeclareMathOperator*{\argmin}{arg\,min}
\begin{document}
\begin{flushright}
	Kamal\\
	Maximilian Neumann
\end{flushright}
\begin{center}
	\bfseries Analysis I Blatt 5
\end{center}
\task
\ttask

Es sei eine reell-wertige Folge $(a_n)_{n \in \mathbb{N}}$ welche einen Grenzwert $\lambda \in \mathbb{R}$ besitzt.
\[ \lambda = \lim_{n \rightarrow \infty} a_n \]
Weiterhin gebe es eine \emph{Umordnung} $(b_n)_{n \in \mathbb{N}}$, für welche es eine bijektive Abbildung $\varphi : \mathbb{N} \rightarrow \mathbb{N}$ gibt sodass
\[ \forall n \in \mathbb{N}: a_n = b_{\varphi(n)} \]
\begin{theorem}
	\[ \lim_{n \rightarrow \infty} b_n = \lambda \]
\end{theorem}
\begin{proof}
\[ \defimpl{\lim} \forall \varepsilon \in \mathbb{R}_+\ \exists n_0 \in \mathbb{N}\ \forall n \in \mathbb{N} : n_0 < n \Rightarrow |a_n-\lambda| < \varepsilon \]
Es sei ein beliebiges $\varepsilon,\ n_0$ gegeben. Es gilt
\begin{equation} \label{eq:lim}
	\exists s \in \mathbb{N} : \forall m \in \varphi(A_{n_0}) : m < s
\end{equation}
Das Bild $\varphi(A_{n_0})$ ist eine endliche Menge, sodass nach Lemma \ref{max}
\begin{align}
	\exists s \in \mathbb{N}\ &\forall m \in \varphi(A_{n_0}) : m < s \label{eq:schranke}\\
	\refimpl{eq:lim}\qquad &\forall m \in \mathbb{N}: s < m \Rightarrow m \in \mathbb{N}\backslash\varphi(A_{n_0}) \nonumber\\
	\refimpl{eq:schranke}\qquad &\forall m \in \mathbb{N}: s < m \Rightarrow |b_m - \lambda| < \varepsilon \nonumber\\
	\Rightarrow \forall \varepsilon \in \mathbb{R}_+\ \exists s \in \mathbb{N}\ &\forall m \in \mathbb{N} : s < m \Rightarrow |b_m-\lambda| < \varepsilon \nonumber\\
	\defimpl{\lim} \lim_{m \rightarrow \infty} b_m = \lambda \nonumber
\end{align}
\end{proof}
\newpage
\begin{lemma}\label{max}
Für eine endliche Menge $N \subseteq \mathbb{N}$ gilt
	\[ \exists s \in \mathbb{N} : \forall n \in N : n < s \]
\end{lemma}
\begin{proof}
durch vollständige Induktion
\paragraph*{Induktionsanfang} $N \eqqcolon \{e\} \subseteq \mathbb{N}$ , $N \sim A_1$ \hfill ($e \in \mathbb{N}$)
\[ \forall i \in N : i < e+1 \]
\paragraph*{Induktionsbehauptung} $\forall N \subseteq \mathbb{N}\:\wedge N \sim A_n\ \exists s \in \mathbb{N} : \forall n \in N : n < s$

Es sei ein $N' = N \cup \{e\}$ mit $e \in \mathbb{N} \:\wedge\:e \notin N$. Nach Konstruktion gilt $N' \sim A_{n+1}\:\wedge\:N' \subseteq \mathbb{N}$.
Es sei $s' \in \mathbb{N}$ definiert als
\[ s' \coloneqq \begin{cases}
	e+1 &\qquad e > s\\
	s &\qquad \text{sonst}
\end{cases} \]
Nach Konstruktion gilt
\[ s' > e \ \wedge\ \forall n \in N : n < s \leq s'\]
sodass
\[ \forall n \in N' : n < s' \]
\end{proof}
\ttask

Es sei eine reell-wertige Folge $(a_n)_{n \in \mathbb{N}}$ welche einen Grenzwert $\lambda \in \mathbb{R}$ besitzt.
\[ \lambda = \lim_{n \rightarrow \infty} a_n \]
Weiterhin gebe es eine \emph{Teilfolge} $(b_n)_{n \in \mathbb{N}}$, für welche es eine streng monoton steigende injektive Abbildung $(\nu_n)_{n \in \mathbb{N}} : \mathbb{N} \rightarrow \mathbb{N}$ gibt sodass
\begin{equation}\label{eq:nu}
	\forall n \in \mathbb{N}: a_{\nu_n} = b_n
\end{equation}
\begin{theorem}
	\[ \lim_{n \rightarrow \infty} b_n = \lambda \]
\end{theorem}
\begin{proof}
\[ \defimpl{\lim} \forall \varepsilon \in \mathbb{R}_+\ \exists n_0 \in \mathbb{N}\ \forall n \in \mathbb{N} : n_0 < n \Rightarrow |a_n-\lambda| < \varepsilon \]
Es sei ein beliebiges $\varepsilon,\ n_0$ gegeben. Es gilt
\begin{align*}
	\defimpl{\lim}\qquad &\forall \varepsilon \in \mathbb{R}_+\ \exists n_0 \in \mathbb{N}\ \forall n \in \mathbb{N} : n_0 < n \Rightarrow |a_n-\lambda| < \varepsilon \\
	\txtimpl{Lemma \ref{lemma:nu}}\qquad  &\forall \varepsilon \in \mathbb{R}_+\ \exists n_0 \in \mathbb{N}\ \forall n \in \mathbb{N} : n_0 < n \leq \nu_n \Rightarrow |a_{\nu_n}-\lambda| < \varepsilon \\
	\refimpl{eq:nu} \qquad  &\forall \varepsilon \in \mathbb{R}_+\ \exists n_0 \in \mathbb{N}\ \forall n \in \mathbb{N} : n_0 < n \Rightarrow |b_n-\lambda| < \varepsilon \\
	\defimpl{\lim} \qquad & \lim_{n \rightarrow \infty} b_n = \lambda
\end{align*}
\end{proof}
\newpage
\begin{lemma}\label{lemma:nu}
	\[ \forall n \in \mathbb{N}: n \leq \nu_n \]
\end{lemma}
\begin{proof}
durch vollständige Induktion

\paragraph*{Induktionsanfang} $\quad 1 \leq \nu_1\ \text{nach Axiom 1.1 3 ii}$
\paragraph*{Induktionsbehauptung} $\quad n \leq \nu_n$
\paragraph*{Induktionsschritt} Z. z. $\quad n+1 \leq \nu_{n+1}$
\[ 	\txtimpl{Monoton}\quad  n \leq \nu_n < \nu_{n+1} \quad\Rightarrow \quad n < \nu_{n+1} \quad\txtimpl{Satz 1.1 v} \quad n+1 \leq \nu_{n+1} \]
\end{proof}

\task
Sei $(a_n)_{n \in \mathbb{N}}$ o. B. d. A eine positive Nullfolge, sodass $\lim_{n \rightarrow \infty} a_n = 0$
\[ \defimpl{\lim} \forall \varepsilon \in \mathbb{R}_+\ \exists n_0 \in \mathbb{N}\ \forall n \in \mathbb{N} : n_0 < n \Rightarrow |a_n| < \varepsilon \]
Es sei ein beliebiges $\varepsilon,\ n_0$ gegeben. Für ein beliebiges $d \in \mathbb{N}$ gilt
\begin{gather*}
	b_{n_0 + d} = \frac{1}{n_0 + d} \left( n_0b_{n_0} + \sum_{k = n_0}^{n_0+d} a_k \right) \\
	\forall n \in \mathbb{N} : n_0 < n \Rightarrow |a_n| < \varepsilon \quad \Longrightarrow \quad \sum_{k = n_0}^{n_0+d} a_k < d \cdot \varepsilon \\
	b_{n_0 + d} < \frac{n_0b_{n_0} + d \cdot \varepsilon}{n_0 + d} = \left( 1 - \frac{1}{1+\frac{n_0}{d}} \right)b_{n_0} + \frac{1
	}{1+\frac{n_0}{d}}\varepsilon   \\
	b_{n_0 + d} - b_{n_0}= - \frac{b_{n_0} - \varepsilon}{1+\frac{n_0}{d}} \\
\end{gather*}
\emph{Irgentwas mit Cauchy, keinen Plan}

\newpage
\task

\begin{theorem}
\begin{align*}
 	\lim_{n \rightarrow \infty} \frac{3n^3 - 5n^2 + 6(-1)^n}{5n^3 - 3n} &= \lim_{n \rightarrow \infty} \frac{3 - 5n^{-1} + 6(-1)^nn^{-3}}{5 - 3n^{-2}} \\
 	\txtimpl{Satz 3.3.2}\qquad &= \frac{3 - \lim_{n \rightarrow \infty} 5n^{-1} + 6(-1)^nn^{-3}}{5 - \lim_{n \rightarrow \infty} 3n^{-2}} \\
 	\txtimpl{Lemma \ref{lemma:geo}}\qquad &= \frac{3 - 6\lim_{n \rightarrow \infty} (-1)^nn^{-3}}{5} \\
 	\txtimpl{Lemma \ref{lemma:alternating}}\qquad &= \frac{3}{5} \\
 \end{align*}
\end{theorem}

\begin{lemma}\label{lemma:geo}
Sei $n \in \mathbb{N}$
\[ \lim_{k \rightarrow \infty} k^{-n} = 0 \]
\end{lemma}
\begin{proof}
\[ \lim_{n \rightarrow \infty} k^{-n} \stackrel{\text{Satz 3.3.2}}{=} \left( \lim_{n \rightarrow \infty} \frac{1}{k}  \right)^n \stackrel{3.2.1}{=} 0 \]
\end{proof}
\begin{lemma}\label{lemma:alternating}
	\[ \lim_{n \rightarrow \infty} \frac{(-1)^n}{n} = 0 \]
\end{lemma}
\begin{proof}
Es sei ein beliebiges $\varepsilon \in \mathbb{R}_+$ gegeben, so gilt für beliebige $n \in \mathbb{N}$
\[ \frac{1}{n} = \left| \frac{(-1)^n}{n} \right| \qquad \left|\frac{1}{n}\right| < \varepsilon \Rightarrow \left| \frac{(-1)^n}{n} \right| < \varepsilon \]
Nach Lemma \ref{lemma:geo} folgt aus der Definition von $\lim$
\begin{gather*}
	\forall \varepsilon \in \mathbb{R}_+\ \exists n_0 \in \mathbb{N}\ \forall n \in \mathbb{N} : n_0 < n \Rightarrow \left|\frac{1}{n}\right| < \varepsilon \Rightarrow \left|\frac{(-1)^n}{n}\right| < \varepsilon \\ 
	\defimpl{\lim} \lim_{n \rightarrow \infty} \frac{(-1)^n}{n} = 0
\end{gather*}
\end{proof}
\begin{align*}
	\lim_{n \rightarrow \infty} \left( 1 - \frac{1}{n^2} \right)^n &= \lim_{n \rightarrow \infty} \left( 1 + \frac{1}{n} \right)^n\left( 1 - \frac{1}{n} \right)^n \\
	\defimpl{e} \qquad &= e \lim_{n \rightarrow \infty} \left( 1 - \frac{1}{n} \right)^n = e\lim_{n \rightarrow \infty} \left( \frac{n}{n-1} \right)^{-n} = e\lim_{n \rightarrow \infty} \left( 1 + \frac{1}{n-1} \right)^{-n}\\
	&= e\lim_{n \rightarrow \infty} \left( 1 + \frac{1}{n-1} \right)^{-(n-1)}\underbrace{\left( 1 + \frac{1}{n-1} \right)^{-1}}_1 = e \left(\lim_{n \rightarrow \infty} \left( 1 + \frac{1}{n-1} \right)^{(n-1)}\right)^{-1} \\
	\defimpl{e} \qquad &= e \cdot e^{-1} = 1
\end{align*}
\end{document}
