\documentclass[a4paper, 12pt]{scrartcl}

\usepackage[utf8]{inputenc}
\usepackage[T1]{fontenc}
\usepackage[ngerman]{babel}

\usepackage{amssymb}
\usepackage{amsmath}
\usepackage{framed}
\usepackage{float}
\usepackage{mathtools}

\usepackage{tikz}

\usepackage{marvosym}
\usepackage{amsthm}
\usepackage{thmtools}
\usepackage{marvosym}

\usepackage[left=2cm, right=2cm, top=2cm]{geometry}

\allowdisplaybreaks

\setlength{\parindent}{0pt}

\declaretheoremstyle[%
  spaceabove=0,%
  spacebelow=6pt,%
  headfont=\normalfont\itshape,%
  postheadspace=1em,%
  headpunct={}
]{mystyle}

\declaretheorem[name={Behauptung}, style=mystyle, unnumbered]{theorem}
\declaretheorem[name={Lemma}, style=mystyle]{lemma}
\declaretheorem[name={Voraussetzung}, style=mystyle, unnumbered]{precondition}
\let\proof\oldproof
\declaretheorem[name={Beweis}, style=mystyle, qed=\qedsymbol, unnumbered]{proof}

\newcounter{taski}
\newcounter{taskii}[taski]
\newcounter{taskiii}[taskii]

\newcommand{\task}{\stepcounter{taski}\textbf{Aufgabe \arabic{taski}}\\}
\newcommand{\ttask}{\stepcounter{taskii}\textbf{(\alph{taskii})}\par}
\newcommand{\tttask}{\stepcounter{taskiii}\quad(\roman{taskiii})\par}

\begin{document}
\hfill Kamal
\begin{center}
	\bfseries Analysis I Blatt 1
\end{center}
\task
\ttask
\begin{theorem}
	\[ n > m \Rightarrow pn > pm \tag*{$m,\ n,\ p \in \mathbb{N}$} \]
\end{theorem}
\begin{proof}
\begin{align*}
	n &= m + r \tag*{$r \in \mathbb{N}$} \\
	pn &= p(m + r) = pm + pr \tag*{$pr \in \mathbb{N}$} \\
	pn &> pm
\end{align*}
\end{proof}
Obiger Beweis gilt analog für $m < n$. Für den Fall $m = n$ folgt sofort $pm = pn$.

\ttask
\begin{theorem}
	\[ m > n \:\wedge\: p > q \ \Rightarrow\ mp > nq \tag*{$m,\ n,\ p,\ q \in \mathbb{N}$} \]
\end{theorem}
\begin{proof}
\[ m = n + r \qquad p = q + s \tag*{$r,\ s \in \mathbb{N}$} \]
\begin{align*}
	mp &= (n+r)(q+s) = n(q+s) + r(q+s) = nq + \underbrace{(ns + rq + rs)}_{\in\:\mathbb{N}} \\
	mp &> nq
\end{align*}
\end{proof}

\task
\begin{theorem}
	\[ \nexists m,\ n \in \mathbb{N}\ (n < m < n+1) \]
\end{theorem}
\begin{proof}
Es wird das Gegenteil der Behauptung angenommen.
\[ n + r = m \qquad m + s = n+1 \tag*{$r,\ s \in \mathbb{N}$} \]
\begin{align*}
	m + s = (n + r) + s &= n + 1 \\
	r + s &= 1 \\
	r &< 1 \qquad \text{\Lightning}
\end{align*}
\end{proof}
\newpage
\task
\begin{theorem}
	\[ 6(1^2 + 2^2 + \hdots + n^2) = n(n+1)(2n+1) \tag*{$n \in \mathbb{N}$} \]
\end{theorem}
\begin{proof}
Für $n = 1$
\begin{equation}\label{IS}
	6(1^2) = 1(1+1)(2 \cdot 1 + 1) = 6 \qquad \text{w. A.}
\end{equation}
Gilt es als Induktionsannahme für ein beliebiges $n$, so ist
\begin{align}
	6(1^2 + 2^2 + \hdots + n^2) &= n(n+1)(2n+1) \nonumber\\
	6(1^2 + 2^2 + \hdots + n^2) + 6(n+1)^2 &= n(n+1)(2n+1)+ 6(n+1)^2 \nonumber\\
	6(1^2 + 2^2 + \hdots + (n+1)^2) &= (n+1)\left( n(2n+1) + 6(n+1) \right) \nonumber\\
	&= (n+1)\left( 2n^2 + 7n + 6 \right) = (n+1)(n+2)(2n+3) \nonumber\\ 
	6(1^2 + 2^2 + \hdots + (n+1)^2) &= (n+1)\left( (n+1)+1 \right)\left( 2(n+1)+1 \right)  \qquad, \label{IA}
\end{align}
sodass es für $n+1$ gilt. Aus Richtigkeit von Induktionsstart \eqref{IS} und Induktionsschritt \eqref{IA} folgt die Richtigkeit der Behauptung.
\end{proof}
\end{document}