\documentclass[a4paper, 12pt]{scrartcl}

\usepackage[utf8]{inputenc}
\usepackage[T1]{fontenc}
\usepackage[ngerman]{babel}

\usepackage{amssymb}
\usepackage{amsmath}
\usepackage{framed}
\usepackage{float}
\usepackage{mathtools}
\usepackage{marvosym}

\usepackage{tikz}
\usepackage{chngcntr}

\usepackage{amsthm}
\usepackage{thmtools}

\usepackage[left=2cm, right=2cm, top=2cm]{geometry}

\allowdisplaybreaks

\setlength{\parindent}{0pt}

\setkomafont{paragraph}{\normalfont\itshape}


\declaretheoremstyle[%
  spaceabove=0,%
  spacebelow=6pt,%
  headfont=\normalfont\itshape,%
  postheadspace=1em,%
  headpunct={}
]{mystyle}

\declaretheorem[name={Behauptung}, style=mystyle, unnumbered]{theorem}
\declaretheorem[name={Lemma}, style=mystyle]{lemma}
\declaretheorem[name={Voraussetzung}, style=mystyle, unnumbered]{precondition}
\let\proof\oldproof
\declaretheorem[name={Beweis}, style=mystyle, qed=\qedsymbol, unnumbered]{proof}

\newcounter{taski}
\newcounter{taskii}[taski]
\newcounter{taskiii}[taskii]

\newcommand{\task}{\stepcounter{taski}\textbf{Aufgabe \arabic{taski}}}
\newcommand{\ttask}{\stepcounter{taskii}\textbf{(\alph{taskii})}}
\newcommand{\tttask}{\stepcounter{taskiii}\quad(\roman{taskiii})}

\newcommand{\defimpl}[1]{\stackrel{\text{Def.}\;#1}{\Longrightarrow}}
\newcommand{\defImpl}[1]{\stackrel{\text{Def.}\;#1}{\Longleftrightarrow}}
\newcommand{\txtimpl}[1]{\stackrel{\text{#1}}{\Longrightarrow}}
\newcommand{\txtImpl}[1]{\stackrel{\text{#1}}{\Longleftrightarrow}}
\newcommand{\refimpl}[1]{\txtimpl{\eqref{#1}}}
\setcounter{taski}{25}

\begin{document}
\hfill \textit{Ellen, Kamal}\\
\task

\ttask
Im folgenden werden mehrere elementare Zeilenoperationen zusammengefasst. Die Bezeichnungen $\mathbf{Z1},\ \cdots,\ \mathbf{Z4}$ beziehen sich auf die Zeilen der \emph{vorherigen} Matrix. Steht kein Operator dabei, so wird die annotierte Zeile durch die jeweilige Zeile aus der vorherigen Matrix ersetzt, um das Tauschen von Zeilen zu simulieren.
\begin{gather*}
	\begin{pmatrix}
		0 & 1 & 2 & 3 & 4 & 5 \\
		2 & 1 & 0 & 3 & 4 & 5 \\
		0 & 1 & 2 & 3 & 5 & 4 \\
		3 & 4 & 5 & 2 & 1 & 0
	\end{pmatrix}
	\begin{matrix}
		\\\cdot 3\\\\\cdot 2 \\
	\end{matrix} \rightarrow
	\begin{pmatrix}
		0 & 1 & 2 & 3 & 4 & 5 \\
		6 & 3 & 0 & 9 & 12 & 15 \\
		0 & 1 & 2 & 3 & 5 & 4 \\
		6 & 8 & 10 & 4 & 2 & 0
	\end{pmatrix}\begin{matrix}
		\mathbf{Z2}\\\mathbf{Z1}\\\\-\mathbf{Z2}
	\end{matrix}\rightarrow
	\begin{pmatrix}
		6 & 3 & 0 & 9 & 12 & 15 \\
		0 & 1 & 2 & 3 & 4 & 5 \\
		0 & 1 & 2 & 3 & 5 & 4 \\
		0 & 5 & 10 & -5 & -10 & -15
	\end{pmatrix}\begin{matrix}
		\\-\tfrac{1}{5}\mathbf{Z4}\\-\mathbf{Z2}\\\\
	\end{matrix} \\ \rightarrow
	\begin{pmatrix}
		6 & 3 & 0 & 9 & 12 & 15 \\
		0 & 0 & 0 & 4 & 6 & 8 \\
		0 & 0 & 0 & 0 & 1 & -1 \\
		0 & 5 & 10 & -5 & -10 & -15
	\end{pmatrix}\begin{matrix}
		\\\mathbf{Z4}\\\mathbf{Z2}\\\mathbf{Z3}
	\end{matrix}\rightarrow
	\begin{pmatrix}
		6 & 3 & 0 & 9 & 12 & 15 \\
		0 & 5 & 10 & -5 & -10 & -15 \\
		0 & 0 & 0 & 4 & 6 & 8 \\
		0 & 0 & 0 & 0 & 1 & -1
	\end{pmatrix} \qquad \curvearrowright \mathrm{ZR}_{\mathbb{Q}}(M) = 4
\end{gather*}

\begin{gather*}
	\begin{pmatrix}
		0 & 1 & 2 & 3 & 4 & 0 \\
		2 & 1 & 0 & 3 & 4 & 0 \\
		0 & 1 & 2 & 3 & 0 & 4 \\
		3 & 4 & 0 & 2 & 1 & 0
	\end{pmatrix}
	\begin{matrix}
		\mathbf{Z2}\\\mathbf{Z1}\\-\mathbf{Z1}\\+\mathbf{Z2}
	\end{matrix} \rightarrow
	\begin{pmatrix}
		2 & 1 & 0 & 3 & 4 & 0 \\
		0 & 1 & 2 & 3 & 4 & 0 \\
		0 & 0 & 0 & 0 & 1 & 4 \\
		0 & 0 & 0 & 0 & 0 & 0
	\end{pmatrix} \qquad \curvearrowright \mathrm{ZR}_{\mathbb{F}_5}(M) = 3
\end{gather*}

\setcounter{taski}{27}
\task

\ttask
\begin{theorem}
	Seien $A,\ B \in M(n \times n,\ K)$, so
	\[ \mathrm{ZR}(AB) \leq \mathrm{ZR}(A) \]
\end{theorem}
\begin{proof}
Sei $A$ bereits in ZSF, so gilt für ein $r \in \mathbb{N}$
\begin{equation}\label{diegleichung}
	r \geq \mathrm{ZR}(A) \quad\txtImpl{Def. ZR}\quad \forall i \in \mathbb{N} : (r < i \leq n \ \Rightarrow\ A \in \mathcal{Z}_i)
\end{equation}

Sei nun $r = \mathrm{ZR}(A)$. Es folgt aus Lemma \ref{daslemma}
\begin{align*}
	\quad& \forall i \in \mathbb{N} : (r < i \leq n \Rightarrow\ A \in \mathcal{Z}_i\ \Rightarrow\ AB \in \mathcal{Z}_i) \\
	\refimpl{diegleichung} \quad& r \geq \mathrm{ZR}(AB) \\
	\defimpl{r} \quad& \mathrm{ZR}(AB) \leq \mathrm{ZR}(A)
\end{align*}
\end{proof}

\ttask

\tttask

\begin{theorem}
	\[ \mathrm{ZR}(A) < n \ \Longleftrightarrow\ \exists v \in K^n\backslash \left\{ 0 \right\} : v \cdot A = 0 \]
\end{theorem}
\begin{proof}
``$\Rightarrow$'': Es sei ein beliebiges $A \in M(n \times n,\ K)$ , $\mathrm{ZR}(A) < n$ , $A$ in ZSF
\[ \refimpl{diegleichung} \quad A \in \mathcal{Z}_n \]
Sei $v \in K^n\backslash \left\{ 0 \right\}$ definiert als
\[ v_{1k} \coloneqq \begin{cases}
	1 &\quad k = n \\
	0 &\quad \text{sonst}
\end{cases} \]
Aus Konstruktion folgt
\begin{align*}
	\txtimpl{Mat. Mul}\quad (v \cdot A)_{1j} &= \sum_{k=1}^n v_{1k}A_{kj} \\
	\defimpl{v}\quad\defimpl{\mathcal{Z}_n}\quad &= \sum_{k=1}^n \begin{cases}
		1 &\quad k = n \\
		0 &\quad \text{sonst}
	\end{cases}\ \cdot\ \begin{cases}
		0 &\quad k = n \\
		A_{kj} &\quad \text{sonst}
	\end{cases}
	\quad= 0
\end{align*}
``$\Leftarrow$'':
\begin{gather*}
	\quad \exists v \in K^n\backslash \left\{ 0 \right\} : v \cdot A = 0 \quad \txtimpl{Def. Lin. Unabh.} \quad \text{Zeilen von $A$ linear abhängig} \\
	\txtimpl{Def. Basis}\ \text{Zeilen von $A$ keine Basis}\ \txtimpl{Def. ZR}\ \mathrm{ZR}(A) \neq n \ \Rightarrow\ \mathrm{ZR}(A) < n
\end{gather*}
\end{proof}

\tttask
\begin{theorem}
Es sei $A \in M(n \times n,\ K)$
	\[ \exists A^{-1} \in M(n \times n,\ K) \ \Longrightarrow\ \mathrm{ZR}(A) = n \]
\end{theorem}
\begin{proof}

durch Widerspruch: Sei ein beliebiges $A,\ A^{-1} \in M(n \times n,\ K)$ , $\mathrm{ZR}(A) < n$ \\ $A$ in ZSF mit
\begin{align*}
	AA^{-1} &= 1 \\
	\defimpl{1} \quad (AA^{-1})_{nn} &= 1 \\
	\mathrm{ZR}(A) < n \refimpl{diegleichung} A \in \mathcal{Z}_n \Rightarrow AA^{-1} \in \mathcal{Z}_n \Rightarrow \quad (AA^{-1})_{nn} &= 0 \qquad \text{\large \Lightning}
\end{align*}
\end{proof}
\begin{lemma}\label{daslemma}
	Es sei für ein beliebiges $i \in A_n$ die Menge $\mathcal{Z}_i \subseteq M(n \times n,\ K)$ als
	\[ \mathcal{Z}_i \coloneqq \left\{ P \mid P \in M(n \times n,\ K)\ ,\ \forall k \in A_n : (P)_{ik} = 0 \right\} \]
	Es gilt für beliebige Matrizen $A,\ B \in M(n \times n,\ K)$
	\[ A \in \mathcal{Z}_i \Rightarrow AB \in \mathcal{Z}_i \]
\end{lemma}
\begin{proof}
Es seien ein beliebiges $i \in A_n$, $A \in \mathcal{Z}_i$ und $B \in M(n \times n,\ K)$
\[ \txtimpl{Mat. Mul}\quad (AB)_{ij} = \sum_{k = 1}^n A_{ik}B_{kj} \quad\defimpl{\mathcal{Z}_i}\quad (AB)_{ij} = 0 \quad\defimpl{\mathcal{Z}_i}\quad AB \in \mathcal{Z}_i \]
\end{proof}
\emph{Anmerkung}: Jede elementare Zeilenumformung ist invertierbar und lässt den Zeilenrang invariant. Demnach kann o.B.d.A in (i) und (ii) von ZSF ausgegangen werden.
\end{document}