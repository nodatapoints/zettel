\documentclass[a4paper, 12pt]{scrartcl}

\usepackage[utf8]{inputenc}
\usepackage[T1]{fontenc}
\usepackage[ngerman]{babel}

\usepackage{amssymb}
\usepackage{amsmath}
\usepackage{framed}
\usepackage{float}
\usepackage{mathtools}
\usepackage{marvosym}

\usepackage{tikz}
\usepackage{chngcntr}

\usepackage{amsthm}
\usepackage{thmtools}

\usepackage[left=2cm, right=2cm, top=2cm]{geometry}

\allowdisplaybreaks

\setlength{\parindent}{0pt}

\setkomafont{paragraph}{\normalfont\itshape}


\declaretheoremstyle[%
  spaceabove=0,%
  spacebelow=6pt,%
  headfont=\normalfont\itshape,%
  postheadspace=1em,%
  headpunct={}
]{mystyle}

\declaretheorem[name={Behauptung}, style=mystyle, unnumbered]{theorem}
\declaretheorem[name={Lemma}, style=mystyle]{lemma}
\declaretheorem[name={Voraussetzung}, style=mystyle, unnumbered]{precondition}
\let\proof\oldproof
\declaretheorem[name={Beweis}, style=mystyle, qed=\qedsymbol, unnumbered]{proof}

\newcounter{taski}
\newcounter{taskii}[taski]
\newcounter{taskiii}[taskii]

\newcommand{\task}{\stepcounter{taski}\textbf{Aufgabe \arabic{taski}}}
\newcommand{\ttask}{\stepcounter{taskii}\textbf{(\alph{taskii})}}
\newcommand{\tttask}{\stepcounter{taskiii}\quad(\roman{taskiii})}

\newcommand{\defimpl}[1]{\stackrel{\text{Def.}\;#1}{\Longrightarrow}}
\newcommand{\defImpl}[1]{\stackrel{\text{Def.}\;#1}{\Longleftrightarrow}}
\newcommand{\txtimpl}[1]{\stackrel{\text{#1}}{\Longrightarrow}}
\newcommand{\txtImpl}[1]{\stackrel{\text{#1}}{\Longleftrightarrow}}
\newcommand{\refimpl}[1]{\txtimpl{\eqref{#1}}}

\setcounter{taski}{32}

\newcommand{\pvec}[1]{\ensuremath{\begin{pmatrix}#1\end{pmatrix}}}

\DeclareMathOperator{\bild}{Bild}
\DeclareMathOperator{\lin}{Lin}
\DeclareMathOperator{\rang}{Rang}
\DeclareMathOperator{\Kern}{Kern}
% \DeclareMathOperator{\hom}{Hom}
\DeclareMathOperator{\Endo}{End}
\DeclareMathOperator{\aut}{Aut}
\DeclareMathOperator{\iso}{Iso}

\newcommand{\R}{\mathbb{R}}

\begin{document}
\hfill \textit{Ellen, Kamal}\\
\task

\ttask\
$w_1$ und $w_2$ linear unabhängig, da bei $\lambda_1,~\lambda_2 \in \mathbb{R}$ mit $\lambda_1w_1 + \lambda_2 w_2 = 0$
\begin{gather*}
	- \lambda_1 - \lambda_2 = - 2\lambda_1 - 2\lambda_2 = 0 \qquad 2 \lambda_1 + 3 \lambda_2 = 0 \\
	\Rightarrow \quad \lambda_2 = 0 \quad\Rightarrow \quad\lambda_1 = - \lambda_2 = 0
\end{gather*}
Es gilt
\begin{gather*}
	w_3 = \begin{pmatrix}
	-1 \\ 1 \\ 1
\end{pmatrix} = -2\begin{pmatrix}
	2 \\ -1 \\ 3
\end{pmatrix} + \begin{pmatrix}
	3 \\ -1 \\ 7
\end{pmatrix} = -2w_1 + w_2 \\
	\Rightarrow \quad w_3 \in \lin((w_1,\ w_2)) \quad \Rightarrow \quad \lin((w_1,\ w_2)) = \lin((w_1,\ w_2,\ w_3)) = \varphi(\lin(v_1,\ v_2,\ v_3)) = \bild(\varphi)
\end{gather*}

$(w_1,~w_2)$ bilden ein ES und somit eine Basis von $\bild(\varphi)$. $\rang(\varphi) = 2$

\ttask\ Nach Dimensionsformel
\begin{align*}
	3 = \dim_\R \R^3 &= \dim_\R(\Kern(\varphi)) + \overbrace{\rang(\varphi)}^2 \\
	\dim_\R(\Kern(\varphi)) &= 3 - 2 = 1
\end{align*}

\ttask\ Sei eine Abbildung $\theta : \mathbb{R}^2 \rightarrow U$ definiert als
\begin{gather*}
	\theta(x) = \begin{pmatrix}
	2 & 3 \\ -1 & -1 \\ 3 & 7 \\
\end{pmatrix}x \\
	\bild(\theta) = \theta(\R^2) = \lin((\theta(e_1),\ \theta(e_2))) = \lin((w_1,\ w_2)) = U \\
	\txtimpl{9.10 b} \quad \theta \text{ surjektiv}
\end{gather*}
Sei $\lambda_1,\ \lambda_2 \in \R$ , $x \in \R^2$ , $x = \lambda_1e_1 + \lambda_2e_2$ und
\begin{gather*}
	0 = \theta(x) = \theta(\lambda_1e_1 + \lambda_2e_2) =\lambda_1\theta(e_1) + \lambda_2\theta(e_2) \\
	0 = \lambda_1w_1 + \lambda_2w_2 \quad \txtimpl{Basis $U$} \quad \lambda_1 = \lambda_2 = 0 \\
	\Rightarrow \quad x = 0 \quad,\quad \Kern(\theta) = \left\{ 0 \right\} \\
	\txtimpl{9.10 c} \quad \theta \text{ injektiv}
\end{gather*}
Demnach ist $\theta$ sowohl injektiv als auch surjektiv, womit es bijektiv ist. Es liegt ein Isomorphismus vor.

\ttask
\begin{gather*}
	f(w_1) = \begin{pmatrix} 3 \\ -4 \end{pmatrix} \qquad f(w_2) = \begin{pmatrix} 4 \\ -5 \end{pmatrix} \\
	\Rightarrow \quad \left( f(w_1),\ f(w_2) \right) \text{ Basis von } \R^2
\end{gather*}
Analog zur vorherigen Teilaufgabe folgt daraus, dass $f$ bijektiv ist. Es liegt ein Isomorphismus vor. ($\theta \leftrightarrow f$)

\setcounter{taski}{34}
Sei die Ableitung $p' \in P_2$ eines Polynoms $p \in P_3$ definiert als
\[ p= a_0p_0 + a_1p_1 + a_2p_2 + a_3p_3 \qquad p' = a_1p_0 + 2a_2p_1 + 3a_3p_2 \]
\task

\ttask
\tttask\  Sei ein beliebiges $p \in \Kern(\varphi)$ gegeben, so gilt
\begin{gather*}
	p \coloneqq a_0p_0 + a_1p_1 + a_2p_2 + a_3p_3 \\
	p' = a_1p_0 + 2a_2p_1 + 3a_3p_2 \txtImpl{Basis} a_1 = a_2 = a_3 = 0 \\
	\Rightarrow\quad (p \in \Kern(\varphi) ~\Leftrightarrow~ a_1 = a_2 = a_3 = 0) \\
	\Rightarrow\quad \Kern(\varphi) = \lin(p_0)
\end{gather*}
Es sei ein beliebiges $p' \in P_2$ gegeben. Es sei ein $p \in P_3$ sodass
\begin{gather*}
	p' \coloneqq a_0'p_0 + a_1'p_1 + a_2'p_2 \\
	p \coloneqq a_0'p_1 + \frac{1}{2}a_1'p_2 + \frac{1}{3}a_2'p_3
\end{gather*}
Aus Konstruktion folgt dass $\varphi(p) = p'$, sodass $\varphi$ surjektiv ist und somit $\bild(\varphi) = P_2$.
\\

\tttask\ Es sei ein beliebiges $ x = (x_1,\ x_2,\ x_3)^\top \in \R^3$ gegeben, so seien $a_0,\ \dots,\ a_3 \in \R$ , $p \in P_3$
\[ p = a_0p_0 + a_1p_1 + a_2p_2 + a_3p_3\qquad\begin{pmatrix}
	a_0 \\ a_1 \\ a_2 \\ a_3 
\end{pmatrix} = \begin{pmatrix}
	1 & 0 & 0 \\
	- \frac 3 2 & 2 & - \frac 1 2 \\
	\frac 1 2 & -1 & \frac 1 2 \\
	0 & 0 & 0
\end{pmatrix}x \]
Es folgt aus Konstruktion dass $\varphi(p) = (0,\ 1,\ 2)$, sodass $\varphi$ surjektiv ist und somit $\bild(\varphi) = \R^3$.
\begin{gather*}
	\forall x \in \R : \big( x(x-1)(x-2) = 2p_1(x) - 3p_2(x) + p_3(x) = 0 \Leftrightarrow x \in \left\{ 0,\ 1,\ 2 \right\} \big) \\
	\Rightarrow 2p_1(x) - 3p_2(x) + p_3(x) \in \Kern(\varphi) \\
	\dim_\R \Kern(\varphi) \stackrel{\text{Dim. Formel}}{=} \dim_\R P_3 - \dim_\R \R^3 = 4 - 3 = 1 \\
	\Rightarrow \quad \Kern(\varphi) = \lin((2p_1(x) - 3p_2(x) + p_3(x)))
\end{gather*}
\ttask
Surjektivität wurde bereits für beide Abbildungen gezeigt. Für beide Abbildungen gilt
\[ \dim_\R \Kern(\varphi) \neq 0 ~\txtimpl{9.10 c}~ \varphi \text{ nicht injektiv} \]
\end{document}