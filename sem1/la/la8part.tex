\documentclass[a4paper, 12pt]{scrartcl}

\usepackage[utf8]{inputenc}
\usepackage[T1]{fontenc}
\usepackage[ngerman]{babel}

\usepackage{amssymb}
\usepackage{amsmath}
\usepackage{framed}
\usepackage{float}
\usepackage{mathtools}
\usepackage{marvosym}

\usepackage{tikz}
\usepackage{chngcntr}

\usepackage{amsthm}
\usepackage{thmtools}

\usepackage[left=2cm, right=2cm, top=2cm]{geometry}

\allowdisplaybreaks

\setlength{\parindent}{0pt}

\setkomafont{paragraph}{\normalfont\itshape}


\declaretheoremstyle[%
  spaceabove=0,%
  spacebelow=6pt,%
  headfont=\normalfont\itshape,%
  postheadspace=1em,%
  headpunct={}
]{mystyle}

\declaretheorem[name={Behauptung}, style=mystyle, unnumbered]{theorem}
\declaretheorem[name={Lemma}, style=mystyle]{lemma}
\declaretheorem[name={Voraussetzung}, style=mystyle, unnumbered]{precondition}
\let\proof\oldproof
\declaretheorem[name={Beweis}, style=mystyle, qed=\qedsymbol, unnumbered]{proof}

\newcounter{taski}
\newcounter{taskii}[taski]
\newcounter{taskiii}[taskii]

\newcommand{\task}{\stepcounter{taski}\textbf{Aufgabe \arabic{taski}}}
\newcommand{\ttask}{\stepcounter{taskii}\textbf{(\alph{taskii})}}
\newcommand{\tttask}{\stepcounter{taskiii}\quad(\roman{taskiii})}

\newcommand{\defimpl}[1]{\stackrel{\text{Def.}\;#1}{\Longrightarrow}}
\newcommand{\defImpl}[1]{\stackrel{\text{Def.}\;#1}{\Longleftrightarrow}}
\newcommand{\txtimpl}[1]{\stackrel{\text{#1}}{\Longrightarrow}}
\newcommand{\txtImpl}[1]{\stackrel{\text{#1}}{\Longleftrightarrow}}
\newcommand{\refimpl}[1]{\txtimpl{\eqref{#1}}}
\setcounter{taski}{30}

\newcommand{\pvec}[1]{\ensuremath{\begin{pmatrix}#1\end{pmatrix}}}

\begin{document}
\hfill \textit{Ellen, Kamal}\\
\task

\ttask
\begin{gather*}
	f(v) = f \left( -\frac 1 4 (v_1 - v_2) + v_3 \right) = \ -\frac 1 4 (f(v_1) - f(v_2)) + f(v_3) = -\frac 1 4 (w_1 - w_2) + w_3 = \frac 1 2 \pvec{7 \\ -1 \\ 8} \\
	f(v') = f \left( v_1 - v_3 \right) = \ f(v_1) - f(v_3) = w_1 - w_3 = \pvec{-2 \\ 1 \\ -3} \\
\end{gather*}

\ttask

\begin{align*}
	A
	\begin{pmatrix}
		 1 & 1 & 1 \\
		 1 & 1 & 0 \\
		 1 & -3 & 1 \\
	\end{pmatrix} &=
	\begin{pmatrix}
	 	1 & 3 & 3 \\
	 	2 & -4 & 1 \\
	 	-1 & 7 & 2
	 \end{pmatrix} \\ 
	A
	\begin{pmatrix}
		 1 & 1 & 1 \\
		 1 & 1 & 0 \\
		 1 & -3 & 1 \\
	\end{pmatrix}
	\begin{pmatrix}
	 	1 & 3 & 3 \\
	 	2 & -4 & 1 \\
	 	-1 & 7 & 2
	 \end{pmatrix}^{-1} = A &=
	\begin{pmatrix}
	 	1 & 3 & 3 \\
	 	2 & -4 & 1 \\
	 	-1 & 7 & 2
	 \end{pmatrix}
	 \begin{pmatrix}
	 	1 & 3 & 3 \\
	 	2 & -4 & 1 \\
	 	-1 & 7 & 2
	 \end{pmatrix}^{-1} \\
	 A &= \frac 1 2 \begin{pmatrix}
	 	7 & -4 & 1 \\
	 	-1&  2 & 3 \\
	 	8 & -6 & 4
	 \end{pmatrix}
\end{align*}

\emph{Anmerkung}: Numerische Berechnung mit \texttt{numpy} durchgeführt.

\ttask
\begin{gather*}
	\varphi(0) = \varphi \left( 2\pvec{1 \\ 1 \\ 1} + \pvec{1 \\ 1 \\ -3} - \pvec{3 \\ 3 \\ -1} \right) = 2\pvec{-1 \\ 2 \\ -3} + \pvec{ 1 \\ -3 \\ 4} - \pvec{-1 \\ 1 \\ 2} = \pvec{0 \\ 0 \\ -4} \neq 0 \quad \text{\large \Lightning}
\end{gather*}

\task

\ttask

\tttask\ nicht linear
\[ 1 = f_1\left((1,~1)\right) = f_1\left((1,~0) + (0,~1)\right) = f_1\left((1,~0)\right) + f_1\left((0,~1)\right) = 0 + 0 \quad \text{\large \Lightning} \]
\tttask\ nicht linear
\[ f_2(0) = (0+1,\ 2 \cdot 0,\ 0 + 0) \neq 0 \quad \text{\large \Lightning} \]
\tttask\ linear
\begin{gather*}
	M = \begin{pmatrix}
		0 & 0 & 1 \\
		0 & 2 & 0 \\
		0 & 0 & 0
	\end{pmatrix} \qquad (x_3,\ 2x_2,\ 0)^\top = A \cdot (x_1,\ x_2,\ x_3)^\top \tag{$\forall x_1\, x_2,\ x_3 \in \mathbb{R}$}\\
	\curvearrowright \quad f_3 = \widetilde{M}
\end{gather*}
\newpage

\ttask

Nach Rechenregeln für Matrizen mit $A,\ B \in M(2 \times 2,
\ \mathbb{R})$ , $\lambda \in \mathbb{R}$
\begin{gather*}
	f_4(A + B) = \pvec{\cos \alpha & -\sin \alpha \\ \sin \alpha & \cos \alpha}(A + B) = \pvec{\cos \alpha & -\sin \alpha \\ \sin \alpha & \cos \alpha} A + \pvec{\cos \alpha & -\sin \alpha \\ \sin \alpha & \cos \alpha} B = f_4(A) + f_4(B)\\
	f_4(\lambda A) = \pvec{\cos \alpha & -\sin \alpha \\ \sin \alpha & \cos \alpha} (\lambda A) = \lambda \pvec{\cos \alpha & -\sin \alpha \\ \sin \alpha & \cos \alpha} A = \lambda f_4(A)
\end{gather*}
sodass beide Bedingungen für lineare Abbildungen erfüllt sind.

\ttask

Für reell-wertige Folgen $(a_n)_{n \in \mathbb{N}},\ (b_n)_{n \in \mathbb{N}},\ (c_n)_{n \in \mathbb{N}},\ (d_n)_{n \in \mathbb{N}}$ und $\lambda \in \mathbb{R}$ sei
\begin{gather*}
	(a_n')_{n \in \mathbb{N}} = f_5((a_n)_{n \in \mathbb{N}}) \qquad (b_n')_{n \in \mathbb{N}} = f_5((b_n)_{n \in \mathbb{N}}) \qquad (c_n')_{n \in \mathbb{N}} = f_5((c_n)_{n \in \mathbb{N}})	\qquad (d_n')_{n \in \mathbb{N}} = f_5((d_n)_{n \in \mathbb{N}}) \\
	\forall n \in \mathbb{N} : c_n = a_n + b_n \\
	\forall n \in \mathbb{N} : d_n = \lambda a_n
\end{gather*}
Es folgt
\begin{align*}
	\forall n \in \mathbb{N} : \quad c_n' &= c_{n+1} = a_{n+ 1} + b_{n+1} = a_n' + b_n' \\
	\Rightarrow \quad & (c'_n)_{n \in \mathbb{N}} = (a_n')_{n \in \mathbb{N}} + (b_n')_{n \in \mathbb{N}} \\\\
	\forall n \in \mathbb{N} : \quad \lambda a_n' &= \lambda a_{n+1} = d_{n+1} = d_n' \\
	\Rightarrow \quad & (d_n')_{n \in \mathbb{N}} = \lambda (a_n')_{n \in \mathbb{N}}
\end{align*}
sodass beide Bedingungen für lineare Abbildungen erfüllt sind.
\end{document}