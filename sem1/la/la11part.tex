\documentclass[a4paper, 12pt]{scrartcl}

\usepackage[utf8]{inputenc}
\usepackage[T1]{fontenc}
\usepackage[ngerman]{babel}

\usepackage{amssymb}
\usepackage{amsmath}
\usepackage{framed}
\usepackage{float}
\usepackage{mathtools}
\usepackage{marvosym}

\usepackage{tikz}
\usepackage{tikz-cd}
\usepackage{chngcntr}

\usepackage{amsthm}
\usepackage{thmtools}
\usepackage{wrapfig}


\usepackage[left=2cm, right=2cm, top=2cm]{geometry}

\allowdisplaybreaks

\setlength{\parindent}{0pt}

\setkomafont{paragraph}{\normalfont\itshape}


\declaretheoremstyle[%
  spaceabove=0,%
  spacebelow=6pt,%
  headfont=\normalfont\itshape,%
  postheadspace=1em,%
  headpunct={}
]{mystyle}

\declaretheorem[name={Behauptung}, style=mystyle, unnumbered]{theorem}
\declaretheorem[name={Lemma}, style=mystyle]{lemma}
\declaretheorem[name={Voraussetzung}, style=mystyle, unnumbered]{precondition}
\let\proof\oldproof
\declaretheorem[name={Beweis}, style=mystyle, qed=\qedsymbol, unnumbered]{proof}

\newcounter{taski}
\newcounter{taskii}[taski]
\newcounter{taskiii}[taskii]

\newcommand{\task}{\stepcounter{taski}\textbf{Aufgabe \arabic{taski}}}
\newcommand{\ttask}{\stepcounter{taskii}\textbf{(\alph{taskii})}}
\newcommand{\tttask}{\stepcounter{taskiii}\quad(\roman{taskiii})}

\newcommand{\defimpl}[1]{\stackrel{\text{Def.}\;#1}{\Longrightarrow}}
\newcommand{\defImpl}[1]{\stackrel{\text{Def.}\;#1}{\Longleftrightarrow}}
\newcommand{\txtimpl}[1]{\stackrel{\text{#1}}{\Longrightarrow}}
\newcommand{\txtImpl}[1]{\stackrel{\text{#1}}{\Longleftrightarrow}}
\newcommand{\refimpl}[1]{\txtimpl{\eqref{#1}}}


\setcounter{taski}{41}

\newcommand{\pvec}[1]{\ensuremath{\begin{pmatrix}#1\end{pmatrix}}}

\begin{document}
\hfill \textit{Ellen, Kamal}\\
\task

\ttask

Der Vektorraum $\mathbb{R}^3$ ist gleich dem Koordinatenraum $\mathbb{R}^3$ des Körpers $\mathbb{R}$. Demnach kann die Koordinatentransformation als Abbildung $\widetilde{H}$ einer Matrix $H \in M(3 \times 3,\ \mathbb{R})$ dargestellt werden. Die Spalten von $H$ sind dabei die Vektoren $v_1,\ v_2,\ v_3$.
\begin{gather*}
    H = \begin{pmatrix} 1 & 0 & 1 \\ 0 & 1 & 3 \\ 1 & 1 & 2 \end{pmatrix} \quad \Phi_B = \widetilde{H} \qquad\qquad H^{-1} = \frac{1}{2} \begin{pmatrix} 1 & -1 & 1 \\ -3 & -1 & 3 \\ 1 & 1 & -1 \end{pmatrix} \quad \Phi^{-1}_B = \widetilde{H^{-1}} \\
    \Phi_B^{-1}(w) = H^{-1}w = \frac{1}{2} \pvec{-1 \\ 5 \\ -1}
\end{gather*}

\ttask

Nach selber Argumentation kann auch $f$ dargestellt werden als $\widetilde{F}$ einer Matrix $F \in M(3 \times 3,\ \mathbb{R})$
\begin{gather*}
    F = \begin{pmatrix} 0 & -1 & 1 \\ -3 & -2 & 3 \\ -2 & -2 & 3 \end{pmatrix} \quad f = \widetilde{F} \qquad \qquad F^{-1} = \begin{pmatrix} 0 & -1 & 1 \\ -3 & -2 & 3 \\ -2 & -2 & 3 \end{pmatrix} = F \quad f^{-1} = \widetilde{F^{-1}} = \widetilde{F} = f
\end{gather*}
Die gefragten Matrizen ergeben sich demnach direkt aus den entsprechenden Produkten
\begin{align*}
    f &= \Phi_{(e_1,e_2,e_3)} \circ \widetilde{M_{(e_1,e_2,e_3)}^{(e_1,e_2,e_3)}}(f) \circ \Phi_{{(e_1,e_2,e_3)}}^{-1}
    &\Longrightarrow \qquad
    F &= E_3 \cdot M_{(e_1,e_2,e_3)}^{(e_1,e_2,e_3)}(f) \cdot E_3^{-1} = M_{(e_1,e_2,e_3)}^{(e_1,e_2,e_3)}(f) \\
    f &= \Phi_{B} \circ \widetilde{M_{B}^{B}}(f) \circ \Phi_{{B}}^{-1}
    &\Longrightarrow \qquad
    F &= H \cdot M_B^B(f) \cdot H^{-1} \\
    && M_B^B(f) &= H^{-1} \cdot F \cdot H = \mathrm{diag}(1,\ 1,\ -1) \\
    \mathrm{id}_{\mathbb{R}^3} &= \Phi_B \circ \widetilde{M^{(e_1,e_2,e_3)}_B(\mathrm{id}_{\mathbb{R}^3})} \circ \Phi_{(e_1,e_2,e_3)}^{-1}
    &\Longrightarrow \qquad
    E_3 &= H \cdot M^{(e_1,e_2,e_3)}_B(\mathrm{id}_{\mathbb{R}^3}) \cdot E_3^{-1} \\
    && M^{(e_1,e_2,e_3)}_B(\mathrm{id}_{\mathbb{R}^3}) &= H^{-1}
\end{align*}

\ttask

\begin{align*}
    f(w) &= \pvec{-1+1\\3-2+3\\2-2+3} = \pvec{0\\4\\3} \\
    (\Phi_B \circ \widetilde{W_B^B(f)} \circ \Phi_B^{-1})(w) &= \underbrace{\begin{pmatrix} 1 & 0 & 1 \\ 0 & 1 & 3 \\ 1 & 1 & 2 \end{pmatrix} \cdot \begin{pmatrix}
        1 & 0 & 0 \\ 0 & 1 & 0 \\ 0 & 0 & -1
    \end{pmatrix} \cdot \frac{1}{2} \begin{pmatrix} 1 & -1 & 1 \\ -3 & -1 & 3 \\ 1 & 1 & -1 \end{pmatrix}}_F \begin{pmatrix} -1 \\ 1 \\ 1 \end{pmatrix} = \pvec{0\\4\\3}
\end{align*}

\newpage
\setcounter{taski}{43}

\task

$\xrightarrow{\sim}$ steht in den folgenden Diagrammen für einen Isomorphismus.

\ttask

\begin{wrapfigure}[8]{l}{4cm}
    \centering
    \begin{tikzcd}[sep=15mm]
        V   \arrow{r}{\mathrm{id}_V}
            \arrow{d}{\Phi^{-1}_B}[above, rotate=90]{\sim}
        & V  \\
        K^s \arrow{r}{\widetilde{E_s}}  & K^s \arrow{u}[right]{\Phi_B}[above, rotate=90]{\sim}
    \end{tikzcd}    
\end{wrapfigure}
Wie im Diagramm zu sehen ist führt der obere Weg wie der untere von $V$ nach $V$. Es gilt 
\[ \mathrm{id}_V = \Phi_B \circ \widetilde{M_B^B(\mathrm{id}_V)} \circ \Phi_B^{-1} = \Phi_B \circ \widetilde{E_s} \circ \Phi_B^{-1} \]
Nach Eindeutigkeit der Abbildung gilt somit
\[ \widetilde{M_B^B(\mathrm{id}_V)} = \widetilde{E_s} \quad,\quad {M_B^B(\mathrm{id}_V)} = {E_s}\]

\ttask

\begin{wrapfigure}[8]{l}{6cm}
    \centering
    \vspace*{-1em}
    \begin{tikzcd}[row sep=15mm, column sep=15mm]
        U \arrow{d}{\Phi^{-1}_A}[above, rotate=90]{\sim} \arrow{r}{f} \arrow[bend left]{rr}{f \circ g} &
        V \arrow{d}{\Phi^{-1}_B}[above, rotate=90]{\sim}\arrow{r}{g} &
        \arrow{d}{\Phi^{-1}_C}[above, rotate=90]{\sim}W
        \\
        K^r \arrow{r}{\widetilde{M_B^A(f)}} \arrow[bend right]{rr}[below]{\widetilde{M_C^A(f \circ g)}} &
        K^s \arrow{r}{\widetilde{M_C^B(g)}} &
        K^t
    \end{tikzcd}
\end{wrapfigure}

\vspace{1em}

Nach Definition sind beide Quadrate sowie der äußerste Ring kommutativ. Die obere Halbschale $f,\ g,\ f \circ g$ ist klarer Weise auch kommutativ. Demnach muss auch die untere Halbschale kommutativ sein. Es gilt nach den Eigenschaften von Matrizenabbildungen
\begin{align*}
    \widetilde{M_C^B(g)} &= \widetilde{M_C^B(g)} \circ \widetilde{M_B^A(f)} \\
    \Rightarrow \qquad M_C^B(g) &= M_C^B(g) \cdot M_B^A(f) \qquad \qquad
\end{align*}

\ttask

\begin{wrapfigure}[5]{l}{4cm}
    \centering
    \vspace*{-1em}
    \begin{tikzcd}[sep=2cm]
        V   \arrow{r}{f}[below]{\sim}
            \arrow{d}{\Phi^{-1}_B}[above, rotate=90]{\sim}
        & W \arrow{d}{\Phi^{-1}_C}[above, rotate=90]{\sim} \\
        K^s \arrow{r}{\widetilde{M_C^B(f)}}  &
        K^s \arrow[bend left]{l}{\widetilde{M_B^C(f^{-1})}}
    \end{tikzcd}    
\end{wrapfigure}
Nach Definition sind sowohl das innere Quadrat als auch der äußere Ring kommutativ. Demnach gilt für den Pfad von $K^s \longrightarrow K^s$
\begin{align*}
    \widetilde{M_B^C(f^{-1})} &= \Phi_B^{-1} \circ f^{-1} \circ \Phi_C \\
    &= \left( \Phi_C^{-1} \circ f \circ \Phi_B \right)^{-1} = \left(\widetilde{M_B^C(f)}\right)^{-1} \\\\\\
\end{align*}

\ttask

\begin{align*}
    f(v_1 + v_2) &= f(v_1) + f(v_2) = \left(\Phi_B \circ \widetilde{M_B^B(f)
    } \circ \Phi_B^{-1}\right)(v_1 + v_2) = \Phi_B \left( M_B^B(f)(e_1 + e_2) \right) \\
    &= \Phi_B \left( \begin{pmatrix}
    1 & 2 \\ 3 & 4
    \end{pmatrix} \begin{pmatrix}
        1 \\ 1
    \end{pmatrix} \right) = \Phi_B \left( \pvec{3 \\ 7} \right) = 3v_1 + 7v_2 
\end{align*}

\end{document}