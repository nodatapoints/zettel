\documentclass[a4paper, 12pt]{scrartcl}

\usepackage[utf8]{inputenc}
\usepackage[T1]{fontenc}
\usepackage[ngerman]{babel}

\usepackage{amssymb}
\usepackage{amsmath}
\usepackage{framed}
\usepackage{float}
\usepackage{mathtools}
\usepackage{marvosym}

\usepackage{tikz}
\usepackage{chngcntr}

\usepackage{amsthm}
\usepackage{thmtools}

\usepackage[left=2cm, right=2cm, top=2cm]{geometry}

\allowdisplaybreaks

\setlength{\parindent}{0pt}

\setkomafont{paragraph}{\normalfont\itshape}


\declaretheoremstyle[%
  spaceabove=0,%
  spacebelow=6pt,%
  headfont=\normalfont\itshape,%
  postheadspace=1em,%
  headpunct={}
]{mystyle}

\declaretheorem[name={Behauptung}, style=mystyle, unnumbered]{theorem}
\declaretheorem[name={Lemma}, style=mystyle]{lemma}
\declaretheorem[name={Voraussetzung}, style=mystyle, unnumbered]{precondition}
\let\proof\oldproof
\declaretheorem[name={Beweis}, style=mystyle, qed=\qedsymbol, unnumbered]{proof}

\newcounter{taski}
\newcounter{taskii}[taski]
\newcounter{taskiii}[taskii]

\newcommand{\task}{\stepcounter{taski}\textbf{Aufgabe \arabic{taski}}\\}
\newcommand{\ttask}{\stepcounter{taskii}\textbf{(\alph{taskii})}}
\newcommand{\tttask}{\stepcounter{taskiii}\quad(\roman{taskiii})}

\newcommand{\defimpl}[1]{\stackrel{\text{Def.}\;#1}{\Longrightarrow}}
\newcommand{\defImpl}[1]{\stackrel{\text{Def.}\;#1}{\Longleftrightarrow}}
\newcommand{\txtimpl}[1]{\stackrel{\text{#1}}{\Longrightarrow}}
\newcommand{\txtImpl}[1]{\stackrel{\text{#1}}{\Longleftrightarrow}}

\setcounter{taski}{16}

\begin{document}
\hfill \textit{Ellen, Kamal}\\
\task
\ttask \ linear abhängig da
\[ 2\begin{pmatrix} 2 \\ 1 \\ 1 \end{pmatrix} + 1\begin{pmatrix} 1 \\ 2 \\ 1 \end{pmatrix} - 1\begin{pmatrix} 2 \\ 1 \\ 0 \end{pmatrix} = \vec{0} \]
\ttask \ Es seien $\lambda_1,\ \lambda_2,\ \lambda_3 \in \mathbb{R}$, welche erfüllen dass
\begin{gather*}
	\forall x \in \mathbb{R}: \lambda_1\sin x + \lambda_2\cos x + \lambda_3 \sin(2x) = 0 
\intertext{Unter Betrachtung spezieller Fälle folgt daraus}
	x = 0 \Rightarrow  \lambda_1 \cdot 0 + \lambda_2 \cdot 1 + \lambda_3 \cdot 0 = 0\ ,\ \lambda_2 = 0 \\
	x = \frac{\tau}{4} \Rightarrow  \lambda_1 \cdot 1 + \lambda_2 \cdot 0 + \lambda_3 \cdot 0 = 0\ ,\ \lambda_1 = 0 \tag{$\pi \coloneqq \frac{1}{2}\tau$}\\
	x = \frac{\tau}{8} \Rightarrow  0 \cdot \sin x + 0 \cdot \cos x + \lambda_3 \cdot 1 = 0\ ,\ \lambda_3 = 0
\end{gather*}
Da $0_{\mathbb{R}^\mathbb{R}} : \mathbb{R} \rightarrow \mathbb{R}$ , $x \mapsto 0$ das Nullelement ist, und $\lambda_1 = \lambda_2 = \lambda_3 = 0$ folgt, sind die Funktionen linear unabhängig.

\ttask\ Es seien $\lambda_1,\ \lambda_2,\ \lambda_3 \in \mathbb{Q}$ welche erfüllen dass
\begin{align*}
	\lambda_1 + \lambda_2 \sqrt{2} + \lambda_3 \sqrt{3} &= 0 \\
	2\lambda_2^2 + 2 \lambda_2 \lambda_3 \sqrt{6} + 3 \lambda_3^2 &= \lambda_1^2 \tag{$\lambda_2\lambda_3 \neq 0$}\\
	\sqrt{6} &= \frac{\lambda_1^2 - 2\lambda_2^2 - 3 \lambda_3^2}{2\lambda_2\lambda_3} \in \mathbb{Q} \qquad \text{\large \Lightning} \\
	&\Rightarrow \lambda_2 = 0 \vee \lambda_3 = 0
\end{align*}
Sei $\lambda_2 = 0$
\begin{align*}
 	\Rightarrow \lambda_1 + \lambda_3\sqrt{3} &= 0 \tag{$\lambda_3 \neq 0$} \\
 	\sqrt{3} &= -\frac{\lambda_1}{\lambda_3} \in \mathbb{Q} \qquad \text{\large \Lightning} \\
 	&\Rightarrow \lambda_3 = 0
\end{align*}
Obige Vorgehensweise gilt analog für $\lambda_2,\ \sqrt{2}$ ausgehend von der Annahme dass $\lambda_3 = 0$. Es bleibt $\lambda_1 + 0 + 0 = 0$, womit folgt dass $\lambda_1 = \lambda_2 = \lambda_3 = 0$. Es liegt lineare Unabhängigkeit vor.

\ttask\ linear abhängig da
\[ a_n + b_n - 2c_n = \begin{cases} 1 + (-1)^n - 2 &\quad n\ \text{gerade} \\ 1 + (-1)^n + 0 &\quad n\ \text{ungerade} \end{cases} = \begin{cases} 1 + 1 - 2 &\quad n\ \text{gerade} \\ 1 -1  &\quad n\ \text{ungerade} \end{cases} = 0 \]
\setcounter{taski}{19}
\task

\ttask

$(v_n)_{n \in \mathbb{N}},\ (w_n)_{n \in \mathbb{N}}$ sind linear unabhängig, da
\[ \nexists \lambda \in \mathbb{R} : v_1 = 1 = \lambda \cdot 0 = \lambda w_1 \]
$((v_n)_{n \in \mathbb{N}},\ (w_n)_{n \in \mathbb{N}})$ stellt ein ES dar, wenn
\begin{theorem}
	\[ \forall (a_n)_{n \in \mathbb{N}} \in F\ \forall n \in \mathbb{N} : a_n = a_1v_n + a_2w_n \]
\end{theorem}
\begin{proof}
\ durch vollständige Induktion
\paragraph*{Induktionsanfang}\ $n = 1 \qquad a_1 = a_1v_1 + a_2w_1 \quad,\quad a_2 = a_2v_2 + a_2w_2$
\paragraph*{Induktionsbehauptung}\ $a_{n+1} = a_1v_{n+1} + a_2w_{n+1} \quad,\quad a_n = a_1v_n + a_2w_n$
\paragraph*{Induktionsschritt} Z. z. $\quad a_{n+2} = a_1v_{n+2} + a_2w_{n+2}$
\begin{align*}
	a_{n+2} &= a_{n+1} + a_{n} =  a_1v_{n+1} + a_2w_{n+1} + a_1v_n + a_2w_n = a_1(v_{n+1} + v_n) + a_2(w_{n+1} + w_n) \\
	&= a_1v_{n+2} + a_2w_{n+2}
\end{align*}
\end{proof}
Damit bilden die Vektoren ein ES und sind linear unabhängig, womit sie eine Basis bilden.

\ttask
\begin{align*}
	r^{n+2} &= r^{n+1} + r^n \tag{$r^n \neq 0$}\\
	r^2 &= r + 1 \\
	0 &= r^2 - r - 1 \\
	r &= \frac{1 \pm \sqrt{5}}{2} \\
	r_1 = \phi \coloneqq \frac{1+\sqrt{5}}{2} \quad&\quad r_2 = \psi \coloneqq \frac{1-\sqrt{5}}{2}
\end{align*}

\ttask

Es gilt mit
\begin{gather*}
	\mu_1 \coloneqq \frac{5-\sqrt{5}}{10} \qquad \mu_2 \coloneqq \frac{5+\sqrt{5}}{10} \\
	\\
	\forall n \in \mathbb{N}: v_n = \mu_1\phi^n - \mu_2\psi^n \qquad \Rightarrow \qquad (v_n)_{n \in \mathbb{N}} = \mu_1(\phi^n)_{n \in \mathbb{N}} - \mu_2(\psi^n)_{n \in \mathbb{N}}\\
	\forall n \in \mathbb{N}: w_n = \mu_1\phi^n + \mu_2\psi^n \qquad \Rightarrow \qquad (w_n)_{n \in \mathbb{N}} = \mu_1(\phi^n)_{n \in \mathbb{N}} + \mu_2(\psi^n)_{n \in \mathbb{N}}
\end{gather*}
Somit ist jeder Vektor der Basis $B = ((v_n)_{n \in \mathbb{N}},\ (w_n)_{n \in \mathbb{N}})$ als Linearkombination des ES $((\phi^n)_{n \in \mathbb{N}},\ (\psi^n)_{n \in \mathbb{N}})$ darstellbar. Da beide Systeme die gleiche Anzahl von Vektoren haben, handelt es sich ebenfalls um eine Basis.

\ttask\ 
Gesucht sind $\lambda_1,\ \lambda_2 \in \mathbb{R}$ sodass
\begin{align*}
	\lambda_1\phi + \lambda_2\psi &= 1 & \lambda_1\phi^2 + \lambda_2\phi^2 &= 1 \\
	&& \lambda_1\phi + \lambda_1 + \lambda_2\psi + \lambda_2 &= 1 \\
	&& \lambda_1 + \lambda_2 &= 0 \\
	\lambda_1(\phi - \psi) &= 1
\end{align*}
\begin{gather*}
	\lambda_1 = \frac{1}{\phi - \psi} \qquad \lambda_2 = -\frac{1}{\phi-\psi}\\
\intertext{Formel von \textsc{Bernoulli}}
	 c_n = \frac{\phi^n - \psi^n}{\phi - \psi} 	
\end{gather*}
\end{document}